\documentclass[12pt,oneside,english,brazil,lmodern,siglas,simbolos,cite=num]{ucsmonograph}

\ifluatex
	\usepackage{fontspec}
\else
	\usepackage[utf8]{inputenc}
	\usepackage[T1]{fontenc}
\fi

\usepackage[brazil]{babel}
\usepackage{graphicx} % Inserção de imagens
\graphicspath{{../Imagens/}} % Definição do caminho para as imagens
\usepackage{amsmath} % Digitação de alguns símbolos matemáticos.
\usepackage{bigints} % Inclui símbolos de integral grandes, para frações etc.
\usepackage{booktabs} % Recursos para formatar tabelas
\usepackage{bm} % Bold Math
\usepackage[framed]{matlab-prettifier}
\usepackage{todonotes}
\setlength\heavyrulewidth{1.5pt}
\setlength\lightrulewidth{0.5pt}
\includeonly{PreTexto} % Inclusão do arquivo de elementos pré-textuais sem deixar páginas em branco

\newcommand{\unidademassa}{[kg]}
\newcommand{\unidademassalinear}{[kg/m]}
\newcommand{\unidadeamortecimento}{[N\cdot s/m]}
\newcommand{\unidaderigidez}{[N/m]}
\newcommand{\unidadeforca}{[N]}
\newcommand{\unidadeposicao}{[m]}
\newcommand{\unidadevelocidade}{[m/s]}
\newcommand{\unidadeaceleracao}{[m/s\textsuperscript{2}]}
\newcommand{\unidadetempo}{[s]}
\newcommand{\unidadeenergia}{[J]}
\newcommand{\unidadeangulo}{[rad]}
\newcommand{\unidadevelocidadeangular}{[rad/s]}
\newcommand{\unidadefrequencia}{[Hz]}
\newcommand{\unidadetorque}{[N\cdot m]}
\newcommand{\unidadearea}{[m\textsuperscript{2}]}
\newcommand{\semunidade}{[-]}
\newcommand{\unidadepressao}{[Pa]}
\newcommand{\unidadesegundomomento}{[m\textsuperscript{4}]} % Arquivo que contém os símbolos.

\titulo{Modelagem numérica de vibração em rolamentos defeituosos}
\autor{Henrique Baron}
\data{2018}
\instituicao{Universidade de Caxias do Sul}
\local{Caxias do Sul}
\preambulo{Trabalho de conclusão de curso apresentado à Universidade de Caxias do Sul como requisito parcial à obtenção do grau de Engenheiro Mecânico.
Área de concentração: Projetos de Máquinas: Estática e Dinâmica Aplicada.}
\orientador{Prof. Me. Paulo Roberto Linzmaier}
\palavraschave{Vibra\c{c}\~{a}o. Modelagem. Defeito. Rolamento.}
\areadoconhecimento{Área do conhecimento de ciências exatas e engenharias}
\avaliadori{Prof. Me. Vagner Grison}
\avaliadorii{Prof. Dr. Marcos Alexandre Luciano}

\incluirsiglas{Siglas} % Inclui o arquivo bibtex com as siglas.
\incluirsimbolos{Simbolos} % Inclui o arquivo bibtex com os símbolos.

\begin{document}
	% Capa e folha de rosto
\imprimircapa
\imprimirfolhaderosto
\clearpage

% Folha de aprovação
\imprimirfolhadeaprovacao{29/11/2018}

% Agradecimentos
\begin{agradecimentos}
%	Gostaria de agradecer primeiramente à minha família pelo enorme suporte e incentivo ao desenvolvimento deste trabalho, em especial nos momentos de maior dificuldade, no qual fizeram tudo o que estava ao seu alcance para me ajudar a superar os problemas que enfrentava.
%	A eles e à minha namorada, Fernanda, agradeço a compreensão e paciência durante este período extenuante de estudo e entrega.
%	
%	Às professoras Dra. Kátia Cavalca e Dra. Isolda Gianni de Lima, agradeço pela prontidão e disposição em me ajudar nos momentos em que eu não pude encontrar uma resposta para minhas dúvidas.
%	
%	Agradeço também à empresa Auttom Automação e Robótica, por propiciar a oportunidade de desenvolvimento pessoal com o tema deste trabalho, e pelo suporte que sempre deu à condução dos meus estudos.
\end{agradecimentos}

% Resumo em português
\begin{resumo}
	\SingleSpacing
	O emprego de mancais de rolamento se aplica desde sistemas mecânicos simples até o maquinário de alta complexidade e precisão.
	Sendo a análise de vibrações o meio utilizado para identificação precoce de falhas nestas unidades, é de grande importância determinar o comportamento deste componente em situação de defeito.
	Com este foco, este trabalho concentra-se no desenvolvimento de um modelo numérico não-linear de três graus de liberdade para a resposta de vibração de um rolamento com um defeito pontual no anel externo.
	Para isso, são empregados os conceitos da teoria \emph{hertziana} de deformação por contato, de modo a determinar a força sobre um elemento rolante quando este incide sobre o defeito.
	O valor de força é utilizado para modelar três perfis de pulso diferentes para a carga de impacto entre a esfera do rolamento e o defeito.
	Ao mesmo tempo, a influência do filme de fluido lubrificante no contato entre esferas e anéis é analisado sob os fundamentos da teoria de lubrificação \emph{elastohidrodinâmica}.
	O modelo desenvolvido é investigado quanto à sua estabilidade e convergência, observando o custo computacional das diferentes formas de pulso para a carga de impacto.
	A resposta de vibração simulada é comparada a dados de simulações e verificações experimentais de trabalhos anteriores no mesmo campo de aplicação, analisando-a no domínio do tempo e no domínio da frequência, para concluir quanto à consistência qualitativa do modelo.
	\vspace{\onelineskip}
	
	\noindent
	\textbf{Palavras-chave}: Vibração. Modelagem. Defeito. Rolamento.
\end{resumo}


% Resumo em inglês
\begin{resumo}[\normalsize\bfseries ABSTRACT]
	\SingleSpacing
	\begin{otherlanguage}{english}
		The application of rolling element bearings ranges from simple mechanical systems to high complexity and precision machinery.
		Since the vibration analysis is the main way to early identify failure in these units, it is of great importance to determine the vibrational pattern of this component in a defect situation.
		An experimental approach for the analysis would offer a strict condition for the problem investigation.
		For this reason, the construction of a numeric model is considered to be a more extensive way to understand the phenomenon, while still eliminating interferences caused by the test system.
		With that in focus, the present work has as objective to model the vibration of damaged rolling element bearings and identify the main characteristics of this behavior.
		\vspace{\onelineskip}
		
		\noindent
		\textbf{Keywords}: Vibration. Model. Defect. Bearing.
	\end{otherlanguage}
\end{resumo}
% Listas de figuras, quadros, tabelas e siglas
\listoffigures*
\cleardoublepage

%\listofquadros*
%\cleardoublepage

\listoftables*
\cleardoublepage

\listofsiglas*
\cleardoublepage

\listofsimbolos*
\cleardoublepage

% Sumário
\tableofcontents*
\cleardoublepage

	\textual % Comando que define que os elementos textuais começaram, para incluir a numeração de páginas.
	
	\chapter{Introdução}
	Em todo sistema produtivo em que se faz o uso de ferramentas ou máquinas, a manutenção é um tema que recebe grande atenção, dada a necessidade de manter o funcionamento de equipamentos com o menor tempo possível de parada.
	Aliado a isso, busca-se continuamente encontrar um balanço entre o custo de repetidas paradas para inspeção e o de uma eventual falha ocasionada por falta de acompanhamento.
	O montante gasto em manutenção pelas empresas no Brasil no ano de 2013 foi equivalente a 4,69\% do PIB nacional, correspondente a um valor de R\$ 206,5 bilhões \cite{seleme:2015}.
	
	Dentre as metodologias de manutenção existentes, há uma divisão em três grupos principais: corretiva, preditiva e preventiva.
	Este último, que se configura pelo planejamento de uma rotina de manutenção e inspeções baseadas em intervalos regulares de tempo, ainda é a mais comum na indústria atualmente.
	Entretanto, a realização de manutenção baseada no tempo vem mostrando uma baixa confiabilidade nos últimos anos.
	Em um teste realizado pelo SKF Group no qual 30 rolamentos de esferas idênticos foram testados até a falha, o tempo de vida das unidades variou de 15 horas até mais de 300 horas \cite{hashemian:2011}.
	 
	Da mesma forma, dados do \gls{sig:NREL} dos Estados Unidos apontam que 76\% dos casos de falha em caixas de engrenagens de aerogeradores são causados por rolamentos.
	Ao mesmo tempo, apenas 10\% dos rolamentos atingem a sua vida útil esperada por conta da diversidade de condições de desgaste prematuro às quais estes componentes podem ser submetidos \cite{peeters:2018}.
	Sendo assim, mesmo com uma expectativa de vida estimada para este componente, a determinação exata de sua durabilidade é impossível.
	
	Para o monitoramento da condição de vida dos rolamentos, a medição de vibração traz a vantagem de fornecer sinais de mudança de comportamento e possíveis falhas em um estágio inicial \cite{al-najjar:2004}.
	A identificação dos defeitos baseia-se no reconhecimento de padrões de frequência e amplitude observados em medições realizadas durante o funcionamento do equipamento \cite{carden:2004}.
	Estes padrões podem ser associados a parâmetros de operação da máquina, e assim fornecer indícios da causa da falha como, por exemplo, o componente do rolamento em que o problema se evidencia.
	
	\section{Justificativa}\label{sec:justificativa}
	Dada a larga utilização de rolamentos em aplicações de engenharia, é essencial o conhecimento sobre o padrão de vibração provocado pelo componente em situação de defeito.
	Determinar o comportamento do sinal lido em tal condição permite identificar de fato a presença de uma falha em seu estágio inicial, oferecendo a possibilidade de programar uma troca dessa unidade muito antes do seu colapso.
	
	Entre as possibilidades para identificar o comportamento de rolamentos que apresentam falhas, pode-se mencionar a modelagem numérica ou a avaliação puramente experimental.
	No presente trabalho, considera-se o desenvolvimento de um modelo numérico, visto que este permite identificar de maneira aprofundada a origem de cada parte do sinal obtido, permitindo ainda expandir a análise para outras situações de defeito, carga e velocidades de rotação.
	Tal expansão da investigação em uma avaliação puramente experimental seria dispendiosa e muito suscetível a variáveis externas relativas ao sistema de teste.
	
	\section{Ambiente de desenvolvimento}	
	O trabalho foi conduzido no departamento de Pesquisa e Desenvolvimento da unidade de educação da empresa Auttom Automação e Robótica Ltda., situada em Caxias do Sul.
	A empresa possui duas divisões: uma concentrada em projetos de automação sob demanda; e a outra, onde o trabalho foi realizado, desenvolve bancadas didáticas e sistemas de ensino para escolas técnicas e universidades.
	
	Atualmente, a unidade de educação da empresa conta com uma linha ampla de produtos nas áreas de automação industrial, energias renováveis, eletricidade e refrigeração.
	Ainda não há uma oferta considerável de produtos para estudo de mecânica nas áreas de estática ou dinâmica, e os dados obtidos com este trabalho devem ser utilizados no desenvolvimento de equipamento didático específico para ensino de vibrações em máquinas.
	
	\section{Objetivos}
	
	\subsection{Objetivo geral}
	O objetivo do presente trabalho é modelar numericamente a vibração provocada por rolamentos defeituosos.
	
	\subsection{Objetivos específicos} \label{sec:objetivos:especificos}
	O objetivo apresentado é distribuído entre os seguintes objetivos específicos:
	\begin{enumerate}
		\item determinar as propriedades do rolamento a ser modelado;
		\item definir as características da força de excitação provocada pelo defeito no componente;
		\item resolver o modelo dinâmico para a vibração do rolamento defeituoso;
		\item empregar recursos de análise compatíveis com as referências consultadas para a comparação dos resultados;
		\item obter um modelo numérico que retorne um comportamento concordante com a literatura consultada.
	\end{enumerate}

	\chapter{Referencial teórico}
	No capítulo a seguir serão apresentados os conceitos inerentes à modelagem numérica e análise de sistemas dinâmicos, empregados no desenvolvimento da simulação proposta.
	
	\section{Análise harmônica}
	Qualquer movimento que se repita em intervalos de tempo iguais é denominado movimento periódico \cite{rao:2008}.
	Dentre eles, entende-se por movimento harmônico simples aquele que é definido por uma função do tipo \cite{timoshenko:1974}
	\begin{equation}
		g(t) = a\cos\omega \gls{s:tempo} + b\sin\omega t
	\end{equation}
	onde $g(t)$ indica a posição ou deslocamento, \gls{s:freq-rad} é a frequência da oscilação, \gls{s:tempo} denota o tempo e $ a $ e $ b $ são duas constantes quaisquer.
	
	Embora este seja o tipo de movimento mais simples de tratar, muitos dos sistemas vibratórios não exibem oscilação harmônica \cite{rao:2008}.
	No entanto, uma função periódica pode ser representada em termos da soma de suas componentes harmônicas através da série de Fourier \cite{clark:1972}.
	
	\subsection{Série de Fourier} \label{sec:fourier:serie}
	Seja $g(t)$ uma função periódica com período \gls{s:periodo}, a série de Fourier é definida como \cite{spiegel:1977}
	\begin{equation}\label{eqn:fourier:serie}
		g(t) = \frac{a_0}{2} + \sum_{n=1}^{\infty}\left(a_n\cos n\gls{s:freq-rad} t + b_n\sin n\gls{s:freq-rad} t \right),
	\end{equation}
	sendo a frequência \gls{s:freq-rad} relacionada ao período \gls{s:periodo} por \cite{dimarogonas:1995}
	\begin{equation}\label{eqn:fourier:periodo}
		\gls{s:freq-rad} = \frac{2\pi}{\gls{s:periodo}}
	\end{equation}
	e os coeficientes \glspl{s:coef-fourier} representam as harmônicas \cite{dimarogonas:1995}, que valem \cite{spiegel:1977}
	\begin{align}\label{eqn:fourier:coefs}
		a_0 &= \frac{2}{\gls{s:periodo}}\int_{0}^{\gls{s:periodo}}g(t)\ dt \notag\\
		a_n &= \frac{2}{\gls{s:periodo}}\int_{0}^{\gls{s:periodo}}g(t)\cos n\gls{s:freq-rad} t\ dt \\
		b_n &= \frac{2}{\gls{s:periodo}}\int_{0}^{\gls{s:periodo}}g(t)\sin n\gls{s:freq-rad} t\ dt. \notag
	\end{align}
	
	Embora seja possível a discussão sobre a convergência da série Fourier na Equação \ref{eqn:fourier:serie} para uma função $g(t)$, isso não é tratado aqui, visto que as condições de convergência desta série são satisfeitas nos problemas de ciência e engenharia, em geral \cite{spiegel:1977}.
	
	Utilizando as identidades de Euler
	\begin{equation}\label{eqn:fourier:ident-euler}
		e^{i\gls{s:freq-rad}} = cos\,\gls{s:freq-rad} + i\sin\gls{s:freq-rad}, \quad e^{-i\gls{s:freq-rad}} = cos\,\gls{s:freq-rad} - i\sin\gls{s:freq-rad}
	\end{equation}
	a série de Fourier da Equação \ref{eqn:fourier:serie} também pode ser representada na sua forma complexa como \cite{spiegel:1977}
	\begin{equation}\label{eqn:fourier:serie-compl}
		g(t) = \sum_{n=-\infty}^{\infty}\alpha_n\,e^{in\gls{s:freq-rad} t},
	\end{equation}
	onde a harmônica \gls{s:coef-fourier-compl} é calculada por \cite{dimarogonas:1995}
	\begin{equation}\label{eqn:fourier:coefs-compl}
		\alpha_n = \frac{1}{2}(a_n - ib_n) = \frac{1}{\gls{s:periodo}}\int_{0}^{\gls{s:periodo}}g(t)\,e^{-in\gls{s:freq-rad} t}\ dt.
	\end{equation}
	
	\subsection{Representação no domínio da frequência} \label{sec:espectros-frequencia}
	A série de Fourier permite a representação de uma função periódica no domínio da frequência \cite{rao:2008}.
	A Figura \ref{fig:funcao-periodica} mostra, por exemplo, o gráfico de $ g(t) = \sin2\pi t + 0,\!7\cos4\pi t + 2 $ no domínio do tempo.
	Pode-se aplicar a Equação \ref{eqn:fourier:coefs-compl} um número determinado de vezes para representar a função no domínio da frequência com um espectro de dois lados \cite{dimarogonas:1995}.
	Neste gráfico, mostrado na Figura \ref{fig:espectro-2sided}, o eixo horizontal representa os múltiplos inteiros $ n $ da frequência \gls{s:freq-rad} utilizada na Equação \ref{eqn:fourier:coefs-compl}, e o eixo vertical é o próprio valor da harmônica \gls{s:coef-fourier-compl} calculada.
	\begin{figure}[ht]
		\incluirimagem{FourierFuncao.png}{Função periódica no domínio do tempo}{o autor (\thedate)}
		\label{fig:funcao-periodica}
	\end{figure}

	\begin{figure}[ht]
		\incluirimagem{FourierEspectro2sided.png}{Espectro de dois lados de uma função periódica}{o autor (\thedate)}
		\label{fig:espectro-2sided}
	\end{figure}
	
	De maneira semelhante, o espectro de frequência da Figura \ref{fig:espectro-2sided} pode ser representado por um espectro de frequências de um lado como mostra a Figura \ref{fig:espectro-1sided}, onde a harmônica \gls{s:coef-fourier-compl} de ordem $ n $ é somada com a harmônica de ordem $ -n $ \cite{randall:1987}.
	É importante observar que a componente constante, isto é, $ n=0 $ permanece com o mesmo valor do espectro de dois lados.
	\begin{figure}[t]
		\incluirimagem{FourierEspectro1sided.png}{Espectro de um lado de uma função periódica}{o autor (\thedate)}
		\label{fig:espectro-1sided}
	\end{figure}
	
	\subsection{Integral de Fourier}
	Para o caso de uma função não-periódica, toma-se a definição formal de que $ \tau\to\infty $, fazendo com que a série de Fourier torne-se uma integral de Fourier \cite{spiegel:1977}.
	Seja $g(t)$ uma função seccionalmente contínua em qualquer intervalo finito e absolutamente integrável em $ (-\infty ,\,\infty) $, a integral de Fourier é definida como \cite{spiegel:1977}
	\begin{equation}\label{eqn:fourier:integral}
		g(t) = \int_{0}^{\infty}\left\lbrace a_{\gls{s:freq-rad}}\cos\gls{s:freq-rad} t + b_{\gls{s:freq-rad}}\sin\gls{s:freq-rad} t\right\rbrace\, d\gls{s:freq-rad},
	\end{equation}
	tal que os coeficientes \glspl{s:coef-fourier-int} são determinados por
	\setlength\jot{3ex plus 2ex minus 1ex}
	\begin{align}
		a_\omega = \frac{1}{\pi}\int_{-\infty}^{\infty}g(t)\cos\gls{s:freq-rad} t\ dt \\
		b_\omega = \frac{1}{\pi}\int_{-\infty}^{\infty}g(t)\sin\gls{s:freq-rad} t\ dt.
	\end{align}
	
	\subsubsection{Transformada de Fourier} \label{sec:fourier:transform}
	Aplicando na Equação \ref{eqn:fourier:integral} a identidade de Euler da Equação \ref{eqn:fourier:ident-euler} de maneira semelhante ao item \ref{sec:fourier:serie}, obtém-se a transformada de Fourier \cite{savi:2017} de $g(t)$, denotada por $G(\omega)$
	\begin{equation} \label{eqn:fourier:transform}
		G(\gls{s:freq-rad}) = \mathcal{F}\{g(t)\} = \int_{-\infty}^{\infty}g(t)\ e^{-i\gls{s:freq-rad} t}\:dt,
	\end{equation}
	que também é denominada como transformada contínua de Fourier \cite{dimarogonas:1995}.
	A transformada inversa de Fourier, por sua vez, é dada por \cite{spiegel:1977}
	\begin{equation} \label{eqn:fourier:transform-inv}
		g(t) = \mathcal{F}^{-1}\{G(\gls{s:freq-rad})\} = \frac{1}{2\pi}\int_{-\infty}^{\infty}G(\gls{s:freq-rad})\ e^{i\gls{s:freq-rad} t}\:d\gls{s:freq-rad}.
	\end{equation}
	
	\subsection{Métodos numéricos}
	Na transformada de Fourier apresentada no item \ref{sec:fourier:transform}, a definição matemática considera o fenômeno de oscilação acontecendo em um intervalo de tempo contínuo.
	É muito comum nos instrumentos de medição modernos, no entanto, que os dados sejam descritos em um intervalo de tempo discreto.
	Isto é, a função -- de vibração, por exemplo -- é representada por uma série de tempo, uma sequência de valores em pontos discretos equidistantes, sendo chamada também por função amostrada no tempo \cite{dimarogonas:1995}.
	Uma vez que os sinais obtidos nos sistemas atuais são representados discretamente, os métodos de análise também devem ser adaptados a esse modo de processamento.
	
	\subsubsection{Funções amostradas no tempo}
	O par de transformadas integrais das Equações \ref{eqn:fourier:transform} e \ref{eqn:fourier:transform-inv} vai do intervalo $ -\infty $ até $ \infty $, o que torna impossível a sua manipulação numérica.
	Um intervalo $ (-\gls{s:periodo}/2,\:\gls{s:periodo}/2) $ é empregado, assumindo que a função $g(t)$ se repita fora desse intervalo em ambas as direções, e permitindo o uso da frequência fundamental \gls{s:freq-hz} que corresponde a \cite{dimarogonas:1995}
	\begin{equation} \label{eqn:fourier:freq-fundam}
		\gls{s:freq-hz} = \frac{1}{\gls{s:periodo}}.
	\end{equation}
	
	Considerando que a amostragem da função $g(t)$ seja feita em intervalos $ \Delta $\gls{s:tempo}, a frequência de amostragem \gls{s:freq-amostra} é definida como \cite{dimarogonas:1995}
	\begin{equation} \label{eqn:fourier:freq-amostra}
		\gls{s:freq-amostra} = \frac{1}{\Delta t}.
	\end{equation}
	
	Considerando ainda que alguma perturbação aconteça a cada intervalo \[\Delta t = \frac{\gls{s:periodo}}{\gls{s:numero-amostras}} \] onde \gls{s:numero-amostras} corresponde ao número de amostras coletadas para o período \gls{s:periodo}, é de se esperar que exista uma frequência inerente no espectro \gls{s:freq-amostra}, o que introduz no espectro de frequências uma periodicidade \gls{s:freq-amostra} \cite{dimarogonas:1995}.
	
	\subsubsection{Transformada Discreta de Fourier}
	Considerando que a função $g(t)$ seja discretizada e truncada tanto no domínio como na frequência de modo que
	\begin{align*}
		t_n = n\Delta t\: ,&\quad n = 0, 1, \dots , \gls{s:numero-amostras}\\
		\mathit{f} = k\Delta\mathit{f} = \frac{k}{\gls{s:numero-amostras}\Delta t}\: ,&\quad k = 0, 1, \dots , \gls{s:numero-amostras}
	\end{align*}
	a transformada de Fourier (direta e inversa) torna-se \cite{randall:1987}
	\begin{align}
		G_k &= \frac{1}{\gls{s:numero-amostras}}\sum_{n=0}^{\gls{s:numero-amostras}-1}g_n e^{-i 2\pi k n/\gls{s:numero-amostras}} \label{eqn:fourier:dft}\\
		g_n &= \sum_{k=0}^{\gls{s:numero-amostras}-1}G_k e^{i 2\pi k n/\gls{s:numero-amostras}} \label{eqn:fourier:dft-inv},
	\end{align}
	que é chamada de \gls{sig:DFT} e, por substituir as integrais contínuas e infinitas das Equações \ref{eqn:fourier:transform} e \ref{eqn:fourier:transform-inv} por somas finitas, é muito mais adaptada à computação digital \cite{randall:1987}.
	Adicionalmente, a Equação \ref{eqn:fourier:dft} pode ser representada como \cite{dimarogonas:1995}
	\begin{equation} \label{eqn:fourier:dft-matricial}
		\vec{G} = \frac{1}{\gls{s:numero-amostras}}\gls{s:matriz-unitarios-dft}\vec{g}
	\end{equation}
	onde $ \vec{G} $ é um vetor contendo as \gls{s:numero-amostras} componentes complexas de frequência $ G_k $ , $ \vec{g} $ é um vetor contendo as \gls{s:numero-amostras} amostras coletadas no tempo, e \gls{s:matriz-unitarios-dft} é uma matriz quadrada $ \gls{s:numero-amostras}\! \times\!	\gls{s:numero-amostras} $ contendo os vetores unitários \gls{s:vetor-matriz-unit-dft} que dependem unicamente do número de amostras e são calculados por
	\begin{equation}
		\textbf{a}_{kn} = e^{-i2\pi kn/\gls{s:numero-amostras}}.
	\end{equation}
	
	\subsubsection{Transformada Rápida de Fourier}
	Com larga aplicação nos instrumentos de análise atuais, a \gls{sig:FFT} é um algoritmo para obtenção da \gls{sig:DFT} que reduz consideravelmente o número de operações em relação ao método tradicional \cite{randall:1987}.
	Ao passo de que a resolução da Equação \ref{eqn:fourier:dft-inv} -- que obtém as componentes de frequência da \gls{sig:DFT} -- precisa de um total de $ \gls{s:numero-amostras}^2 $ computações aritméticas, o método da \gls{sig:FFT} fornece o resultado em menos de $ 2\gls{s:numero-amostras} \log_2\! \gls{s:numero-amostras} $ operações \cite{cooley:1965}.
	Para um caso em que \gls{s:numero-amostras} representa 1024 amostras, essa redução é da ordem de 100 vezes \cite{randall:1987}.	
	
	O algoritmo envolve o reordenamento e fatoração da matriz complexa \gls{s:matriz-unitarios-dft} em matrizes que produzem rotações progressivamente menores nos vetores unitários \cite{randall:1987}, e oferece grandes vantagens quando aplicado amostras cuja quantidade é uma potência inteira de dois, devido à natureza binária das operações computacionais \cite{cooley:1965}.
	Embora a abordagem completa do procedimento fuja do escopo deste texto, cabe ressaltar que o resultado obtido e as limitações envolvidas são os mesmos relativos ao cálculo da \gls{sig:DFT} \cite{randall:1987}.
	
	\section{Vibração de sistemas discretos} \label{sec:disc}
	Denominam-se sistemas discretos aqueles que são descritos por um número finito de graus de liberdade.
	Embora grande parte das estruturas e máquinas possuam elementos elásticos e tenham, portanto, infinitos graus de liberdade, é comum discretizá-los através da divisão dos corpos rígidos em um número conhecido e distribuído de massas pontuais, para simplificação do problema \cite{rao:2008}.
	
	\subsection{Equação de movimento}
	O movimento linear de um sistema massa-mola-amortecedor qualquer é expresso por \cite{rao:2008}
	\begin{equation} \label{eqn:discr:movimento}
		\gls{s:matriz-massa}\,\ddot{\vec{x}} + \gls{s:matriz-amort}\,\dot{\vec{x}} + \gls{s:matriz-rigidez}\,\vec{x} = \vec{F}
	\end{equation}
	onde \gls{s:matriz-massa}, \gls{s:matriz-amort} e \gls{s:matriz-rigidez} representam as matrizes de massa, amortecimento e rigidez, respectivamente, e são dadas por
	\begin{gather} 
		\gls{s:matriz-massa} = 
		\begin{bmatrix} \label{eqn:matr:massa}
			m_{11} & m_{12} & m_{13} & \dots & m_{1n}\\
			m_{21} & m_{22} & m_{23} & \dots & m_{2n}\\
			\vdots\\
			m_{n1} & m_{n2} & m_{n3} & \dots & m_{nn}\\
		\end{bmatrix}\\
		\gls{s:matriz-amort} = 
		\begin{bmatrix} \label{eqn:matr:amort}
			c_{11} & c_{12} & c_{13} & \dots & c_{1n}\\
			c_{21} & c_{22} & c_{23} & \dots & c_{2n}\\
			\vdots\\
			c_{n1} & c_{n2} & c_{n3} & \dots & c_{nn}\\
		\end{bmatrix}\\
		\gls{s:matriz-rigidez} =
		\begin{bmatrix} \label{eqn:matr:rigid}
			k_{11} & k_{12} & k_{13} & \dots & k_{1n}\\
			k_{21} & k_{22} & k_{23} & \dots & k_{2n}\\
			\vdots\\
			k_{n1} & k_{n2} & k_{n3} & \dots & k_{nn}\\
		\end{bmatrix},
	\end{gather}
	sendo \gls{s:coef-massa}, \gls{s:coef-amortecimento} e \gls{s:coef-rigidez} são os coeficientes de massa, amortecimento e rigidez da matriz correspondente.	Além destes, os termos \gls{s:vetor-desloc}, \gls{s:vetor-vel}, \gls{s:vetor-acel} e \gls{s:vetor-forcas} são os vetores de deslocamento, velocidade, aceleração e força, definidos por \cite{rao:2008}
	\begin{align} \label{eqn:vetores}
		\vec{x} =
		\begin{Bmatrix}
			x_1(t)\\ x_2(t)\\ \vdots\\ x_n(t)\\
		\end{Bmatrix},\qquad &
		\dot{\vec{x}} = 
		\begin{Bmatrix}
			\dot{x}_1(t)\\ \dot{x}_2(t)\\ \vdots\\ \dot{x}_n(t)\\
		\end{Bmatrix}, \notag\\
		\ddot{\vec{x}} = 
		\begin{Bmatrix}
			\ddot{x}_1(t)\\ \ddot{x}_2(t)\\ \vdots \\ \ddot{x}_n(t)\\  
		\end{Bmatrix},\qquad &
		\vec{F} = 
		\begin{Bmatrix}
			F_1(t)\\ F_2(t)\\ \vdots\\ F_n(t)\\
		\end{Bmatrix}.
	\end{align}
	
	\section{Vibração de sistemas contínuos}
	Em contraste aos sistemas discretos, os sistemas contínuos são aqueles em que não se consegue identificar massas, molas e amortecedores pontuais na vibração de um corpo \cite{rao:2008}.
	Para estes casos, é possível determinar o comportamento vibratório de maneira analítica e sem discretização.
	Através dessa abordagem podem ser modeladas barras, eixos, cabos, vigas e outros elementos \cite{timoshenko:1974}.
	
	Ao tratar um corpo como um elemento elástico contínuo, ele é considerado como sendo composto de um número infinito de partículas infinitesimais, e é tratado como um sistema com infinitos gruas de liberdade.
	Dessa maneira, estruturas geometricamente complexas tornam-se demasiado difíceis -- ou mesmo impossíveis -- de serem modeladas por este método, restando para estes casos métodos que envolvam a discretização em um número finito de graus de liberdade \cite{timoshenko:1974}.
	
	\subsection{Equação de movimento}
	A equação geral de movimento de um sistema contínuo pode ser deduzida a partir da vibração vertical de uma corda sob tensão.
	Esteja uma corda com uma massa por unidade de comprimento equivalente a \gls{s:massa-espec-lin}, sujeita a uma força transversal \gls{s:forca} que é função do tempo \gls{s:tempo} e da coordenada $x$, como mostra o lado esquerdo da Figura \ref{fig:vibracao-corda}.
	O lado direito da figura mostra em detalhe o intervalo infinitesimal $dx$ entre os pontos $x_1$ e $x_2$.
	Nele também são esquematizados o deslocamento \gls{s:desloc-elastico} na direção $z$, a força de tração \gls{s:tensao-corda}, o ângulo de inclinação da corda em relação ao eixo $x$ denotado por $ \theta $ e, por fim, a força \gls{s:forca}$(x,t)$ atuante sobre o segmento infinitesimal de comprimento $ds$ da corda.
	\begin{figure}[t] 
	\incluirimagem{VibracaoCorda.png}{Vibração de uma corda tensionada}{adaptado de \citeauthoronline{rao:2008} (\citeyear{rao:2008})} \label{fig:vibracao-corda}
	\end{figure}

	Nesta dedução, desconsidera-se a ação da atmosfera e da gravidade sobre a corda, além de supor-se que não existem perdas de energia na sua vibração.
	Também define-se a corda vibrante como tendo uma seção transversal pequena em comparação com seu comprimento, de maneira que as tensões ao longo da área da seção variem pouco e se tornem puramente axiais, desprezando assim tensões de flexão e cisalhamento \cite{clark:1972}.
	
	Com estas considerações, o balanço de forças na direção $z$ para o elemento da Figura \ref{fig:vibracao-corda} é o somatório da força transversal \gls{s:forca} e da componente vertical da força de tração \gls{s:tensao-corda} \cite{rao:2008}.
	A força líquida corresponde à força de inércia atuante sobre o elemento, portanto
	\begin{equation} \label{eqn:cont:balanco}
		(P + dP)\;\sin(\theta + d\theta) + F\:dx - P\;\sin\:\theta =
		\gls{s:massa-espec-lin}\:dx\:\frac{\partial^2 \gls{s:desloc-elastico}}
		{\partial t^2}.
	\end{equation}
	
	Para um comprimento elementar $dx$ a força de tração $dP$ pode ser escrita como \[dP = \frac{\partial P}{\partial x}\:dx.\] 
	Empregando essa substituição e tomando a tensão \gls{s:tensao-corda} como constante ao longo do comprimento; fazendo a simplificação de $\sin\theta$ para um ângulo pequeno e considerando que a corda esteja em vibração livre, ou seja, tomando $\gls{s:forca}(x,t) = 0$, pode-se simplificar a Equação \ref{eqn:cont:balanco} para
	\begin{align}
		\mathit{c}^2\:\frac{\partial^2
		\gls{s:desloc-elastico}}{\partial x^2}=\frac{\partial^2
		\gls{s:desloc-elastico}}{\partial t^2} \label{eqn:cont:onda}\\
		\mathit{c} = \left(\frac{P}{\gls{s:massa-espec-lin}}\right)^{1/2}.
	\end{align}
	
	A Equação \ref{eqn:cont:onda} também é conhecida como equação de onda \cite{rao:2008}. 
	
	\section{Contato seco entre sólidos} \label{sec:contato-seco}
	A primeira solução para o contato entre duas superfícies sólidas foi apresentada por \citeauthoronline{hertz:1881} em 1881, e envolvia o cálculo de integrais elípticas e um fator de elipticidade \cite{hertz:1881}.
	Nesse trabalho, é provado que a área de contato entre dois corpos toma a forma de uma elipse, além de serem apresentadas soluções para o deslocamento de pontos em uma superfície, e as tensões envolvidas na deformação.
	Em termos da relação entre força e deformação, a teoria batizada de \emph{contato hertziano} apresenta um comportamento não-linear do deslocamento em resposta à carga aplicada, o que difere da rigidez linear empregada nas equações do item \ref{sec:disc}.
	
	\subsection{Curvatura das superfícies de contato} \label{sec:curvatura-contato}
	De maneira geral, pode-se exemplificar o contato entre duas superfícies com dois elipsoides, como mostra a Figura \ref{fig:contato-elipses}.
	Adota-se o eixo x como paralelo à direção do movimento, sendo também o semieixo menor da elipse de contato \cite{hamrock:1991}.
	\begin{figure}[ht]
		\incluirimagem{ContatoEliptico.png}{Contato entre duas superfícies elípticas}{adaptado de \citeauthoronline{hamrock:1991} (\citeyear{hamrock:1991})}
		\label{fig:contato-elipses}
	\end{figure}

	Em rolamentos de esferas, utiliza-se para os raios de curvatura nas coordenadas x e y as medidas indicadas na Figura \ref{fig:raios-rolamento} para a esfera e os anéis interno e externo.
	Ainda de acordo com a mesma figura, adota-se a esfera como a superfície ''a'' e o anel como a superfície ''b'' do respectivo contato.
	\begin{figure}[b]
		\incluirimagem{RaiosRolamento.png}{Raios de curvatura utilizados em rolamentos de esferas}{o autor (\thedate)}
		\label{fig:raios-rolamento}
	\end{figure}

	Para a modelagem do contato entre a esfera e cada anel, deve-se determinar os raios de curvatura equivalentes \gls{s:raio-equiv-x} e \gls{s:raio-equiv-y}, dados por
	\begin{gather}
		\frac{1}{\gls{s:raio-equiv-x}} = \frac{1}{r_{ax}} +
		\frac{1}{r_{bx}} \label{eqn:seco:r-equiv-x} \\
		\frac{1}{\gls{s:raio-equiv-y}} = \frac{1}{r_{ay}} +
		\frac{1}{r_{by}} \label{eqn:seco:r-equiv-y},
	\end{gather}
	que redefinem o problema do contato entre dois elipsoides -- como mostrado na Figura \ref{fig:contato-elipses} -- para o contato de um elipsoide de raios de curvatura \gls{s:raio-equiv-x} e \gls{s:raio-equiv-y} com uma superfície plana \cite{hamrock:1991}.
	
	Outros dois parâmetros importantes utilizados nos cálculos posteriores são a soma de curvatura \gls{s:curvatura-soma}, definida como \cite{wijnant:1998}
	\begin{equation} \label{eqn:seco:curv-soma}
		\frac{1}{\gls{s:curvatura-soma}} =
		\frac{1}{\gls{s:raio-equiv-x}} + \frac{1}{\gls{s:raio-equiv-y}}
	\end{equation}
	e a diferença de curvatura \gls{s:curvatura-diferenca}, que vale
	\begin{equation} \label{eqn:seco:curv-diferenca}
		\gls{s:curvatura-diferenca} = \gls{s:curvatura-soma}
		\left( \frac{1}{\gls{s:raio-equiv-x}} - 
		\frac{1}{\gls{s:raio-equiv-y}} \right).
	\end{equation}
	
	Cabe ressaltar que, diferentemente da soma de curvatura, a diferença de curvatura é adimensional.
	
	\subsection{Integrais elípticas} \label{sec:int-elipticas}
	Conforme proposto em \cite{hertz:1881}, a solução para o contato entre os dois sólidos depende do cálculo das integrais elípticas de primeiro e segundo tipos, além do parâmetro de elipticidade, denotados respectivamente por \gls{s:int-eliptica-1}, \gls{s:int-eliptica-2} e \gls{s:param-elipticidade}.
	
	Um método para a obtenção destes três parâmetros, proposto em \cite{hamrock:1991}, calcula as integrais elípticas como
	\begin{align}
		\gls{s:int-eliptica-1} &= \int_{0}^{\pi/2} \left[ 1 - \left( 1-\frac{1}{\gls{s:param-elipticidade}^2}\right)\sin^2\phi
		\right]^{-1/2}d\phi \label{eqn:seco:int-elip-1} \\
		\gls{s:int-eliptica-2} &= \int_{0}^{\pi/2} \left[ 1 - \left(
		1-\frac{1}{\gls{s:param-elipticidade}^2}\right)\sin^2\phi
		\right]^{1/2}d\phi \label{eqn:seco:int-elip-2},
	\end{align}
	e relaciona estes valores com \gls{s:param-elipticidade} e \gls{s:curvatura-diferenca} através de uma equação transcendente
	\begin{equation} \label{eqn:seco:param-elip}
		\gls{s:param-elipticidade} = \left[ \frac{
		2\gls{s:int-eliptica-1} - \gls{s:int-eliptica-2}(
		1+\gls{s:curvatura-diferenca})}{
		\gls{s:int-eliptica-2}(1-\gls{s:curvatura-diferenca})}
		\right]^{1/2},
	\end{equation}
	propondo a iteração destes cálculos até a convergência de \gls{s:param-elipticidade}, determinando em seguida os valores finais para \gls{s:int-eliptica-1} e \gls{s:int-eliptica-2}.
	
	\subsection{Deformação elástica por contato} \label{sec:deform-contato}
	Pela teoria de Hertz, a carga aplicada ao contato \gls{s:carga-radial} e a deformação máxima \gls{s:deformacao-maxima} são relacionadas por \cite{tandon:1997,hamrock:1991}
	\begin{equation} \label{eqn:seco:def-hertz-curta}
		\gls{s:carga-radial} = \gls{s:rigidez-hertz}\,
		\gls{s:deformacao-maxima}^{\gls{s:exp-naolin}}.
	\end{equation}
	
	Para o caso de rolamentos de esferas, o expoente \gls{s:exp-naolin} vale $ 1,5 $ e \gls{s:rigidez-hertz} é a rigidez não-linear, que vale \cite{hamrock:1991}
	\begin{equation} \label{eqn:seco:rig-hertz}
		\gls{s:rigidez-hertz} = \pi\gls{s:param-elipticidade}\,
		\gls{s:modulo-elast-efetivo}\left( \frac{
		2\gls{s:int-eliptica-2}\gls{s:curvatura-soma}}{
		9\gls{s:int-eliptica-1}^3} \right)^{1/2}.
	\end{equation}
	
	Nesta equação, o termo \gls{s:modulo-elast-efetivo} é o módulo de elasticidade efetivo, que combina o módulo de elasticidade \gls{s:modulo-elast} e o coeficiente de Poisson \gls{s:coef-poisson} de cada um dos dois sólidos a e b em contato através da fórmula \cite{wijnant:1998}
	\begin{equation}
		\frac{2}{\gls{s:modulo-elast-efetivo}} = 
		\frac{1-\gls{s:coef-poisson}_a}{\gls{s:modulo-elast}_a} + 
		\frac{1-\gls{s:coef-poisson}_b}{\gls{s:modulo-elast}_b}.
	\end{equation}
	
	Para o cálculo específico da deformação máxima \gls{s:deformacao-maxima}, as Equações \ref{eqn:seco:def-hertz-curta} e \ref{eqn:seco:rig-hertz} podem ser combinadas e rearranjadas para \cite{hamrock:1991}
	\begin{equation} \label{eqn:seco:deformacao-max}
		\gls{s:deformacao-maxima} = \gls{s:int-eliptica-1} \left[
		\left( \frac{9}{2\gls{s:int-eliptica-2} \gls{s:curvatura-soma}} \right) \left(
		\frac{\gls{s:carga-radial}}{\pi\gls{s:param-elipticidade}\,
		\gls{s:modulo-elast-efetivo}} \right)^2 \right]^{1/3}.
	\end{equation}
	
	\section{Contato lubrificado}
	Em contraste ao contato seco, define-se como contato lubrificado a situação na qual duas faces opostas do rolamento estão completamente separadas por um filme de fluido \cite{hamrock:1991}.
	Na análise de vibração, a camada de lubrificante desempenha um papel importante, visto que ela chega a ter uma espessura da mesma ordem de grandeza que a própria rugosidade da superfície dos componentes \cite{sassi:2007}.
	Esta seção aborda de maneira superficial os conceitos de contato lubrificado aplicados a rolamentos de esferas empregados no desenvolvimento deste trabalho.
	
	\subsection{Conformidade de superfícies}
	Um dos fatores que exerce influência no aparecimento de diferentes regimes de lubrificação entre as superfícies é a conformidade no contato entre elas.
	São denominadas conformes as superfícies que fazem contato em um encaixe justo e com alta conformidade geométrica \cite{hamrock:1991}, isto é, são geometricamente semelhantes.
	Um exemplo prático de superfícies conformes são os mancais de deslizamento.
	Neste componente, a folga diametral é tipicamente de um milésimo do diâmetro do eixo deslizante.
	Desse modo, a superfície que suporta a carga do contato permanece constante independentemente da força aplicada.
	
	Por outro lado, em diversas outras aplicações de contato lubrificado, as superfícies não têm uma boa conformidade geométrica.
	É o caso de pares de engrenagens, conjuntos came-seguidor, e rolamentos.
	Entre superfícies não-conformes, toda a carga aplicada fica apoiada em uma pequena superfície de contato lubrificada, como mostra a situação da direita na \figurename\ \ref{fig:contato-conforme}.
	Uma característica relevante do contato não-conforme é que o tamanho dessa superfície varia muito em função da carga aplicada, mas sempre permanece em torno de três ordens de magnitude menor do que a área de contato lubrificado entre superfícies conformes \cite{hamrock:1991}.
	
	\begin{figure}[b]
		\incluirimagem{SuperficiesConformes.png}{Contato entre superfícies conformes e não-conformes}{adaptado de \citeauthoronline{hamrock:1991} (\citeyear{hamrock:1991})}
		\label{fig:contato-conforme}
	\end{figure}
	
	\subsection{Lubrificação elastohidrodinâmica}
	A \gls{sig:EHL} é um regime de lubrificação que ocorre no contato entre superfícies não-conformes, onde a deformação elástica dos corpos torna-se significante \cite{spikes:2006}.
	Esta teoria de lubrificação é dividida entre \gls{sig:EHL} dura, para materiais com alto módulo de elasticidade como metais, e \gls{sig:EHL} macia, para os materiais com módulo de elasticidade baixo, como borrachas \cite{hamrock:1991}.
	No primeiro caso, ocorre uma alta pressão na superfície do contato devido à não-conformidade das geometrias, já que a carga acaba concentrada em um ponto ou uma linha.
	Isso provoca dois efeitos principais \cite{spikes:2006}: a deformação dos corpos cria uma região de contato localmente conforme, que toma a forma de uma elipse (como mostrado no item \ref{sec:contato-seco}) ou de um retângulo;
	além disso, a viscosidade do fluido lubrificante aumenta muito na região de maior pressão.
	Esse conjunto de efeitos faz com que a \gls{sig:EHL} também leve o nome de lubrificação \emph{piezo-viscoelástica} \cite{spikes:2006}.
	Tendo como objeto de interesse a aplicação em rolamentos rígidos de esferas, apenas a teoria da \gls{sig:EHL} dura será abordada.
	
	A teoria da \gls{sig:EHL} estuda o fluxo de lubrificante que separa os dois corpos do contato direto, que pode ser descrito pela equação de Reynolds \cite{wijnant:1998,hamrock:1991}.
	Considerando que os dois corpos tenham uma componente de velocidade apenas na direção $x$, ela pode ser representada por
	\begin{equation} \label{eqn:ehl:reynolds}
		\frac{\partial}{\partial x}\left(
		\frac{\gls{s:massa-espec}\gls{s:esp-filme}^3}{\gls{s:visc-abs}}\,
		\frac{\partial\gls{s:pressao}}{\partial x} \right) +
		%
		\frac{\partial}{\partial y}\left(
		\frac{\gls{s:massa-espec}\gls{s:esp-filme}^3}{\gls{s:visc-abs}}\,
		\frac{\partial\gls{s:pressao}}{\partial y} \right) =
		%
		6\,\gls{s:soma-veloc}\frac{\partial\gls{s:massa-espec}
		\gls{s:esp-filme}}{\partial x} +
		%
		6\,\gls{s:massa-espec}
		\gls{s:esp-filme}\frac{\partial\gls{s:soma-veloc}}{\partial x} +
		%
		12\frac{\partial\gls{s:massa-espec}\gls{s:esp-filme}}{
		\partial t},
	\end{equation}
	onde os símbolos \gls{s:massa-espec} e \gls{s:visc-abs} representam respectivamente a massa específica e a viscosidade absoluta do fluido, \gls{s:pressao} é a pressão hidrostática acima da pressão atmosférica, \gls{s:esp-filme} indica a espessura do filme de fluido e \gls{s:soma-veloc} é a soma de velocidades entre as duas superfícies, obtida diretamente com a velocidade de escoamento \gls{s:vel-esc-x} do fluido em cada superfície:
	\begin{equation}
		\gls{s:soma-veloc} = \gls{s:vel-esc-x}_a + \gls{s:vel-esc-x}_b.
	\end{equation}
	
	A análise da \gls{sig:EHL} se combina com a determinação da espessura \gls{s:esp-filme} do filme de lubrificante, que para uma suposição de que os dois corpos tomam a forma parabólica, pode ser definida como \cite{wijnant:1998}
	\begin{equation} \label{eqn:ehl:esp-filme-comp}
		\gls{s:esp-filme}(x,y) = - \gls{s:aprox-mutua} +
		%
		\frac{x^2}{2\gls{s:raio-equiv-x}} +
		%
		\frac{y^2}{2\gls{s:raio-equiv-y}} +
		%
		\frac{2}{\pi\gls{s:modulo-elast-efetivo}}
		%
		\iint_{\gls{s:superf-contato}} \frac{\gls{s:pressao}(x',\,y')
		dx'\,dy'}{\sqrt{(x-x')^2 + (y-y')^2}}.
	\end{equation}
	Nesta equação, o termo \gls{s:aprox-mutua} que aparece indica a aproximação mútua de dois pontos \emph{remotos} entre os dois sólidos em contato \cite{wijnant:1998}.
	O termo \emph{remoto} é empregado para denotar que a deformação elástica é desprezível nestes pontos.
	
	Conhecendo-se o valor da aproximação mútua \gls{s:aprox-mutua}, o problema trazido pelas Equações \ref{eqn:ehl:reynolds} e \ref{eqn:ehl:esp-filme-comp} torna-se bem definido e possuirá uma única solução.
	Este valor, no entanto, geralmente é desconhecido.
	Um método de solução é proposto na referência \cite{nonato:2010}, empregando o método de análise \emph{multinível multiintegração}, modelando um rolamento de esferas tanto como um sistema dinâmico linear como não-linear.
	
	\subsubsection{Simplificação com parâmetros adimensionais} \label{sec:ehl:simpl}
	Uma alternativa simplificada de solução para a espessura do filme de fluido é trazida nas referências \cite{hamrock:1991,spikes:2006}, utilizando fatores adimensionais.
	Nesta proposta, a espessura do filme pode ser resolvida por
	\begin{equation} \label{eqn:ehl:esp-filme}
		\frac{\gls{s:esp-filme}}{\gls{s:raio-equiv-x}} = 
		%
		2,69\gls{s:veloc-adim}^{0,67}\gls{s:material-adim}^{0,53}
		\gls{s:carga-adim}^{-0,067}
		%
		(1-0,61e^{-0,73\gls{s:param-elipticidade}}),
	\end{equation}
	de modo que \gls{s:veloc-adim}, \gls{s:material-adim}, \gls{s:carga-adim} são parâmetros adimensionais de velocidade, material e carga respectivamente.
	Considerando um vetor de velocidade apenas na direção $x$, estes parâmetros valem \cite{hamrock:1991}
	\begin{align}
		\gls{s:veloc-adim} &= \frac{\gls{s:soma-veloc}\gls{s:visc-abs-0}}
		{2\gls{s:modulo-elast-efetivo}\gls{s:raio-equiv-x}}
		\label{eqn:ehl:veloc-adim}\\
		\gls{s:material-adim} &= \gls{s:coef-visc-pres}
		\gls{s:modulo-elast-efetivo} \label{eqn:ehl:material-adim}\\
		\gls{s:carga-adim} &= \frac{\gls{s:carga-radial}}
		{\gls{s:modulo-elast-efetivo}(\gls{s:raio-equiv-x})^2}
		\label{eqn:ehl:carga-adim},
	\end{align}
	sendo \gls{s:visc-abs-0} a viscosidade absoluta do lubrificante à pressão atmosférica e temperatura constante, e \gls{s:coef-visc-pres} é o coeficiente de viscosidade-pressão, uma propriedade do fluido que é relacionada a \gls{s:visc-abs-0} e ao índice de viscosidade-pressão \gls{s:indice-visc-pres} pela equação \cite{wijnant:1998}
	\begin{equation} \label{eqn:ehl:indice-visc-pressao}
		\gls{s:indice-visc-pres} = 
		\frac{1,96\cdot10^8\gls{s:coef-visc-pres}}
		{\ln(\gls{s:visc-abs-0})+9,67}.
	\end{equation}
	
	O índice de viscosidade-pressão \gls{s:indice-visc-pres} para diversos fluidos é exibido na \figurename\ \ref{fig:indices-visc-pressao}, em função das respectivas viscosidades absolutas.
	Mesmo em outras situações de análise, o valor desse índice em uma variedade de óleos minerais tende a convergir para o valor de $0,74$ \cite{roelands:1966}.
	
	\begin{figure}[t]
		\incluirimagem[1.3]{IndiceViscosidadePressao.png}{Índices de viscosidade-pressão para diversos fluidos}{adaptado de \citeauthoronline{roelands:1966} (\citeyear{roelands:1966})}
		\label{fig:indices-visc-pressao}
	\end{figure}

	\subsubsection{Propriedades dinâmicas do filme de fluido} \label{sec:props-filme-fluido}
	A determinação completa das propriedades dinâmicas do fluido envolveria a solução completa do problema da \gls{sig:EHL}.
	Desse modo, alternativas simplificadas para a solução do problema podem ser propostas.
	Entre elas, pode-se empregar um valor linear e constante de amortecimento para o filme de fluido, em conjunto com um valor não-linear de rigidez \gls{s:rig-naolin-filme}.
	Esta propriedade de rigidez tem um comportamento similar à deformação \emph{hertziana} apresentada no item \ref{sec:contato-seco}, utilizando um expoente \gls{s:exp-naolin} definido, com a adição de uma força de sustentação \gls{s:forca-sust-filme} constante \cite{nonato:2014}.
	Desse modo, a força de restauração do filme de fluido \gls{s:forca-rest-filme} pode ser calculada por
	\begin{equation} \label{eqn:ehl:forca-rest}
		\gls{s:rig-naolin-filme}\gls{s:deformacao}^{\gls{s:exp-naolin}} +
		\gls{s:forca-sust-filme} = \gls{s:forca-rest-filme}.
	\end{equation}
	
	A referência \cite{nonato:2014} empregou o modelo utilizado em \cite{nonato:2010} para obter as propriedades de rigidez não-linear, força de sustentação e amortecimento linear para um filme de lubrificante ISO VG 32 em um rolamento radial rígido de esferas tipo 6006.
	No experimento, os resultados foram ajustados, e os valores para \gls{s:rig-naolin-filme} e \gls{s:forca-sust-filme} para diversas velocidades de rotação e carga radial constante são apresentados na \tablename\ \ref{tbl:rig-filme}.
	A tabela também indica o expoente \gls{s:exp-naolin} que deve ser utilizado para cada rigidez e o resíduo do ajuste de da curva.
	Nas outras colunas, \gls{s:veloc-rotacao} indica a velocidade de rotação do rolamento em \gls{sig:RPM} e, de maneira semelhante às seções anteriores, \gls{s:carga-radial} representa a carga radial aplicada ao rolamento.	
	
	\begin{table}[ht]
	\caption{Rigidez não-linear do filme de fluido para um rolamento 6006}
	\def\arraystretch{1.2}
	\centering
	\begin{tabular}{l|l|l|l|l|l}
	\toprule
	$\bm{N_r\ [RPM]}$ & 
	$\bm{w_z\ [N]}$ &
	$\bm{k_{EHD}} \cdot \bm{10^8\ [N/m^j]}$ &
	$\bm{j}$ &
	$\bm{F_{EHD}\ [N]}$ &
	$\bm{1}-\bm{R^2}$\\ \midrule
	%
	$600$&$100$ & $153,8$ & $1,452$ & $2,073$ & $3,39\cdot 10^{-7}$\\
	$1200$&$100$ & $100,0$ & $1,418$ & $3,155$ & $1,39\cdot 10^{-6}$\\
	$1800$&$100$ & $68,3$ & $1,388$ & $4,023$ & $3,28\cdot 10^{-6}$\\
	$2400$&$100$ & $48,2$ & $1,360$ & $4,781$ & $6,10\cdot 10^{-6}$\\
	\bottomrule
	\end{tabular}\par
	\hspace{5pt}
	\fonte{\citeauthoronline{nonato:2014} (\citeyear{nonato:2014})}
	\label{tbl:rig-filme}
	\end{table}
	
	Da mesma forma, na \tablename\ \ref{tbl:amort-filme} são apresentados os resultados da referência \cite{nonato:2014} para o amortecimento elastohidrodinâmico \gls{s:amort-filme} do filme de fluido para o rolamento 6006, junto com o resíduo do ajuste de curva realizado.
	Naquela pesquisa, identificou-se que o amortecimento, diferentemente da rigidez, não depende da carga \cite{nonato:2014}.
	
	\begin{table}[ht]
	\caption{Amortecimento linear do filme de fluido para um rolamento 6006}
	\def\arraystretch{1.2}
	\centering
	\begin{tabular}{l|c|c}
	\toprule
	$\bm{N_r\ [RPM]}$ &
	$\bm{c_{EHD}\ [N.s/m]}$ &
	$\bm{1}-\bm{R^2}$ \\ \midrule
	$ 600 $ & $ 31,602 $ & $ 0,307 $ \\
	$ 1200 $ & $ 22,289 $ & $ 0,303 $\\
	$ 1800 $ & $ 18,027 $ & $ 0,301 $ \\
	$ 2400 $ & $ 15,438 $ & $ 0,300 $ \\ \bottomrule
	\end{tabular}\par
	\hspace{5pt}
	\fonte{\citeauthoronline{nonato:2014} (\citeyear{nonato:2014})}
	\label{tbl:amort-filme}
	\end{table}
	
	\section{Rolamentos rígidos de esferas}
	As seções a seguir apresentam os elementos principais de rolamentos rígidos de esferas e a aplicação dos conceitos demonstrados nos itens anteriores para a modelagem e determinação de propriedades deste componente,	trazendo também uma síntese de outros trabalhos desenvolvidos neste campo de estudo.
	
	\subsection{Falhas em rolamentos}
	Ao analisar a vibração em rolamentos, é de especial interesse identificar o sinal emitido quando apresentam falhas, para permitir diagnósticos no equipamento com antecedência.
	Os defeitos em rolamentos de esferas ou de rolos podem ser classificados como distribuídos ou localizados.
	Entre os defeitos distribuídos, pode-se citar a rugosidade superficial e ondulação nas pistas do rolamento, elementos rolantes fora das dimensões apropriadas, entre outras imperfeições inerentes da fabricação do componente.
	Já os defeitos localizados geralmente são resultados do processo de fadiga, e se apresentam como trincas, lascados ou pequenos buracos nas pistas do rolamento \cite{tandon:1997,sassi:2007}.
	Tendo como interesse no presente trabalho a investigação de falhas relativas ao uso e desgaste do equipamento, apenas os defeitos localizados serão considerados.
	
	\subsection{Trabalhos anteriores}
	Diferentes métodos de análise para a vibração provocada por defeitos localizados em rolamentos foram desenvolvidos desde a metade do século passado.
	\citeauthor{mcfadden:1984} propuseram um modelo bastante simples para modelar um defeito pontual em uma pista interna de um rolamento, que considera a carga radial aplicada e a condução do sinal até o acelerômetro que mediria a vibração, através do mancal.
	Abordagens mais complexas foram propostas nos anos seguintes, como uma análise proposta por \citeauthor{tandon:1997} que modela os anéis do rolamento como sistemas contínuos, além de incluir cargas radiais e axiais no desenvolvimento.
	
	\citeauthor{sassi:2007} apresentaram um modelo que também trata os anéis do rolamento como corpos flexíveis e define a região de análise como um sistema de três graus de liberdade.
	A abordagem considera também as cargas nas direções radial e axial, além de incluir os efeitos de lubrificação entre as pistas e esferas e ruído aleatório no sinal, causado pelo escorregamento dos elementos rolantes.	\citeauthor{patil:2010} modelaram o rolamento como um sistema de dois graus de liberdade que, em contraste aos outros autores, considera os anéis como componentes rígidos e analisa as esferas como deformáveis.
		
	\subsection{Características construtivas}
	As principais medidas do rolamento são ilustradas na Figura \ref{fig:rolamento-medidas}.
	Nas seções posteriores, os subscritos ''i'' e ''o'' nas variáveis relacionam o valor aos anéis interno e externo do rolamento, respectivamente.
	
	O diâmetro específico \gls{s:rol-diam-espec}, denominado em inglês \foreignlanguage{english}{pitch diameter}, é calculado por
	\begin{equation} \label{eqn:rol:diam-espec}
		\gls{s:rol-diam-espec} = \frac{\gls{s:rol-diam-int} + 
		\gls{s:rol-diam-ext}}{2}
	\end{equation}
	sendo \gls{s:rol-diam-int} o diâmetro interno do anel interno, e \gls{s:rol-diam-ext} o diâmetro externo do anel externo.
	
	Nos elementos rolantes, é importante conhecer o número de esferas \gls{s:rol-num-esferas} e o diâmetro \gls{s:rol-diam-esferas} de cada uma.
	No item \ref{sec:curvatura-contato}, a Figura \ref{fig:raios-rolamento} já indica a necessidade do conhecimento dessa medida das esferas.	
	Também na mesma figura, é visível que o raio da pista -- denominada em inglês pelo termo \foreignlanguage{english}{groove} -- do anel, \gls{s:rol-raio-groove}, também é relevante nos cálculos.
	
	No ponto de contato entre a esfera e o anel, uma dimensão importante para a determinação posterior da distribuição de carga do rolamento é a folga diametral \gls{s:rol-folga-diam}.
	Essa medida é fornecida pelo fabricante do componente como folga radial (isto é, metade da folga diametral), e dentro de uma faixa de valores.
	A norma ISO prevê cinco classes de folga radial para rolamentos: 2, N, 3, 4 e 5. A classe 2 é a que apresenta menores folgas, enquanto as maiores estão presentes na classe 5.
	A classe N abrevia a palavra ''normal'' e é empregada na maioria dos rolamentos comerciais \cite{skfClearance}.
		
	\begin{figure}[ht]
		\incluirimagem{MedidasRolamento.png}{Aspectos construtivos principais de um rolamento de esferas}{o autor (\thedate)}
		\label{fig:rolamento-medidas}
	\end{figure}

	\subsection{Distribuição de carga}
	Ao carregar um rolamento com uma força \gls{s:carga-radial}, a distribuição da carga entre os elementos rolantes se dá em uma extensão angular $ 2\gls{s:rol-extensao-carregamento} $ \cite{sassi:2007}, como ilustra a área em verde na Figura \ref{fig:carregamento-rolamento}.
	A carga em um elemento rolante qualquer \gls{s:rol-carga-elemento} é definida nas referências \cite{mcfadden:1984,sassi:2007,tandon:1997,cong:2013} pela equação de Stribeck
	\begin{equation}
		\gls{s:rol-carga-elemento} =
		\begin{cases}
		\gls{s:carga-radial}\,[1-(1/2\vartheta)(1-\cos\gls{s:rol-angulo-carregamento})]^{\gls{s:exp-naolin}}, & -\gls{s:rol-extensao-carregamento}\leq\gls{s:rol-angulo-carregamento}\leq\gls{s:rol-extensao-carregamento} \\
		0, & \mbox{outras posições}
		\end{cases}
	\end{equation}
	sendo $ \vartheta $ um fator de distribuição de carga, que depende do ângulo de contato e da relação entre carga radial e axial aplicada ao rolamento, e \gls{s:rol-angulo-carregamento} o ângulo do elemento rolante analisado em relação ao elemento mais solicitado.
	O expoente \gls{s:exp-naolin} vale $ 1,5 $, de maneira similar à Equação \ref{eqn:seco:def-hertz-curta}.
	
	\begin{figure}[t]
		\incluirimagem{CarregamentoRolamento.png}{Distribuição do carregamento em um rolamento de esferas}{adaptado de \citeauthoronline{sassi:2007} (\citeyear{sassi:2007})}
		\label{fig:carregamento-rolamento}
	\end{figure}

	Essa abordagem, no entanto, não considera a folga diametral do rolamento nem trata da deformação não-linear dos elementos rolantes.
	Havendo apenas carga radial sobre o rolamento, é possível empregar a solução apresentada por \citeauthor{hamrock:1991}.
	Nesta, a extensão angular \gls{s:rol-extensao-carregamento} da distribuição de carga é obtida por
	\begin{equation} \label{eqn:rol:extensao-carregamento}
		\gls{s:rol-extensao-carregamento} = \arccos \frac{
		\gls{s:rol-folga-diam}}{2\gls{s:deformacao-maxima}},
	\end{equation}
	utilizando o valor de deformação máxima calculado pela Equação \ref{eqn:seco:deformacao-max}.
	
	Já a carga em um elemento rolante qualquer emprega também a rigidez não-linear \gls{s:rigidez-hertz} da Equação \ref{eqn:seco:rig-hertz} e é calculada por
	\begin{equation} \label{eqn:rol:carga-elemento}
		\gls{s:rol-carga-elemento} = \gls{s:rigidez-hertz} \left(
		\gls{s:deformacao-maxima}\cos\gls{s:rol-angulo-carregamento}
		 - \frac{\gls{s:rol-folga-diam}}{2} \right)^{\gls{s:exp-naolin}}.
	\end{equation}
	
	Ainda conforme \cite{hamrock:1991}, a carga total aplicada \gls{s:carga-radial} deve ser igual ao somatório da carga distribuída entre todos os elementos rolantes solicitados:
	\begin{equation}\label{eqn:rol:dist-carga-geral}
		\gls{s:carga-radial} = \gls{s:rigidez-hertz}\sum\left(
		\gls{s:deformacao-maxima}\cos\gls{s:rol-angulo-carregamento}
		 - \frac{\gls{s:rol-folga-diam}}{2} \right)^{\gls{s:exp-naolin}}\,
		\cos\gls{s:rol-angulo-carregamento}
	\end{equation}
	o que é representado para um rolamento de esferas na forma integral como
	\begin{equation}\label{eqn:rol:dist-carga-esf}
		\gls{s:carga-radial} = \frac{\gls{s:rol-num-esferas}}{\pi}\,
		\gls{s:rigidez-hertz}\,\gls{s:deformacao-maxima}^{3/2}
		\int_{0}^{\gls{s:rol-extensao-carregamento}}
		\left( \cos\gls{s:rol-angulo-carregamento} - \frac{
		\gls{s:rol-folga-diam}}{2\gls{s:deformacao-maxima}} \right)^{
		3/2} \cos\gls{s:rol-angulo-carregamento}\ 
		d\gls{s:rol-angulo-carregamento}.
	\end{equation}
	
	É possível ainda obter uma expressão que relacione a carga radial aplicada \gls{s:carga-radial} com a carga na esfera mais solicitada \gls{s:rol-carga-maxima-elemento}.
	A carga na esfera mais solicitada pode ser obtida fazendo $ \gls{s:rol-angulo-carregamento} = 0 $ na Equação \ref{eqn:rol:dist-carga-geral} e removendo o somatório \cite{hamrock:1991}:
	\begin{equation}
		\gls{s:rol-carga-maxima-elemento} = \gls{s:rigidez-hertz}
		\left( 1 - \frac{\gls{s:rol-folga-diam}}{
		2\gls{s:deformacao-maxima}} \right)^{\gls{s:exp-naolin}}.
	\end{equation}
	
	Em um segundo passo, essa expressão pode ser dividida pela Equação \ref{eqn:rol:dist-carga-esf}.
	Rearranjando os termos, chega-se a
	\begin{equation} \label{eqn:rol:carga-radial-com-maxima}
		\gls{s:carga-radial} = \frac{\gls{s:rol-num-esferas}\,
		\gls{s:rol-carga-maxima-elemento}}{Z_w},
	\end{equation}
	onde $ Z_w $ representa
	\begin{equation} \label{eqn:rol:zw-carga-radial}
		Z_w = \frac{\displaystyle
		\pi\left( 1 - \frac{\gls{s:rol-folga-diam}}{
		2\gls{s:deformacao-maxima}} \right)^{3/2}}
		{\displaystyle\int_{0}^{\gls{s:rol-extensao-carregamento}}
		\left( \cos\gls{s:rol-angulo-carregamento} - \frac
		{\gls{s:rol-folga-diam}}{2\gls{s:deformacao-maxima}} \right)^
		{3/2} \cos\gls{s:rol-angulo-carregamento}\ 
		d\gls{s:rol-angulo-carregamento}}.
	\end{equation}
	
	\subsection{Determinação de propriedades}
	Como mostrado no item \ref{sec:disc}, a modelagem dinâmica de um sistema depende da determinação de propriedades como rigidez e frequência natural.
	A seguir serão apresentados os métodos para determinação destas propriedades para os componentes de um rolamento.
	
	\subsubsection{Anéis interno e externo} \label{sec:props-aneis}
	Na definição de um modelo para um rolamento de esferas, a deformação dos anéis interno e externo é compreendida modelando estas peças como um sistema contínuo \cite{sassi:2007}.
	Em tal sistema, a frequência natural do $n$-ésimo modo de vibração pode ser determinada através da equação \cite{tandon:1997,sassi:2007,cong:2013}
	\begin{equation} \label{eqn:rol:freq-natural-modo}
		\gls{s:freq-natural} = \frac{n\,(n^2-1)}{\sqrt{1+n^2}}\:
		\sqrt{\frac{\gls{s:modulo-elast}\gls{s:segundo-momento-area}}{
		\gls{s:massa-espec-lin}\gls{s:raio-eixo-neutro}^4}}
	\end{equation}
	onde \gls{s:modulo-elast} é o módulo de elasticidade do material, e \gls{s:segundo-momento-area}, \gls{s:massa-espec-lin} e \gls{s:raio-eixo-neutro} são, respectivamente, o segundo momento de área, a massa por unidade de comprimento, e o raio da linha neutra relativos à seção transversal do anel.
	
	Para os anéis, os modos de vibração $n=0$ e $n=1$ são modos rígidos.
	O primeiro e principal modo em que de fato ocorre deformação no sólido é o modo para $n=2$.
	Embora a vibração ocorra em diversos modos simultaneamente, a amplitude da oscilação nos modos posteriores a $n=2$ não é expressiva.
	Portanto, a Equação \ref{eqn:rol:freq-natural-modo} é utilizada no modelo dos anéis somente neste modo, por questões de simplificação \cite{sassi:2007}.
	
	Uma vez conhecida a frequência natural e a massa \gls{s:rol-massa-anel} de cada anel, é possível encontrar o valor da rigidez \gls{s:rol-rig-anel} para os anéis interno e externo através das equações \cite{sassi:2007}
	\begin{align}
		\gls{s:rol-rig-anel}_i &= \gls{s:rol-massa-anel}_i\,
			(\gls{s:freq-natural}_i)^2 \label{eqn:rol:rig-anel-int}\\
		\gls{s:rol-rig-anel}_o &= \gls{s:rol-massa-anel}_o\,
			(\gls{s:freq-natural}_o)^2 \label{eqn:rol:rig-anel-ext}.
	\end{align}
	
	\subsubsection{Esferas}
	As esferas são os elementos mais rígidos do conjunto mecânico do rolamento \cite{sassi:2007}, o que leva a sua deformação a ser ignorada em alguns modelos, considerando as esferas como corpos rígidos \cite{mcfadden:1984,tandon:1997,sassi:2007,cong:2013}.
	
	É possível, no entanto, modelar a esfera como um corpo deformável.
	Neste caso, a melhor abordagem é tratar a esfera segundo a teoria \emph{hertziana} de contato entre superfícies \cite{patil:2010}, cuja relação entre força e deformação é dada pela Equação \ref{eqn:seco:def-hertz-curta}
	\begin{equation*}
		\gls{s:carga-radial} = \gls{s:rigidez-hertz}\,
		\gls{s:deformacao-maxima}^{\gls{s:exp-naolin}}
	\end{equation*}
	utilizando para o expoente \gls{s:exp-naolin} o valor de $1,5$.
	Para este contato, no entanto, a rigidez \emph{hertziana} \gls{s:rigidez-hertz} deve considerar o contato entre a esfera e os dois anéis simultaneamente.
	Isso é atingido calculando \gls{s:rigidez-hertz} como \cite{patil:2010,hamrock:1991}
	\begin{equation}
		\frac{1}{\gls{s:rigidez-hertz}} = \left\lbrace \left[
		\frac{1}{(\gls{s:rigidez-hertz})_i} \right]^{1/\gls{s:exp-naolin}}
		+ \left[
		\frac{1}{(\gls{s:rigidez-hertz})_o} \right]^{1/\gls{s:exp-naolin}}
		\right\rbrace^{\gls{s:exp-naolin}}
	\end{equation}
	de modo que $ (\gls{s:rigidez-hertz})_i $ e $ (\gls{s:rigidez-hertz})_o $ são aplicações individuais da Equação \ref{eqn:seco:rig-hertz} para o contato com os anéis interno e externo, respectivamente.
	
	\subsection{Frequências características}
	Para a modelagem e análise da vibração de um rolamento com defeito pontual, um conjunto de frequências características é definido.
	Considerando, de maneira mais genérica, que ambos os anéis estejam girando e sejam $ \gls{s:freq-rad}_i $ e $ \gls{s:freq-rad}_o $ as velocidades angulares dos anéis interno e externo do rolamento respectivamente, definem-se as frequências \cite{sassi:2007}
	\begin{align}
		\gls{s:freq-rol-gaiola} &= \frac{1}{2}\cdot\left[ \omega_i\left(1-\frac{\gls{s:rol-diam-esferas}\,\cos\alpha}{\gls{s:rol-diam-espec}} \right) + \omega_o\left(1+\frac{\gls{s:rol-diam-esferas}\cos\alpha}{\gls{s:rol-diam-espec}} \right) \right] \label{eqn:rol:ftf}\\
		\gls{s:freq-rol-def-out} &= \frac{N_b}{2}\cdot(\omega_i-\omega_o)\cdot\left(1-\frac{\gls{s:rol-diam-esferas}\cos\alpha}{\gls{s:rol-diam-espec}} \right) \label{eqn:rol:bpfo}\\
		\gls{s:freq-rol-def-in} &= \frac{N_b}{2}\cdot(\omega_i-\omega_o)\cdot\left(1+\frac{\gls{s:rol-diam-esferas}\cos\alpha}{\gls{s:rol-diam-espec}} \right) \label{eqn:rol:bpfi}\\
		\gls{s:freq-rol-esfera} &= \frac{\gls{s:rol-diam-espec}}{2\,\gls{s:rol-diam-esferas}}\cdot(\omega_i-\omega_o)\cdot\left(1-\frac{\gls{s:rol-diam-esferas}^2\cos^2\alpha}{\gls{s:rol-diam-espec}^2} \right) \label{eqn:rol:bsf}
	\end{align}
	que correspondem, na ordem de apresentação, à frequência fundamental da gaiola, frequência de passagem de esferas por um defeito na pista externa, frequência de passagem de esferas por um defeito na pista interna e, por fim, a frequência de rotação das esferas.
	Ainda nas Equações \ref{eqn:rol:ftf} a \ref{eqn:rol:bsf}, \gls{s:rol-num-esferas} representa o número de esferas ou rolos do rolamento, \gls{s:rol-angulo-contato} é o ângulo de contato das esferas, \gls{s:rol-diam-esferas} indica o diâmetro das esferas, e \gls{s:rol-diam-espec} é o diâmetro específico definido na Equação \ref{eqn:rol:diam-espec}.
	
	\subsection{Força de impacto}
	Nos modelos numéricos investigados, um fator importante na determinação da reposta de vibração é a força de impacto, que pode ser calculada de maneiras diferentes.
	Em geral, essa força é influenciada pelo carregamento estático e pela amplificação durante a passagem dos elementos rolantes pelo defeito, que pode ser tratada como um impulso \cite{mcfadden:1984,sassi:2007,cong:2013} ou como um pulso de duração finita \cite{tandon:1997}.
	
	\subsubsection{Composição da força}
	As referências \cite{mcfadden:1984,tandon:1997} definem a amplitude da força de impacto como sendo dependente somente do carregamento estático \gls{s:rol-carga-elemento} no elemento que passa pelo defeito.
	No modelo proposto na referência \cite{sassi:2007}, por outro lado, a força total de impacto é a soma de \gls{s:rol-carga-elemento} com uma componente dinâmica \gls{s:rol-forca-din-impacto}, descrita por
	\begin{equation}
		\gls{s:rol-forca-din-impacto} =
		\gls{s:rol-carga-elemento}\cdot\gls{s:rol-coef-impacto}\cdot
		\left(\frac{\gls{s:rol-comp-defeito}}
		{\gls{s:rol-diam-esferas}}\right)^2
	\end{equation}
	onde \gls{s:rol-comp-defeito} é o comprimento do defeito e \gls{s:rol-coef-impacto} é um coeficiente de impacto que depende tanto do material como da geometria do rolamento.
	
	Incluindo a componente dinâmica, a força total de impacto \gls{s:rol-forca-tot-impacto} resulta em
	\begin{equation} \label{eqn:rol:forca-impacto-tot}
		\gls{s:rol-forca-tot-impacto} = \gls{s:rol-carga-elemento}
		\left[1 + \gls{s:rol-coef-impacto}\cdot
		\left(\frac{\gls{s:rol-comp-defeito}}
		{\gls{s:rol-diam-esferas}}\right)^2 \right]
	\end{equation}
	de modo que o carregamento \gls{s:rol-carga-elemento} é definido na Equação \ref{eqn:rol:carga-elemento}.
	
	\subsubsection{Impactos modulados pela força}
	A incidência de um elemento rolante sobre o defeito em uma das frequências calculadas nas Equações \ref{eqn:rol:ftf} a \ref{eqn:rol:bsf} resultará em um impacto, cuja amplitude será dependente da força total de impacto naquele ângulo de rotação \cite{mcfadden:1984}.
	Esta aplicação de força em um período curto de tempo é tratada em alguns dos modelos como um impulso unitário \cite{mcfadden:1984,sassi:2007,cong:2013}, que é também chamado de função delta de Kronecker e obedece à seguinte definição \cite{rao:2008}
	\begin{equation} \label{eqn:delta-kron}
		\gls{s:func-impulso}(t) = 
		\begin{cases}
			1,& t=0\\
			0,& t\ne 0
		\end{cases}
	\end{equation}
	
	Considerando o giro do eixo com o rolamento em regime permanente, a função $ p(t) $ que define a influência dos impactos pode ser modelada com uma série infinita de impulsos de igual amplitude $ p_0 $, separados por um período $ \gls{s:periodo}_d $, que é o recíproco da frequência de incidência de um elemento rolante sobre o defeito.
	Um exemplo da função $ p(t) $ está representado na Figura \ref{fig:funcao-impulso} e ela pode ser definida como \cite{mcfadden:1984}
	\begin{equation} \label{eqn:rol:impulsos}
		p(t) = p_0\sum_{k=-\infty}^{\infty}\gls{s:func-impulso}(t-k\,\gls{s:periodo}_d).
	\end{equation}
	
	\begin{figure}[t]
		\incluirimagem{FuncaoImpulso.png}{Função com impulsos periódicos}{adaptado de \citeauthoronline{cong:2013} (\citeyear{cong:2013})}
		\label{fig:funcao-impulso}
	\end{figure}

	A influência do comportamento em impulsos, modulado pela força de impacto total, é obtida multiplicando as Equações \ref{eqn:rol:forca-impacto-tot} e \ref{eqn:rol:impulsos}.
	
	\subsection{Transmissão da oscilação}
	Além do comportamento pulsante que molda a resposta de vibração de um rolamento a partir da força de impacto, outros dois fatores ainda colaboram para modificar o sinal que é efetivamente lido durante a medição da oscilação no componente:
	a condução da vibração através do mancal, dependendo da posição de instalação do transdutor; e o decaimento do impulso lido \cite{mcfadden:1984}.
	
	\subsubsection{Posição relativa do medidor de vibração}
	Quando um impulso é provocado no rolamento, frequências de ressonância serão excitadas nele e na máquina em que ele está instalado, permitindo a medição do sinal em um acelerômetro fixado próximo ao componente.
	Considerando a detecção do transdutor em uma direção, espera-se que a amplitude máxima da vibração medida ocorra quando a oscilação for paralela ao eixo de medição do acelerômetro.
	Da mesma forma, estima-se que a resposta medida pelo dispositivo será próxima de zero quando o impulso ocorrer em uma direção perpendicular ao eixo o acelerômetro \cite{mcfadden:1984}.
	
	Dessa maneira, pode-se determinar a amplitude de vibração \gls{s:rol-amplitude-sensor} medida pelo transdutor em função da amplitude máxima $ \gls{s:rol-amplitude-sensor}_{max} $ e do ângulo $ \theta $ entre esse valor limite e o eixo de medição do acelerômetro, sendo definida por \cite{sassi:2007}
	\begin{equation}
		\gls{s:rol-amplitude-sensor}(\theta) = \gls{s:raio} - \left( \frac{\sin^2\theta}{[\gls{s:raio}-\gls{s:rol-amplitude-sensor}_{max}]^2} + \frac{\cos^2\theta}{[\gls{s:raio}+\gls{s:rol-amplitude-sensor}_{max}]^2} \right)^{-1/2}.
	\end{equation}
	
	Nessa equação, \gls{s:raio} denota o raio do anel não-deformado.
	Uma simplificação pode ser feita no caso do anel avaliado ser fixo, o que torna o ângulo $ \theta $ constante.
	
	\subsubsection{Decaimento do sinal de impulso}
	Ao gerar os impulsos, a resposta de vibração do rolamento excitará frequências de ressonância no sistema \cite{cong:2013}.
	A ressonância provocada por um impulso apresenta-se como uma onda cossenoidal, cuja amplitude sofre um decaimento exponencial como mostra a Figura \ref{fig:decaimento-impulso}, que é expresso por \cite{mcfadden:1984,cong:2013}
	\begin{equation}
		\gls{s:func-decaimento}(t) =
		\begin{cases}
			\cos(\gls{s:freq-natural}\gls{s:tempo})\,e^{-B\gls{s:tempo}}, & \gls{s:tempo}>0\\
			0, & \gls{s:tempo}\leq0
		\end{cases}
	\end{equation}
	onde $ B $ é um parâmetro que define a taxa de decaimento da amplitude, e a frequência natural \gls{s:freq-natural} corresponde à frequência natural do modo obtida na Equação \ref{eqn:rol:freq-natural-modo}.
	
	\begin{figure}[ht]
		\incluirimagem{DecaimentoImpulso.png}{Decaimento exponencial de um sinal de impulso}{adaptado de \citeauthoronline{cong:2013} (\citeyear{cong:2013})}
		\label{fig:decaimento-impulso}
	\end{figure}

	Denotando como $ g(t) $ a reposta de vibração do sistema com o produto das funções definidas até o momento, isto é:
	\begin{enumerate}
		\item a força total de impacto, \gls{s:rol-forca-tot-impacto};
		\item a função de impulsos provocada pela passagem nos defeitos, \gls{s:func-impulso}; e
		\item a amplitude máxima de acordo com a posição do transdutor, \gls{s:rol-amplitude-sensor};
	\end{enumerate}
	pode-se obter o sinal efetivo recebido pelo transdutor ao realizar a convolução entre este produto e a função \gls{s:func-decaimento} \cite{mcfadden:1984,cong:2013}
	\begin{equation}
		g(t) = [\gls{s:rol-forca-tot-impacto}(t)\cdot \gls{s:func-impulso}(t)\cdot \gls{s:rol-amplitude-sensor}(t)]\ast\gls{s:func-decaimento}(t).
	\end{equation}
	
	\chapter{Materiais e métodos}
	O capítulo a seguir apresenta a metodologia adotada para o desenvolvimento do modelo numérico da vibração de um rolamento defeituoso, utilizando como ferramenta de modelagem o \foreignlanguage{english}{software} MATLAB\textsuperscript\textregistered.
	Todo o código-fonte do modelo está disponível no Apêndice \ref{apc:codigo}.
	
	A apresentação dos procedimentos começa pela delimitação de escopo do problema, seguindo com os dados de entrada do modelo.
	Posteriormente, a forma de obtenção das propriedades do rolamento será apresentada para que, por fim, a solução adotada para o modelo dinâmico seja demonstrada.
	
	\section{Proposta e condições} \label{sec:proposta-condicoes}
	O presente trabalho tem como proposta modelar a vibração provocada por um único defeito na pista externa de um rolamento rígido de esferas, tipo 6004-2RSH.
	Considera-se o rolamento como inserido em um mancal, de modo que a seu anel externo seja fixo, enquanto o anel interno é o que possui um eixo acoplado e está livre para girar.
	
	A carga aplicada é puramente radial -- o modelo de rolamento adotado não tolera cargas axiais -- e é mantida constante.
	Uma vez que o defeito encontra-se na pista externa, que é fixa, define-se que a carga aplicada também está alinhada ao defeito.
	
	A modelagem considerará a velocidade de rotação constante, e indo ao encontro dos modelos propostos nas referências \cite{mcfadden:1984,sassi:2007,patil:2010,cong:2013,tandon:1997}, é suposta condição isotérmica em todo o rolamento e durante todo o experimento.
	
	Para considerar os efeitos do filme de lubrificante, serão tomadas as propriedades do óleo hidráulico ISO VG 32, obtidas na referência \cite{iso-vg32} e semelhantemente utilizadas em \cite{nonato:2014}.
	Para as propriedades elastohidrodinâmicas do fluido, serão empregados os dados das referências \cite{roelands:1966,nonato:2014}, apresentados no item \ref{sec:props-filme-fluido}.
	Embora os dados se apliquem ao modelo de um rolamento 6006, a diferença dimensional entre os dois elementos será desconsiderada, aplicando-se os mesmos valores.
	
	\section{Dados de entrada} \label{sec:dados-entrada}
	O modelo desenvolvido assume como velocidade de rotação do eixo um valor de 1800 \gls{sig:RPM} e uma carga radial de 100 N.
	Conforme determinado no item \ref{sec:proposta-condicoes}, a carga encontra-se alinhada ao defeito.
	Os dados gerais do rolamento 6004-2RSH utilizados no modelo são mostrados na \tablename\ \ref{tbl:6004-gerais}.
	Na \tablename\ \ref{tbl:6004-aneis} são detalhadas propridades específicas dos anéis do rolamento.
	Parte destes valores foi obtida diretamente em \cite{skf6004}, enquanto outros -- como propriedades de partes independentes -- foram extraídos de um modelo 3D do rolamento, obtido da mesma fonte e ilustrado na Figura \ref{fig:3d-rolamento}.
	
	\begin{table}[t]
	\caption{Dados do rolamento 6004-2RSH}
	\def\arraystretch{1.2}
	\centering
	\begin{tabular}{l|c|l}
	\toprule
	\textbf{Propriedade} & \textbf{Símbolo} & \textbf{Valor} \\\midrule
	Diâmetro interno & \gls{s:rol-diam-int} & $20\ \text{mm}$ \\
	Diâmetro externo & \gls{s:rol-diam-ext} & $42\ \text{mm}$ \\
	Número de esferas & \gls{s:rol-num-esferas} & $ 9 $\\
	Diâmetro da esfera & \gls{s:rol-diam-esferas} & $6,35\ \text{mm}$ \\
	Folga diametral & \gls{s:rol-folga-diam} & $ 16\ \mu\text{m} $ \\
	Ângulo de contato & \gls{s:rol-angulo-contato} & $ 0\text\textdegree$\\
	Massa da esfera & \gls{s:rol-massa-esfera} & $ 0,00105\ \text{kg} $
	\\\bottomrule
	\end{tabular}
	\label{tbl:6004-gerais}
	\end{table}

	\begin{table}[b]
	\def\arraystretch{1.2}
	\caption{Dados dos anéis do rolamento 6004-2RSH}
	\begin{tabular}{l|c|l|l}
	\toprule
	\textbf{Propriedade} & \textbf{Símbolo} & \textbf{Anel interno} & \textbf{Anel externo} \\\midrule
	Diâmetro, face de montagem & $D$ & $20\ \text{mm}$ & $ 42\ \text{mm}$ \\
	Diâmetro, face da esfera & $D_2$ & $24,65\ \text{mm}$ & $37,19\ \text{mm}$ \\
	Raio de curvatura, eixo x & $r_{bx}$ & $12,32\ \text{mm}$ & $18,68\ \text{mm}$ \\
	Raio de curvatura, eixo y & $r_{by}$ & $3,18\ \text{mm}$ & $3,18\ \text{mm}$ \\
	Raio da linha neutra & \gls{s:raio-eixo-neutro} & $11,65\ \text{mm}$ & $19,43\ \text{mm}$ \\
	Segundo momento de área & \gls{s:segundo-momento-area} & $37,424\cdot 10^{-12}\ \text{m}^4$ & $31,802\cdot 10^{-12}\ \text{m}^4$ \\
	Massa & \gls{s:rol-massa-anel} & $0,022\ \text{kg}$ & $0,035\ \text{kg}$ \\
	Massa por unidade de compr. & \gls{s:massa-espec-lin} & $0,301\ \text{kg/m}$ & $0,289\ \text{kg/m}$ \\\bottomrule
	\end{tabular}
	\label{tbl:6004-aneis}
	\end{table}
	
	Para ambos os anéis e esferas foi considerado o mesmo aço como material, considerando um módulo de elasticidade de $200\ \text{GPa}$, coeficiente de Poisson de $ 0,3 $ e massa específica de $ 7833\ \text{kg/m}^3$ \cite{hamrock:1991}.
	
	As propriedades utilizadas no óleo hidráulico ISO VG 32 empregado no modelo são $32\cdot 10^{-6}\ \text{m\textsuperscript{2}/s}$ para a viscosidade cinemática \gls{s:visc-cin} e massa específica \gls{s:massa-espec} de $861\ \text{kg/m\textsuperscript{3}}$ \cite{iso-vg32}.
	O índice de viscosidade-pressão empregado foi o valor constante encontrado em \cite{roelands:1966} para óleos minerais, de $0,74$.
	Os valores de rigidez e amortecimento também foram obtidos diretamente a partir de \cite{nonato:2014}.
	O modelo utiliza os valores das Tabelas \ref{tbl:rig-filme} e \ref{tbl:amort-filme} para a velocidade de $1800\ \text{RPM}$.

	\begin{figure}[t]
		\incluirimagem[0.7]{Modelo3DRolamento.png}{Modelo 3D do rolamento 6004}{SKF.}
		\label{fig:3d-rolamento}
	\end{figure}

	\section{Determinação de propriedades}
	A partir as propriedades iniciais trazidas no item \ref{sec:dados-entrada}, um conjunto de propriedades derivadas é calculado, para que os cálculos do modelo possam ser efetivamente realizados.
	A obtenção destes valores a partir da teoria apresentada no capítulo anterior é segmentada nas seções abaixo.

	\subsection{Diâmetros e raios de elementos}	
	Como mostrado na Equação \ref{eqn:rol:diam-espec}, o diâmetro específico \gls{s:rol-diam-espec} é calculado a partir dos dados da \tablename\ \ref{tbl:6004-gerais}:
	\begin{equation*}
		\gls{s:rol-diam-espec} = \frac{20 + 42}{2} = 31\ \text{mm}.
	\end{equation*}
	
	Para as esferas, o diâmetro \gls{s:rol-diam-esferas} nominal fornecido pelo fabricante pode ser utilizado de maneira satisfatória para obtenção das frequências características do rolamento, mostradas nas Equações \ref{eqn:rol:ftf} a \ref{eqn:rol:bsf}.
	No entanto, para utilização do raio da esfera (conforme mostrado na Figura \ref{fig:raios-rolamento}) nos cálculos relacionados ao contato de superfícies, a folga diametral -- não considerada nas medidas nominais do fabricante -- precisa ser ponderada.
	Dessa maneira, pode-se calcular
	\begin{equation}
		r_{ax} = r_{ay} = \frac{D_{2o} - D_{2i} - c_d}{4} = 3,1310\ \text{mm}.
	\end{equation}
	
	\subsection{Velocidades e frequências características}
	Aplicando a condição estabelecida no item \ref{sec:proposta-condicoes} de que o anel externo do rolamento é fixo no mancal e que a velocidade de rotação do eixo é impressa no anel interno, definem-se as velocidades angulares
	\setlength\jot{2ex plus 1ex minus 1ex}
	\begin{align}
		\gls{s:freq-rad}_o &= 0 \\
		\gls{s:freq-rad}_i &= \frac{1800\ \text{RPM}\cdot\pi}{30} = 188,5\ \text{rad/s}.
	\end{align}
	
	Com estas velocidades angulares, é possível determinar as frequências características do rolamento, descritas nas Equações \ref{eqn:rol:ftf} a \ref{eqn:rol:bsf}.
	Aplicando-se o modelo a um defeito na pista externa do rolamento, apenas a frequência de passagem de uma esfera pela pista externa -- \gls{s:freq-rol-def-out}, Equação \ref{eqn:rol:bpfo} -- interessa:
	\begin{equation*}
		\gls{s:freq-rol-def-out} = \frac{9}{2}\cdot(188,5 - 0)\cdot
		\left( 1 - \frac{6,35\,\cos 0}{31} \right) =
		674,48\ \text{rad/s} =	107,35\ \text{Hz}.
	\end{equation*}
	
	\subsection{Rigidezes e frequências naturais}
	Seguindo a linha de desenvolvimento dos modelos de \cite{mcfadden:1984,tandon:1997,sassi:2007,cong:2013}, as esferas são consideradas corpos rígidos, então a sua deformação será desprezada no comportamento dinâmico do sistema.
	
	Os anéis, por sua vez, são tratados como elementos flexíveis -- ou contínuos.
	Conforme proposto no item \ref{sec:props-aneis}, apenas o modo de vibração $n=2$ é considerado.
	Assim, a Equação \ref{eqn:rol:freq-natural-modo} pode ser empregada para obter a frequência natural dos anéis interno e externo como
	\begin{align*}
		\gls{s:freq-natural}_i &= \frac{2(2^2 - 1)}{\sqrt{1+2^2}}\,
		\sqrt{\frac{200\cdot 10^9\cdot 37,424\cdot 10^{-12}}{0,301
		\cdot 0,01165^4}} = 9,8587\cdot 10^4\ \text{rad/s} \\
		\gls{s:freq-natural}_o &= \frac{2(2^2 - 1)}{\sqrt{1+2^2}}\,
		\sqrt{\frac{200\cdot 10^9\cdot 31,802\cdot 10^{-12}}{0,289
		\cdot 0,01943^4}} = 3,3344\cdot 10^4\ \text{rad/s},
	\end{align*}
	e as suas rigidezes podem então ser obtidas com o par de Equações \ref{eqn:rol:rig-anel-int} e \ref{eqn:rol:rig-anel-ext}:
	\begin{align*}
		\gls{s:rol-rig-anel}_i &= 0,022\cdot(9,8587\cdot 10^4)^2 =
		2,1383\cdot10^8\ \text{N/m} \\
		\gls{s:rol-rig-anel}_o &= 0,035\cdot(3,3344\cdot 10^4)^2 =
		3,8913\cdot10^7\ \text{N/m}.
	\end{align*}
	
	\section{Desenvolvimento do modelo}
	O modelo desenvolvido para a vibração do rolamento defeituoso trata-se de um modelo com três massas e três graus de liberdade, ilustrado na Figura \ref{fig:modelo-dinamico}.
	A esfera é a massa central do sistema, rígida, que está conectada aos dois anéis por um conjunto mola-amortecedor de cada lado.
	Esse conjunto é a representação do filme do lubrificante existente entre cada par de corpos.
	O filme de fluido é modelado com uma característica linear de amortecimento e não-linear de rigidez, conforme mostrado no item \ref{sec:props-filme-fluido}.
	
	Os anéis são representados como uma associação massa-mola.
	O comportamento de mola no anel está relacionado à sua deformação por ser modelado como um sistema contínuo.
	Quaisquer efeitos de amortecimento interno ou externo deste componente são desconsiderados.
	
	A excitação que provocará a vibração no sistema é originada pelo impacto decorrente da passagem de um elemento rolante pela falha no anel externo.
	A força transmitida em cada impacto está diretamente relacionada à carga estática \gls{s:rol-carga-elemento} da esfera no momento da incidência no defeito, que depende das características do contato entre a esfera e as pistas dos anéis interno e externo.
		
	\begin{figure}[t]
		\incluirimagem{ModeloDinamico.png}{Representação esquemática do modelo dinâmico}{o autor (\thedate)}
		\label{fig:modelo-dinamico}
	\end{figure}
	
	\subsection{Caracterização do contato entre os sólidos}
	Uma vez que a força de excitação do sistema dinâmico depende da carga \gls{s:rol-carga-elemento}, é preciso empregar os conceitos da teoria \emph{hertziana} de deformação por contato mostrada no item \ref{sec:contato-seco} para a resolução do problema.
	
	Conhecidos os raios de curvatura da esfera e dos anéis, mostrados nas \tablename s \ref{tbl:6004-gerais} e \ref{tbl:6004-aneis}, é possível determinar para as duas superfícies de contato -- esfera com anel interno e esfera com anel externo -- valores para os raios de curvatura equivalentes \gls{s:raio-equiv-x} e \gls{s:raio-equiv-y}, além da soma e diferença de curvatura \gls{s:curvatura-soma} e \gls{s:curvatura-diferenca}, mostrados nas Equações \ref{eqn:seco:r-equiv-x} a \ref{eqn:seco:curv-diferenca}.
	Para o contato entre a esfera e a pista interna, tem-se
	\begin{align*}
		\frac{1}{\gls{s:raio-equiv-x}_i} = \frac{1}{3,131} +
		\frac{1}{12,32} &\implies \gls{s:raio-equiv-x}_i =
		2,497\ \text{mm} \\
		\frac{1}{\gls{s:raio-equiv-y}_i} = \frac{1}{3,131} +
		\frac{1}{3,18} &\implies \gls{s:raio-equiv-y}_i =
		1,578\ \text{mm} \\
		\frac{1}{\gls{s:curvatura-soma}_i} = \frac{1}{2,497} +
		\frac{1}{1,578} &\implies \gls{s:curvatura-soma}_i =
		0,9667\ \text{mm}\\
		\gls{s:curvatura-diferenca}_i = 0,9667\left( 
		\frac{1}{2,497} - \frac{1}{1,578} \right) &\implies
		\gls{s:curvatura-diferenca}_i = -0,2255.
	\end{align*}
	
	O procedimento pode ser repetido para o contato entre esfera e pista externa, obtendo os valores finais
	\begin{align*}
		\gls{s:raio-equiv-x}_o &= 2,682\ \text{mm}\\
		\gls{s:raio-equiv-y}_o &= 1,578\ \text{mm}\\
		\gls{s:curvatura-soma}_o &= 0,9333\ \text{mm}\\
		\gls{s:curvatura-diferenca}_o &= -0,2592.
	\end{align*}
	
	Possuindo valores para a diferença de curvatura \gls{s:curvatura-diferenca}, é possível encontrar os valores para as integrais elípticas \gls{s:int-eliptica-1} e \gls{s:int-eliptica-2} e para o parâmetro de elipticidade \gls{s:param-elipticidade} através das Equações \ref{eqn:seco:int-elip-1} a \ref{eqn:seco:param-elip}.
	Como identificado no item \ref{sec:int-elipticas}, a obtenção desses valores se dá através de um processo iterativo.
	
	No código-fonte do Apêndice \ref{apc:codigo}, esse processo é ilustrado pela função \textsc{ParametrosElipseContato}, e consiste em arbitrar um valor inicial para \gls{s:param-elipticidade}, iterando em seguida os cálculos de \gls{s:int-eliptica-1}, \gls{s:int-eliptica-2} e do novo valor para \gls{s:param-elipticidade} até que seja satisfeita a condição
	\begin{equation} \label{eqn:mdl:cond-parada-k}
		|\gls{s:param-elipticidade}_n - \gls{s:param-elipticidade}_{n+1}| \leq 10^{-7}.
	\end{equation}
	
	Para a situação deste modelo, o valor inicial arbitrado para \gls{s:param-elipticidade} foi $2$, e os valores destas três propriedades encontrados para as duas superfícies de contato são
	\begin{align*}
		\gls{s:int-eliptica-1}_i = 1,9193, &\qquad
		\gls{s:int-eliptica-1}_o = 1,9232 \\
		\gls{s:int-eliptica-2}_i = 1,3140, &\qquad
		\gls{s:int-eliptica-2}_o = 1,3120 \\
		\gls{s:param-elipticidade}_i = 1,3235, &\qquad
		\gls{s:param-elipticidade}_o = 1,3191.
	\end{align*}
	
	\subsection{Obtenção da força de impacto} \label{sec:mdl:forca-impacto}
	Embora haja a possibilidade de incluir uma componente dinâmica na força total de impacto, o modelo desenvolvido considera que a excitação do sistema provocada no momento do impacto é igual à carga estática \gls{s:rol-carga-elemento} à qual está sujeito o elemento rolante que está passando pelo defeito.
	
	Como o item \ref{sec:proposta-condicoes} aponta, pressupõe-se que a carga radial está alinhada com o defeito na pista externa.
	Isso faz com que \gls{s:rol-angulo-carregamento} seja zero na Equação \ref{eqn:rol:dist-carga-geral} e dessa maneira o carregamento \gls{s:rol-carga-elemento} da esfera incidindo sobre o defeito seja igual à carga da esfera mais solicitada \gls{s:rol-carga-maxima-elemento}. 
	
	É possível, portanto, utilizar as Equações \ref{eqn:rol:carga-radial-com-maxima} e \ref{eqn:rol:zw-carga-radial} para determinar a força de impacto,
	se a primeira destas for rearranjada para
	\begin{equation} \label{eqn:mdl:carga-maxima}
		\gls{s:rol-carga-maxima-elemento} = 
		\frac{Z_w\gls{s:carga-radial}}{\gls{s:rol-num-esferas}}
	\end{equation}
	e for nela empregada então a carga radial constante \gls{s:carga-radial} aplicada ao rolamento e o número de esferas \gls{s:rol-num-esferas} do componente.
	
	Para encontrar uma solução para o valor de $Z_w$, também foi empregado um método iterativo, conforme mostrado em \cite{hamrock:1991}.
	O processo consiste em determinar um valor inicial para $Z_w$, sugerido na mesma referência como $5$, calculando em seguida o valor de \gls{s:rol-carga-maxima-elemento} como mostrado na Equação \ref{eqn:mdl:carga-maxima}.
	Representando o valor retornado a força à qual está submetido um elemento rolante, utiliza-se esta quantidade no lugar de \gls{s:carga-radial} na Equação \ref{eqn:seco:deformacao-max}, resultando em
	\begin{equation} \label{eqn:mdl:deformacao-max}
	(\gls{s:deformacao-maxima})_{i,o} = \gls{s:int-eliptica-1} \left[
	\left( \frac{9}{2\gls{s:int-eliptica-2} \gls{s:curvatura-soma}} \right) \left(
	\frac{\gls{s:rol-carga-maxima-elemento}}
	{\pi\gls{s:param-elipticidade}\,\gls{s:modulo-elast-efetivo}}
	\right)^2 \right]^{1/3}.
	\end{equation}
	
	A deformação máxima obtida na Equação \ref{eqn:mdl:deformacao-max} deve ser calculada para as duas superfícies de contato -- esfera com o anel interno e esfera com o anel externo.
	As deformações $\gls{s:deformacao-maxima}_i$ e $\gls{s:deformacao-maxima}_o$ são somadas para efetivamente resultar em \gls{s:deformacao-maxima}:
	\begin{equation} \label{eqn:mdl:soma-deformacoes}
		\gls{s:deformacao-maxima} = \gls{s:deformacao-maxima}_i +
		\gls{s:deformacao-maxima}_o.
	\end{equation}
	
	Após a obtenção da deformação máxima pela Equação \ref{eqn:mdl:soma-deformacoes}, o resultado deve ser empregado na determinação de um valor para a extensão angular \gls{s:rol-extensao-carregamento} com a Equação \ref{eqn:rol:extensao-carregamento}
	\begin{equation*}
	\gls{s:rol-extensao-carregamento} = \arccos \frac{
		\gls{s:rol-folga-diam}}{2\gls{s:deformacao-maxima}}
	\end{equation*}
	e utilizá-la no cálculo da integral no denominador da Equação \ref{eqn:rol:zw-carga-radial}.
	
	A resolução desta integral em conjunto com o novo valor calculado para \gls{s:deformacao-maxima} e a folga diametral constante \gls{s:rol-folga-diam} possibilitam o recálculo do valor de $Z_w$ segundo a Equação \ref{eqn:rol:zw-carga-radial}
	\begin{equation*}
	Z_w = \frac{\displaystyle
		\pi\left( 1 - \frac{\gls{s:rol-folga-diam}}{
			2\gls{s:deformacao-maxima}} \right)^{3/2}}
	{\displaystyle\int_{0}^{\gls{s:rol-extensao-carregamento}}
		\left( \cos\gls{s:rol-angulo-carregamento} - \frac
		{\gls{s:rol-folga-diam}}{2\gls{s:deformacao-maxima}} \right)^
		{3/2} \cos\gls{s:rol-angulo-carregamento}\ 
		d\gls{s:rol-angulo-carregamento}}.
	\end{equation*}
	
	O novo valor calculado para $Z_w$ deve então ser realimentado na Equação \ref{eqn:mdl:carga-maxima}, possibilitando repetir os passos do procedimento.
	O processo iterativo continua, obedecendo um critério de parada semelhante ao da Equação \ref{eqn:mdl:cond-parada-k}
	\begin{equation}
		| (Z_w)_n - (Z_w)_{n-1} | \leq 10^{-7}.
	\end{equation}
	
	O valor final de $Z_w$ obtido é utilizado uma última vez na Equação \ref{eqn:mdl:carga-maxima} para obter a força sobre o elemento rolante no momento do impacto.
	O procedimento iterativo para o cálculo de \gls{s:rol-carga-maxima-elemento} é empregado na função \textsc{ObterCargaMaximaEsfera} do código-fonte do Apêndice \ref{apc:codigo}.
	Com os valores de entrada utilizados, chega-se a uma carga máxima  $\gls{s:rol-carga-maxima-elemento} = 79,38\ N.$
	
	A carga máxima encontrada é empregada para desenvolver um perfil de impactos ao longo do tempo.
	Cada impacto tem a força equivalente a \gls{s:rol-carga-maxima-elemento} e os impactos ocorrem na frequência em que as esferas passam pelo defeito no anel externo, isto é, \gls{s:freq-rol-def-out}.
	Os impactos ao longo do tempo são modelados, conforme as referências \cite{mcfadden:1984,tandon:1997}, como uma série infinita de impulsos ou de pulsos, em três formas diferentes:
	\begin{enumerate}
		\item impulso infinitesimal com o delta de Kronecker, conforme mostrado na Equação \ref{eqn:delta-kron}, exibido na \figurename\ \ref{fig:modelo-imp-kron};
		\item pulso triangular simétrico, com duração de $250\ \mu\text{s}$, exibido na \figurename\ \ref{fig:modelo-imp-tri};
		\item pulso retangular com duração de $0,5\%$ do período correspondente à frequência \gls{s:freq-rol-def-out} (aproximadamente $47\ \mu\text{s}$), exibido na \figurename\ \ref{fig:modelo-imp-rect}.
	\end{enumerate}
	
	\begin{figure}[b]
		\incluirimagem[0.5]{ModeloImpulsoUnicoKron.png}{Perfil de impulso unitário de impacto}{o autor (\thedate)}
		\label{fig:modelo-imp-kron}
	\end{figure}
	
	\begin{figure}[ht]
		\incluirimagem[0.5]{ModeloImpulsoUnicoTri.png}{Perfil de pulso triangular de impacto}{o autor (\thedate)}
		\label{fig:modelo-imp-tri}
	\end{figure}

	\begin{figure}[ht]
		\incluirimagem[0.5]{ModeloImpulsoUnicoRect.png}{Perfil de pulso retangular de impacto}{o autor (\thedate)}
		\label{fig:modelo-imp-rect}
	\end{figure}

	\subsection{Construção do modelo dinâmico}
	Por caracterizar-se como um modelo de três graus de liberdade, a sua solução é composta por um sistema de três equações diferenciais ordinárias.
	O sistema é montado na forma matricial apresentada no item \ref{sec:disc}, e a matriz de massa \gls{s:matriz-massa} é dada por
	\begin{equation*}
		\gls{s:matriz-massa} =
	\begin{bmatrix}
		\gls{s:rol-massa-anel}_o	&	0	&	0\\
		0	&	\gls{s:rol-massa-esfera}	&	0\\
		0	&	0	&	\gls{s:rol-massa-anel}_i
	\end{bmatrix}.
	\end{equation*}
	
	A matriz de amortecimento é composta por combinações dos amortecimentos lineares do filme de fluido, mostrados na \tablename\ \ref{tbl:amort-filme}.
	Neste modelo são empregados os mesmos valores de amortecimento para os filmes dos anéis interno e externo.
	Mesmo assim, por questões de clareza, a matriz é representada com os amortecimentos em variáveis independentes como
	\begin{equation*}
		\gls{s:matriz-amort} = 
	\begin{bmatrix}
		\gls{s:amort-filme}_o & -\gls{s:amort-filme}_o & 0\\
		-\gls{s:amort-filme}_o & \gls{s:amort-filme}_o +
		\gls{s:amort-filme}_i & -\gls{s:amort-filme}_i\\
		0 & -\gls{s:amort-filme}_i & \gls{s:amort-filme}_i
	\end{bmatrix}.
	\end{equation*}
	
	Na matriz de rigidez existe a necessidade de considerar a característica não-linear do amortecimento do filme de fluido.
	Dessa maneira, não é possível montar uma única matriz para o sistema.
	Como solução, adota-se uma matriz de rigidezes lineares, $\gls{s:matriz-rigidez}_{lin}$, que contém os termos de rigidez dos anéis interno e externo, resultando em
	\begin{equation*}
		\gls{s:matriz-rigidez}_{lin} =
	\begin{bmatrix}
		\gls{s:rol-rig-anel}_o	&	0	&	0\\
		0	&	0	&	0\\
		0	&	0	&	\gls{s:rol-rig-anel}_i
	\end{bmatrix}.
	\end{equation*}
	
	A montagem da parte não-linear do sistema começa pela definição da força de restauração elastohidrodinâmica do filme de fluido, determinada na Equação \ref{eqn:ehl:forca-rest}.
	Definindo \gls{s:forca-rest-filme} como uma função do deslocamento, de modo que
	\begin{equation*}
		\gls{s:forca-rest-filme}(\gls{s:deformacao},t) =
		\gls{s:rig-naolin-filme}\,[\gls{s:deformacao}(t)]^{
		\gls{s:exp-naolin}} + \gls{s:forca-sust-filme},
	\end{equation*}
	é possível determinar matrizes de valores unitários $\gls{s:matriz-rigidez}_f$ que permitam a avaliação da função de \gls{s:forca-rest-filme} apenas nas equações do sistema em que essa componente de força realmente existe.
	
	Assim como na determinação do amortecimento, os valores para o filme no anel externo e interno são iguais, mas se adotarão conjuntos de variáveis individuais para cada parte.
	Portanto, estabelece-se $\gls{s:matriz-rigidez}_{fo}$ como a matriz que determina as componentes da força de restauração relativas ao anel externo, e $\gls{s:forca-rest-filme}_o$ como a função que utiliza as propriedades do filme de lubrificante do anel externo.
	
	O vetor com as forças de restauração desta camada respectivas a cada equação diferencial do sistema pode ser então obtido aplicando a função $\gls{s:forca-rest-filme}_o(\gls{s:deformacao},t)$ a cada elemento do vetor de deslocamentos \gls{s:vetor-desloc} e calcular o produto com a matriz $\gls{s:matriz-rigidez}_{fo}$, obtendo então
	\begin{equation*}
		\gls{s:matriz-rigidez}_{fo}\times
		\gls{s:forca-rest-filme}_o(\gls{s:vetor-desloc}).
	\end{equation*}
	
	O procedimento pode ser executado de maneira análoga para a rigidez do filme de lubrificante do anel interno.
	No sistema modelado, as matrizes $\gls{s:matriz-rigidez}_{fo}$ e $\gls{s:matriz-rigidez}_{fi}$ valem
	\begin{equation*}
		\gls{s:matriz-rigidez}_{fo} =
		\begin{bmatrix}
		1	&	-1	&	0\\
		-1	&	1	&	0\\
		0	&	0	&	0\\
		\end{bmatrix},\qquad
		\gls{s:matriz-rigidez}_{fi} =
		\begin{bmatrix}
		0	&	0	&	0\\
		0	&	1	&	-1\\
		0	&	-1	&	1\\
		\end{bmatrix}.
	\end{equation*}
	
	O vetor de forças externas \gls{s:vetor-forcas}, por fim, contém apenas um elemento não-nulo, que é a função \gls{s:fun-forca-impacto} que define a força gerada pelos impactos ao longo do tempo, determinada no item \ref{sec:mdl:forca-impacto}.
	Considerando a força de impacto como sendo originada no anel externo do rolamento, o vetor de forças externas vale
	\begin{equation*}
		\gls{s:vetor-forcas} = 
		\begin{Bmatrix}
		\gls{s:fun-forca-impacto} \\ 0 \\ 0
		\end{Bmatrix}.
	\end{equation*}
	
	A equação diferencial de movimento completa do sistema modelado resulta, então, em
	\begin{equation} \label{eqn:mdl:sistema}
		\gls{s:matriz-massa}\,\gls{s:vetor-acel} +
		%
		\gls{s:matriz-amort}\,\gls{s:vetor-vel} +
		%
		\gls{s:matriz-rigidez}_{lin}\,\gls{s:vetor-desloc} +
		%
		\gls{s:matriz-rigidez}_{fo}\,\gls{s:forca-rest-filme}_o
		(\gls{s:vetor-desloc}) +
		%
		\gls{s:matriz-rigidez}_{fi}\,\gls{s:forca-rest-filme}_i
		(\gls{s:vetor-desloc}) = 
		%
		\gls{s:vetor-forcas},
	\end{equation}
	onde por uma questão de simplificação à leitura, o símbolo de produto vetorial foi omitido.
	
	A montagem do sistema é executada como parte do método iterativo para a resolução das equações diferenciais, e é executada na função \textsc{SisNaoLinOrdem2} do Apêndice \ref{apc:codigo}.
		
	\subsection{Determinação de condições de contorno}
	Para encontrar a solução particular do sistema matricial da Equação \ref{eqn:mdl:sistema}, são necessárias seis condições de contorno.
	Elas correspondem à posição e a velocidade iniciais do anel externo, da esfera e do anel interno.
	No modelo desenvolvido, as três velocidades iniciais são consideradas nulas.
	A posição inicial do anel externo também será tomada como zero, enquanto as posições iniciais dos outros componentes dependem da espessura do filme de fluido, cujo método de obtenção é apresentado no item \ref{sec:ehl:simpl} utilizando parâmetros adimensionais.
	
	A determinação do coeficiente de visosidade-pressão \gls{s:coef-visc-pres} depende da viscosidade absoluta do fluido à pressão atmosférica \gls{s:visc-abs-0}.
	Possuindo a viscosidade cinemática \gls{s:visc-cin} e a massa específica \gls{s:massa-espec} do lubrificante, essa propriedade pode ser encontrada diretamente por \cite{roelands:1966}
	\begin{equation}
		\gls{s:visc-abs-0} = \gls{s:visc-cin}\,\gls{s:massa-espec} =
		0,02755\ \text{Pa.s},
	\end{equation}
	permitindo estabelecer \gls{s:coef-visc-pres} utilizando $\gls{s:indice-visc-pres} = 0,74$ e rearranjando a Equação \ref{eqn:ehl:indice-visc-pressao} para chegar a
	\begin{equation}
		\gls{s:coef-visc-pres} = \gls{s:indice-visc-pres}\cdot
		\frac{\log(\gls{s:visc-abs-0})+9,67}{1,96\cdot 10^8} =
		2,295\cdot 10^{-8}.
	\end{equation}
	
	Em seguida, os parâmetros adimensionais das Equações \ref{eqn:ehl:veloc-adim}, \ref{eqn:ehl:material-adim} e \ref{eqn:ehl:carga-adim} para os filmes de fluido dos dois anéis podem ser encontrados:
	\begin{align*}
		\gls{s:veloc-adim}_i &= 0,7336\cdot 10^{-10} \qquad
		\gls{s:veloc-adim}_o = 0,1399\cdot 10^{-10}\\
		%
		\gls{s:material-adim}_i &= \gls{s:material-adim}_o =
		5,0437\cdot 10^3\\
		%
		\gls{s:carga-adim}_i &= 0,5795\cdot 10^{-4} \qquad
		\gls{s:carga-adim}_o = 0,5023\cdot 10^{-4}.
	\end{align*}
	
	Uma vez que os parâmetros adimensionais, o raio equivalente \gls{s:raio-equiv-x} e o parâmetro de elipticidade \gls{s:param-elipticidade} são conhecidos para cada contato, a Equação \ref{eqn:ehl:esp-filme} é utilizada para determinar as espessuras dos filmes de fluido, resultando em
	\begin{equation*}
		\gls{s:esp-filme}_i = 0,1474\cdot 10^{-6}\ m \qquad
		\gls{s:esp-filme}_o = 0,05262\cdot 10^{-6}\ m.
	\end{equation*}
	
	O cálculo das espessuras de filme de fluido pode ser encontrado no código-fonte do Apêndice \ref{apc:codigo} encapsulado na função \textsc{EspessuraFilmeLub}.
	
	Com as espessuras de filme de fluido determinadas, as condições de contorno para o sistema da Equação \ref{eqn:mdl:sistema} resultam em
	\begin{equation*}
		\gls{s:vetor-desloc}_0 =
		\begin{Bmatrix}
		0 \\ \gls{s:esp-filme}_o \\ \gls{s:esp-filme}_i
		\end{Bmatrix} \qquad
		\gls{s:vetor-vel}_0 = 
		\begin{Bmatrix}
		0 \\ 0 \\ 0
		\end{Bmatrix}.
	\end{equation*}
	
	\subsection{Resolução do sistema} \label{sec:mdl:sol-sistema}
	As simulações ocorreram em duas partes: inicialmente o modelo foi gerado com uma duração de $0,5$ segundo com os três perfis de impacto mostrados nas {\figurename s} \ref{fig:modelo-imp-kron}, \ref{fig:modelo-imp-tri} e \ref{fig:modelo-imp-rect}.	
	Após avaliar o perfil de força de impacto mais apropriado, uma simulação com duração de quatro segundos é realizada para obter a resposta de vibração ao longo do tempo e o seu espectro de frequências.
	Em todos os testes, o modelo dinâmico foi resolvido com o algoritmo de solução -- denominado pelo termo em inglês \foreignlanguage{english}{solver} -- de Adams-Bashforth-Moulton de ordens 1 a 12.
	É um \foreignlanguage{english}{solver} de passo e ordem variáveis, e de passo múltiplo, ou seja, o cálculo da solução em um instante de tempo depende da solucão em diversos instantes anteriores \cite{ode113}.
	Além disso, uma frequência de amostragem \gls{s:freq-amostra} de $50\ \text{kHz}$ foi adotada de maneira igual nas duas situações de teste.
	
	\chapter{Resultados}
	O capítulo a seguir apresenta os resultados obtidos com o trabalho desenvolvido.
	A apresentação destes se divide em duas partes: a primeira apresenta os resultados quanto à convergência e desempenho do modelo proposto;
	a segunda compara a vibração simulada do rolamento com a literatura consultada e verifica a sua consistência qualitativa, conforme propunha o objetivo deste trabalho.
	
	\section{Convergência e desempenho do modelo}
	Durante o desenvolvimento do modelo, dois aspectos apresentaram problemas à convergência da solução.
	Um deles foi o aparecimento de componentes imaginárias na carga máxima \gls{s:rol-carga-maxima-elemento}, e o outro foi a incapacidade de convergência do modelo com um dos perfis de força de impacto.
	Entre os outros dois perfis de impacto testados, diferenças relevantes de desempenho foram percebidas.
	
	\subsection{Força de impacto}
	A força de impacto provocada pelo defeito, como mostrado no item \ref{sec:mdl:forca-impacto}, é igual à carga na esfera mais solicitada do rolamento, que é calculada pela função \textsc{ObterCargaMaximaEsfera} do código trazido no Apêndice \ref{apc:codigo}.
	Entre as variáveis utilizadas por este cálculo, a carga radial \gls{s:carga-radial} aplicada e a folga diametral do rolamento \gls{s:rol-folga-diam} demonstraram grande importância na consistência do resultado.
	
	Conforme a norma ISO, a faixa de folga radial para rolamentos da classe N com diâmetro interno de $18\ \text{a}\ 24\ \text{mm}$ vai de $5\ \text{a}\ 20\ \mu\text{m}$ \cite{skf6004}.
	Baseado nisso, um valor médio de $12,5\ \mu\text{m}$ foi escolhido inicialmente para a folga radial, resultando, portanto, em $25\ \mu\text{m}$ para a folga diametral \gls{s:rol-folga-diam}.
	
	Constatou-se então que os valores da carga máxima da esfera mais solicitada \gls{s:rol-carga-maxima-elemento} apresentavam uma componente imaginária, além de uma componente real inconsistente com a literatura consultada.
	Analisando a Equação \ref{eqn:mdl:carga-maxima}, nota-se que o único termo que pode dar origem a valores imaginários é $Z_w$.
	Uma investigação no cálculo deste termo, visto na Equação \ref{eqn:rol:zw-carga-radial}, identificou que valores para a deformação elástica máxima \gls{s:deformacao-maxima} menores do que a folga radial $\gls{s:rol-folga-diam}/2$ eram gerados ainda no começo do procedimento iterativo demonstrado no item \ref{sec:mdl:forca-impacto}.
	Isso torna claro o aparecimento de termos imaginários no numerador da fração na Equação \ref{eqn:rol:zw-carga-radial}.
	
	O resultado de uma investigação nos resultados do procedimento iterativo para o cálculo de \gls{s:rol-carga-maxima-elemento} é exibido na \figurename\ \ref{fig:modelo-converg-carga-max}.
	No gráfico, são exibidos valores da carga na esfera mais solicitada para diferentes valores da carga radial \gls{s:carga-radial}, iniciando a partir de uma folga radial $\gls{s:rol-folga-diam}/2$ nula até o máximo valor desta variável que trazia apenas valores reais para \gls{s:rol-carga-maxima-elemento}.
	Apesar de o aumento da carga máxima com o crescimento da carga radial ser um efeito óbvio, o resultado também cresce com o aumento da folga radial.
	Isso está de acordo com a explicação trazida em \cite{hamrock:1991}, de que o aumento da folga do rolamento promove uma acomodação menos uniforme da carga do anel interno sobre as esferas, de modo que a carga tende a ser suportada por menos elementos.
	
	\begin{figure}[t]
		\incluirimagem[0.75]{ModeloConvergCargaMaxima.png}{Carga máxima da esfera em função da carga radial e folga diametral}{o autor (\thedate)}
		\label{fig:modelo-converg-carga-max}
	\end{figure}

	De toda forma, o método para cálculo de \gls{s:rol-carga-maxima-elemento} demonstrou um limite para a seleção no valor da folga radial do rolamento. Por este motivo, adotou-se o valor de $8\ \mu\text{m}$, que resulta no valor de $16\ \mu\text{m}$ para \gls{s:rol-folga-diam} exibido na \tablename\ \ref{tbl:6004-gerais}.
	O novo valor foi escolhido por ser um dos valores inteiros mais próximos da folga determinada inicialmente, mas que ainda estão no intervalo de convergência do método de cálculo de \gls{s:rol-carga-maxima-elemento}.
	Adicionalmente, o novo valor determinado para a folga radial ainda está dentro da faixa de valores especificado pela norma.
	
	\subsection{Perfil da força de impacto}
	Dos três perfis para a função da força de impacto \gls{s:fun-forca-impacto} testados, apenas dois levaram o modelo a convergir.
	A utilização de impulsos unitários com o delta de Kronecker como exibido na \figurename\ \ref{fig:modelo-imp-kron} impedia o atingimento de uma solução.
	A investigação da causa vai além dos objetivos deste trabalho.
	Portanto, apenas as formas de pulso triangular (\figurename\ \ref{fig:modelo-imp-tri}) e retangular (\figurename\ \ref{fig:modelo-imp-rect}) foram consideradas.
	
	A resposta de vibração do anel externo para os outros dois perfis de força de impacto pôde ser calculada e os resultados apresentaram-se diferentes, como pode ser observado na \figurename\ \ref{fig:modelo-comp-pulsos}.
	A diferença entre os dois resultados se justifica pela diferença de forma e de largura entre as funções de força de impacto.
	Ao comparar o comportamento dos pulsos entre as {\figurename s} \ref{fig:modelo-imp-tri} e \ref{fig:modelo-imp-rect}, é perceptível a maior largura do pulso triangular.
	A diferença nessa medida implica uma área maior sob o gráfico da função \gls{s:fun-forca-impacto} o que, por integração da definição da força como $F = m\cdot a$, indica maior crescimento da velocidade do corpo em que a força foi aplicada.
	Por conseguinte, o deslocamento tem um aumento momentâneo em apenas uma direção vertical do gráfico.
	
	\begin{figure}[b]
		\incluirimagem[0.42]{ModeloResultadosPulsos.png}{Deslocamento do anel externo para diferentes forças de impacto}{o autor (\thedate)}
		\label{fig:modelo-comp-pulsos}
	\end{figure}

	Já a forma do impulso é a resposta para a maior amplitude observada no deslocamento provocado pelo pulso retangular, quando desconsiderado o pico assimétrico na direção vertical positiva do outro gráfico.
	A característica retangular da variação da força indica um valor muito alto -- teoricamente, infinito -- da derivada da aceleração, chamada de impulso ou \foreignlanguage{english}{jerk} \cite{norton:2010}.
	Essa variação abrupta de aceleração provocou um deslocamento maior, visto que o conjunto dos parâmetros de rigidez e amortecimento (que são os mesmos entre os dois testes) definem o tempo de resposta de um sistema dinâmico \cite{lutz:2005}.
	
	Além da diferença na resposta de vibração, houve uma variação expressiva no custo computacional da solução entre as diferentes formas de força de impacto.
	Ao passo de que a simulação com pulsos triangulares durante $0,5$ segundo levou uma média de $184,03$ segundos entre três execuções, o teste da função com forma retangular, sob as mesmas condições, teve um tempo médio de $21,16$ segundos, o que representa uma redução de $88,5\%$ na duração da resolução do modelo.
	Dada a diferença no custo computacional e uma maior similaridade da resposta de deslocamento com o modelo apresentado em \cite{sassi:2007}, adotou-se para a simulação completa da vibração do rolamento a função da força de impacto \gls{s:fun-forca-impacto} com pulsos retangulares.

	\section{Resposta de vibração simulada}
	Na simulação com os parâmetros selecionados, foram obtidos os resultados para o deslocamento do anel externo do rolamento ao longo do tempo, além dos espectros de amplitude e de envelope, da aceleração do anel externo.
	A análise destes indicadores e comparação às referências consultadas é feita em duas partes.
	A primeira analisará o sinal ao longo do tempo, e a segunda caracterizará os resultados obtidos no domínio da frequência.
	
	\subsection{Análise de tempo} \label{sec:res:analise-tempo}
	A \figurename\ \ref{fig:modelo-ext-tempo} apresenta a resposta de deslocamento do anel externo à vibração do sistema.
	É possível observar um comportamento repetitivo de picos de deslocamento, causados pelos sucessivos impactos de uma esfera do rolamento com o defeito modelado.
	O intervalo de tempo entre cada pico corresponde ao período de \gls{s:freq-rol-def-out}, indo ao encontro da observação experimental em \cite{cong:2013}, de que os picos do gráfico ao longo do tempo se apresentam na frequência característica do defeito presente no rolamento.
	
	\begin{figure}[t]
		\incluirimagem[0.75]{ModeloExtPosTempo.png}{Deslocamento do anel externo ao longo do tempo}{o autor (\thedate)}
		\label{fig:modelo-ext-tempo}
	\end{figure}
	
	Enquanto os picos deste sinal têm uma variação senoidal nos resultados das referências \cite{mcfadden:1984,sassi:2007,cong:2013,patil:2010}, a resposta de vibração deste modelo possui valores de pico praticamente constantes.
	Isso se deve à característica da carga no momento do impacto, conforme demonstrado em \cite{mcfadden:1984,cong:2013}.
	No caso das referências consultadas, os modelos previam um defeito no anel interno do rolamento com uma aplicação constante de carga; ou um defeito no anel externo com uma carga oriunda de um rotor desbalanceado, cujo vetor de força rotaciona na mesma velocidade angular do eixo.
	O modelo desenvolvido neste trabalho prevê uma condição mais simples, com o defeito localizado no anel externo, fixo, e uma aplicação de carga constante.
	
	É visível também que, nos resultados exibidos na \figurename\ \ref{fig:modelo-ext-tempo}, a posição do anel externo não atinge a posição zero até a ocorrência do impacto seguinte, diferentemente do modelo desenvolvido em \cite{sassi:2007}.
	Entre as possíveis causas dessa diferença pode-se citar a diferente velocidade de rotação empregada ($720$ \gls{sig:RPM} na referência contra $1800$ \gls{sig:RPM} no modelo desenvolvido);
	e diferentes métodos para obtenção dos valores de amortecimento para os filmes de lubrificante.
	Em \cite{sassi:2007}, um conjunto de parâmetros para a lubrificação hidrodinâmica de mancais de deslizamento é utilizado, em vez da teoria \gls{sig:EHL}, para a obtenção desta propriedade.
	
	\subsection{Análise de frequência}
	Como um panorama, o espectro de amplitude exibido na \figurename\ \ref{fig:modelo-ext-ampspec} decompõe as frequências presentes no sinal de aceleração do anel externo.
	A linha mais destacada deste espectro encontra-se a uma frequência de $5260\ \text{Hz}$, bastante próxima à frequência natural do anel externo, de $5307\ \text{Hz}$.
	Isso alinha-se com o resultado do modelo proposto em \cite{sassi:2007}, no qual o espectro de frequências possui uma componente destacada em uma alta frequência, indicada como a frequência de ressonância do rolamento.
	
	\begin{figure}[ht]
		\incluirimagem[0.7]{ModeloExtAccAmpSpectrum.png}{Espectro de frequências do anel externo}{o autor (\thedate)}
		\label{fig:modelo-ext-ampspec}
	\end{figure}

	Para eliminar a interferência da frequência natural do rolamento na obtenção de informações sobre possíveis falhas, as referências consultadas empregam o espectro de envelope, que destacará as componentes de baixa frequência em que se apresentam os defeitos.
	De maneira correspondente, o espectro de envelope para o sinal de aceleração do anel externo é exibido na \figurename\ \ref{fig:modelo-ext-envspec-total}.

	\begin{figure}[ht]
		\incluirimagem[0.7]{ModeloExtAccEnvSpectrum.png}{Espectro de envelope do anel externo}{o autor (\thedate)}
		\label{fig:modelo-ext-envspec-total}
	\end{figure}

	Uma ampliação no espectro de envelope da \figurename\ \ref{fig:modelo-ext-envspec-total} é exibido na \figurename\ \ref{fig:modelo-ext-envspec-600hz}.
	Nele, é possível verificar a existência de linhas altas igualmente espaçadas.
	Estas linhas estão em frequências múltiplas de $107,35\ \text{Hz}$, o que corresponde exatamente a \gls{s:freq-rol-def-out}.
	A presença de linhas mais altas na frequência característica do defeito vai ao encontro das observações feitas nas referências \cite{mcfadden:1984,sassi:2007,patil:2010,cong:2013}, tanto no desenvolvimento de seus modelos como em seus testes experimentais.

	\begin{figure}[ht]
		\incluirimagem[0.7]{ModeloExtAccEnvSpec600Hz.png}{Espectro de envelope do anel externo, ampliado}{o autor (\thedate)}
		\label{fig:modelo-ext-envspec-600hz}
	\end{figure}

	Uma característica que é pontuada nos testes experimentais de \cite{mcfadden:1984} e também nos modelos e testes de \cite{sassi:2007,cong:2013} é a presença de bandas laterais à frequência característica do defeito e suas harmônicas no espectro de envelope.
	Estas bandas laterais distam das linhas principais -- na frequência do defeito -- o equivalente à frequência de rotação do eixo.
	Uma constatação feita em \cite{mcfadden:1984} na verificação experimental é que no espectro de envelope também haverá uma linha de altura semelhante à das bandas laterais, porém, isolada na frequência de rotação do eixo.
	Ao observar as linhas de espectro nas frequências múltiplas de \gls{s:freq-rol-def-out} na \figurename\ \ref{fig:modelo-ext-envspec-600hz}, seria razoável afirmar que as bandas laterais mencionadas nas referências não foram evidenciadas.
	Uma ampliação posterior do espectro, mostrada na \figurename\ \ref{fig:modelo-ext-envspec-sidebands}, indica que estas linhas foram geradas, mas possuem uma amplitude quase desprezível.
	Além da baixa amplitude, a frequência em que aparecem destoa da frequência de rotação do eixo.
	Enquanto a rotação de 1800 \gls{sig:RPM} impressa corresponde a uma frequência de rotação de $30\ \text{Hz}$, a linha isolada aparece em $23,5\ \text{Hz}$.
	O padrão de distância, no entanto, acabou preservado: as bandas laterais distam da linha principal os mesmos $23,5\ \text{Hz}$.
	A causa para o aparecimento destas linhas em uma frequência diferente daquela de rotação do eixo não foi identificada.

	Embora o aparecimento das bandas laterais em amplitudes baixas pareça um indício de erro no modelo, o aumento destas linhas laterais e da linha inicial isolada na frequência do eixo está diretamente ligado à carga oscilante \cite{cong:2013}.
	Como mencionado no item \ref{sec:res:analise-tempo}, a carga oscilante pode aparecer com um defeito no anel interno combinado a uma carga fixa, ou a um defeito no anel externo associado a uma carga girante, como um rotor desbalanceado.
	No segundo caso, uma carga desbalanceada muito grande pode inclusive tornar as bandas laterais dominantes no espectro de envelope, em relação à banda na frequência característica do defeito.
	Uma vez que a carga adotada no desenvolvimento deste modelo é constante e a falha se encontra no anel externo do rolamento, também fixo, é razoável esperar que a altura das bandas laterais e da banda inicial sejam minimizados.
	
	\begin{figure}[H]
		\incluirimagem[0.7]{ModeloExtAccEnvSpecSideBands.png}{Bandas laterais no espectro de envelope}{o autor (\thedate)}
		\label{fig:modelo-ext-envspec-sidebands}
	\end{figure}
	
	\chapter{Conclusão} \label{sec:conclusao}
	No presente trabalho, o desenvolvimento completo de um modelo para a predição de falha em um rolamento defeituoso foi realizado.
	A simulação foi conduzida em uma condição específica de falha e carregamento para fins de verificação.
	O código disponibilizado no Apêndice \ref{apc:codigo} deste documento, no entanto, é aplicável a outras condições de falha sem a necessidade de grandes adaptações.
	A expansão da funcionalidade deste modelo a outras formas construtivas de rolamento exigirá uma nova análise das condições de contato analisadas no item \ref{sec:contato-seco}, mas é de toda maneira cabível ao trabalho já desenvolvido.
	
	A determinação de propriedades elencada como objetivo foi parcialmente atingida, de modo que informações dos componentes foram completamente obtidas e a interação destes como contato seco foi plenamente caracterizada.
	A investigação de parâmetros relacionados ao filme de lubrificante sob o regime de \gls{sig:EHL} não foi realizada profundamente dada a complexidade do assunto em relação ao escopo amplo do trabalho.
	
	Diferentes modelos para a força de excitação provocada pelos impactos das esferas do rolamento com o defeito foram experimentados, possibilitando a escolha daquele que demonstrou maior eficiência na simulação, na questão de custo computacional.
	
	A resolução completa do modelo dinâmico não-linear permitiu a comparação com a literatura consultada, possibilitando identificar semelhanças com outros modelos desenvolvidos no sinal de vibração ao longo do tempo.
	Em especial nos resultados obtidos no domínio da frequência, uma sintonia grande com as observações teóricas e experimentais das referências consultadas identificou que o comportamento do modelo é fiel à realidade para o modo de falha considerado, ao menos em caráter qualitativo.
	
	Uma verificação da precisão quantitativa do modelo depende de uma análise experimental.
	Visto que as dimensões do rolamento ensaiado variavam entre cada referência consultada, não foi possível obter um padrão quantitativo a seguir e estabelecer como meta para o trabalho.
	Não obstante, a experimentação do modelo desenvolvido com outras condições de carregamento e defeito é necessária para concluir mais assertivamente a respeito da sua confiabilidade.
	
	De toda maneira, o modelo apresentou resultados satisfatórios, concordantes com a literatura e demonstrando semelhança qualitativa com a teoria da vibração em rolamentos defeituosos.
	Com o objetivo atingido, este trabalho estabelece o primeiro passo para um desenvolvimento que pode ser tanto ampliado para outras dimensões de rolamento e condições de falha; como aprofundado para uma maior compreensão de fenômenos complexos associados à dinâmica de rolamentos, como a teoria da \gls{sig:EHL}.
	
	\chapter{Trabalho futuro}
	O modelo desenvolvido configura-se como um trabalho amplo, com diversas possibilidades tanto de expansão de escopo como de aprofundamento de detalhe, como mencionado no capítulo \ref{sec:conclusao}.
	Além disso, uma necessidade imediata do trabalho é a sua verificação experimental.
	
	Portanto, três linhas de desenvolvimento futuro diferentes são sugeridas como continuação do modelo criado: a primeira é o teste de um rolamento com defeito na pista externa, a coleta dos dados experimentais e a comparação com os resultados gerados pelo modelo.
	Embora os resultados atuais possuam uma boa confiabilidade no caráter quantitativo, não é possível saber quanto a amplitude dos valores gerados dista de uma medição em situação real.
	
	A segunda possibilidade é a experimentação do modelo com uma condição de carga diferente.
	É de especial interesse o teste de uma condição em que a força do impacto tenha uma característica oscilante ao longo do tempo.
	Isso é obtido ao modelar um defeito na pista interna com uma carga constante, ou manter o defeito na pista externa (fixa) e tornar a carga oriunda de um rotor desbalanceado, por exemplo.
	No caso deste, a carga radial girará na mesma velocidade angular do eixo.
	Visto que o código atual demandaria apenas a adaptação da força de impacto constante para um comportamento oscilante, entende-se que seja possível combinar este trabalho a uma verificação experimental.
	
	Por fim, a terceira linha de trabalho seria o aprofundamento das características do modelo em relação à dinâmica do contato entre os componentes do rolamento.
	Os conceitos complexos da teoria da \gls{sig:EHL} não puderam ser rigorosamente explorados neste trabalho, e o cálculo de tais valores de rigidez e amortecimento para a camada de filme no rolamento envolve um trabalho avançado de modelagem numérica, cuja comprovação experimental é complexa.
	De toda maneira, a presença desse regime de lubrificação no rolamento é comprovado, e o seu estudo para adaptação no modelo o tornaria mais próximo do comportamento real.
	
	\postextual
	
	\bibliography{Bibliografia}
	
	\apendices
	\chapter{Código MATLAB\textsuperscript{\textregistered} do modelo} \label{apc:codigo}
	Este apêndice contém o código desenvolvido para a solução do problema de modelagem numérica estudado neste trabalho.
	Cada bloco de código corresponde a um arquivo independente, sendo o primeiro deles, contido no arquivo \texttt{FalhaPistaExterna.m}, o \foreignlanguage{english}{script} principal.
	Adicionalmente, os arquivos estão armazenados em um repositório \emph{on-line} de código aberto, podendo ser descarregados a partir do endereço \url{https://github.com/HenriqueBaron/tcc-modelo/releases}.
	
	Arquivo \texttt{FalhaPistaExterna.m}:
	\lstinputlisting[style=Matlab-editor]{../Modelo-copia/FalhaPistaExterna.m}
	
	Arquivo \texttt{GerarDadosEntrada.m}:
	\lstinputlisting[style=Matlab-editor]{../Modelo-copia/GerarDadosEntrada.m}
	
	Arquivo \texttt{RaiosCurvatura.m}:
	\lstinputlisting[style=Matlab-editor]{../Modelo-copia/RaiosCurvatura.m}
	
	Arquivo \texttt{ParametrosElipseContato.m}:
	\lstinputlisting[style=Matlab-editor]{../Modelo-copia/ParametrosElipseContato.m}
	
	Arquivo \texttt{ObterCargaMaximaEsfera.m}:
	\lstinputlisting[style=Matlab-editor]{../Modelo-copia/ObterCargaMaximaEsfera.m}
	
	Arquivo \texttt{DeformacaoElastica.m}:
	\lstinputlisting[style=Matlab-editor]{../Modelo-copia/DeformacaoElastica.m}
	
	Arquivo \texttt{ImpulsosTriangulo.m}:
	\lstinputlisting[style=Matlab-editor]{../Modelo-copia/ImpulsosTriangulo.m}
	
	Arquivo \texttt{EspessuraFilmeLub.m}:
	\lstinputlisting[style=Matlab-editor]{../Modelo-copia/EspessuraFilmeLub.m}
	
	Arquivo \texttt{SisNaoLinOrdem2.m}:
	\lstinputlisting[style=Matlab-editor]{../Modelo-copia/SisNaoLinOrdem2.m}
	
	Arquivo \texttt{ExibirResultados.m}:
	\lstinputlisting[style=Matlab-editor]{../Modelo-copia/ExibirResultados.m}
\end{document}