\documentclass[12pt,oneside,english,brazil,lmodern,siglas,simbolos,cite=num]{ucsmonograph}

\ifluatex
	\usepackage{fontspec}
\else
	\usepackage[utf8]{inputenc}
	\usepackage[T1]{fontenc}
\fi

\usepackage[brazil]{babel}
\usepackage{graphicx} % Inserção de imagens
\graphicspath{{../Imagens/}} % Definição do caminho para as imagens
\usepackage{pdfpages} % Permite a inclusão de páginas em PDF dentro do documento. Será útil para incluir uma cópia digitalizada da folha de aprovação, após as assinaturas.
\usepackage{amsmath} % Digitação de alguns símbolos matemáticos.
\usepackage{bigints} % Inclui símbolos de integral grandes, para frações etc.
\usepackage{todonotes}
\includeonly{PreTexto} % Inclusão do arquivo de elementos pré-textuais sem deixar páginas em branco

\newcommand{\unidademassa}{[kg]}
\newcommand{\unidademassalinear}{[kg/m]}
\newcommand{\unidadeamortecimento}{[N\cdot s/m]}
\newcommand{\unidaderigidez}{[N/m]}
\newcommand{\unidadeforca}{[N]}
\newcommand{\unidadeposicao}{[m]}
\newcommand{\unidadevelocidade}{[m/s]}
\newcommand{\unidadeaceleracao}{[m/s\textsuperscript{2}]}
\newcommand{\unidadetempo}{[s]}
\newcommand{\unidadeenergia}{[J]}
\newcommand{\unidadeangulo}{[rad]}
\newcommand{\unidadevelocidadeangular}{[rad/s]}
\newcommand{\unidadefrequencia}{[Hz]}
\newcommand{\unidadetorque}{[N\cdot m]}
\newcommand{\unidadearea}{[m\textsuperscript{2}]}
\newcommand{\semunidade}{[-]}
\newcommand{\unidadepressao}{[Pa]}
\newcommand{\unidadesegundomomento}{[m\textsuperscript{4}]} % Arquivo que contém os símbolos.

\titulo{Modelagem numérica de vibração em máquinas rotativas}
\autor{Henrique Baron}
\data{2018}
\instituicao{Universidade de Caxias do Sul}
\local{Caxias do Sul}
\preambulo{Trabalho de conclusão de curso apresentado à Universidade de Caxias do Sul como requisito parcial à obtenção do grau de Engenheiro Mecânico.
Área de concentração: Projetos de Máquinas: Estática e Dinâmica Aplicada.}
\orientador{prof. Me. Paulo Roberto Linzmaier}
\palavraschave{Vibra\c{c}\~{a}o. Modelagem. Rolamentos. Desbalanceamento.}

\incluirsiglas{Siglas} % Inclui o arquivo bibtex com as siglas.
\incluirsimbolos{Simbolos} % Inclui o arquivo bibtex com os símbolos.

\begin{document}
	% Capa e folha de rosto
\imprimircapa
\imprimirfolhaderosto
\clearpage

% Folha de aprovação
\imprimirfolhadeaprovacao{29/11/2018}

% Agradecimentos
\begin{agradecimentos}
%	Gostaria de agradecer primeiramente à minha família pelo enorme suporte e incentivo ao desenvolvimento deste trabalho, em especial nos momentos de maior dificuldade, no qual fizeram tudo o que estava ao seu alcance para me ajudar a superar os problemas que enfrentava.
%	A eles e à minha namorada, Fernanda, agradeço a compreensão e paciência durante este período extenuante de estudo e entrega.
%	
%	Às professoras Dra. Kátia Cavalca e Dra. Isolda Gianni de Lima, agradeço pela prontidão e disposição em me ajudar nos momentos em que eu não pude encontrar uma resposta para minhas dúvidas.
%	
%	Agradeço também à empresa Auttom Automação e Robótica, por propiciar a oportunidade de desenvolvimento pessoal com o tema deste trabalho, e pelo suporte que sempre deu à condução dos meus estudos.
\end{agradecimentos}

% Resumo em português
\begin{resumo}
	\SingleSpacing
	O emprego de mancais de rolamento se aplica desde sistemas mecânicos simples até o maquinário de alta complexidade e precisão.
	Sendo a análise de vibrações o meio utilizado para identificação precoce de falhas nestas unidades, é de grande importância determinar o comportamento deste componente em situação de defeito.
	Com este foco, este trabalho concentra-se no desenvolvimento de um modelo numérico não-linear de três graus de liberdade para a resposta de vibração de um rolamento com um defeito pontual no anel externo.
	Para isso, são empregados os conceitos da teoria \emph{hertziana} de deformação por contato, de modo a determinar a força sobre um elemento rolante quando este incide sobre o defeito.
	O valor de força é utilizado para modelar três perfis de pulso diferentes para a carga de impacto entre a esfera do rolamento e o defeito.
	Ao mesmo tempo, a influência do filme de fluido lubrificante no contato entre esferas e anéis é analisado sob os fundamentos da teoria de lubrificação \emph{elastohidrodinâmica}.
	O modelo desenvolvido é investigado quanto à sua estabilidade e convergência, observando o custo computacional das diferentes formas de pulso para a carga de impacto.
	A resposta de vibração simulada é comparada a dados de simulações e verificações experimentais de trabalhos anteriores no mesmo campo de aplicação, analisando-a no domínio do tempo e no domínio da frequência, para concluir quanto à consistência qualitativa do modelo.
	\vspace{\onelineskip}
	
	\noindent
	\textbf{Palavras-chave}: Vibração. Modelagem. Defeito. Rolamento.
\end{resumo}


% Resumo em inglês
\begin{resumo}[\normalsize\bfseries ABSTRACT]
	\SingleSpacing
	\begin{otherlanguage}{english}
		The application of rolling element bearings ranges from simple mechanical systems to high complexity and precision machinery.
		Since the vibration analysis is the main way to early identify failure in these units, it is of great importance to determine the vibrational pattern of this component in a defect situation.
		An experimental approach for the analysis would offer a strict condition for the problem investigation.
		For this reason, the construction of a numeric model is considered to be a more extensive way to understand the phenomenon, while still eliminating interferences caused by the test system.
		With that in focus, the present work has as objective to model the vibration of damaged rolling element bearings and identify the main characteristics of this behavior.
		\vspace{\onelineskip}
		
		\noindent
		\textbf{Keywords}: Vibration. Model. Defect. Bearing.
	\end{otherlanguage}
\end{resumo}
% Listas de figuras, quadros, tabelas e siglas
\listoffigures*
\cleardoublepage

%\listofquadros*
%\cleardoublepage

\listoftables*
\cleardoublepage

\listofsiglas*
\cleardoublepage

\listofsimbolos*
\cleardoublepage

% Sumário
\tableofcontents*
\cleardoublepage

	\textual % Comando que define que os elementos textuais começaram, para incluir a numeração de páginas.
	
	\chapter{Introdução}
	Em todo sistema produtivo em que se faz o uso de ferramentas ou máquinas, a manutenção é um tema que recebe grande atenção, dada a necessidade de manter o funcionamento de equipamentos com o menor tempo possível de parada.
	Aliado a isso, busca-se continuamente encontrar um balanço entre o custo de repetidas paradas para inspeção e o de uma eventual falha ocasionada por falta de acompanhamento.
	O montante gasto em manutenção pelas empresas no Brasil no ano de 2013 foi equivalente a 4,69\% do PIB nacional, correspondente a um valor de R\$ 206,5 bilhões \cite{seleme:2015}.
	
	Dentre as metodologias de manutenção existentes, há uma divisão em três grupos principais: corretiva, preditiva e preventiva.
	Este último, que se configura pelo planejamento de uma rotina de manutenção e inspeções baseadas em intervalos regulares de tempo, ainda é a mais comum na indústria atualmente.
	Porém, a realização de manutenção baseada no tempo vem mostrando uma baixa confiabilidade nos últimos anos.
	Em um teste realizado pelo SKF Group no qual 30 rolamentos de esferas idênticos foram testados até a falha, o tempo de vida das unidades variou de 15 horas até mais de 300 horas \cite{hashemian:2011}.
	 
	Da mesma forma, dados do \gls{sig:NREL} dos Estados Unidos apontam que 76\% dos casos de falha em caixas de engrenagens de aerogeradores são causados por rolamentos, sendo os defeitos nas próprias engrenagens a segunda maior causa em número de ocorrências (17,1\%).
	Ao mesmo tempo, apenas 10\% dos rolamentos atingem a sua vida útil esperada por conta da diversidade de condições de desgaste prematuro às quais estes componentes podem ser submetidos \cite{peeters:2018}.
	Sendo assim, mesmo com uma expectativa de vida estimada para partes de uma máquina, é impossível determinar de maneira exata a sua durabilidade.
	Isso corrobora, portanto, que a aplicação de uma metodologia de manutenção com inspeções periódicas não elimina o risco de paradas provocadas por falhas inesperadas.
	
	Atualmente, o aumento da oferta de produtos na área de instrumentação e coleta de dados possibilita à indústria o crescimento do uso de técnicas de manutenção preditiva para análise de condições dos equipamentos.
	A grande vantagem da manutenção preditiva é que, diferentemente da preventiva, o gatilho para uma intervenção não é um intervalo de tempo, mas sim a mudança em algum sinal emitido pela máquina, que pode ser o indício de um defeito.
	Entre as técnicas existentes, a medição da vibração em pontos específicos de um sistema traz a vantagem de fornecer sinais de mudança na condição do equipamento em um estágio inicial do possível problema \cite{al-najjar:2004}.
	
	O campo de aplicação mais desenvolvido para a detecção de falhas através de vibrações é o monitoramento de máquinas rotativas.
	Sua metodologia de análise é baseada no reconhecimento de padrões de frequência e amplitude dos sinais medidos durante o funcionamento normal do equipamento \cite{carden:2004}.
	Componentes ligados ao eixo de um motor, por exemplo, emitirão vibrações com uma frequência igual ou múltipla da frequência de rotação do eixo.
	Uma vez entendido um padrão de comportamento, mudanças em propriedades físicas do sistema -- como rigidez, amortecimento ou distribuição de massa -- causarão alterações detectáveis nos dados coletados \cite{qiao:2011}, permitindo identificar a origem da variação no sinal analisado.
	Isso faz da análise de vibração um dos principais artifícios da manutenção preditiva nestes equipamentos.
	
	\section{Justificativa}\label{sec:justificativa}
	Dada a larga utilização de rolamentos em aplicações de engenharia, é essencial o conhecimento sobre o padrão de vibração provocado pelo componente em situação de defeito.
	Determinar o comportamento do sinal lido em tal condição permite identificar de fato a presença de uma falha no componente em um estágio inicial, oferecendo a possibilidade de programar uma troca dessa unidade muito antes do seu colapso.
	
	Entre as possibilidades para identificar o comportamento de rolamentos que apresentam falhas, pode-se mencionar a modelagem numérica ou a avaliação puramente experimental.
	No presente trabalho, considera-se o desenvolvimento de um modelo numérico, visto que este permite identificar de maneira aprofundada a origem de cada parte do sinal obtido, permitindo ainda expandir a análise para outras situações de defeito, carga e velocidades de rotação.
	Tal expansão da investigação em uma avaliação puramente experimental seria dispendiosa e muito suscetível a variáveis externas relativas ao sistema de teste.
	
	\section{Ambiente de desenvolvimento}	
	O trabalho foi conduzido no departamento de Pesquisa e Desenvolvimento da unidade de educação da empresa Auttom Automação e Robótica Ltda., situada em Caxias do Sul.
	A empresa possui duas divisões: uma concentrada em projetos de automação sob demanda; e a outra, onde o trabalho foi realizado, desenvolve bancadas didáticas e sistemas de ensino para escolas técnicas e universidades.
	
	Atualmente, a unidade de educação da empresa conta com uma linha ampla de produtos nas áreas de automação industrial, energias renováveis, eletricidade e refrigeração.
	Ainda não há uma oferta considerável de produtos para estudo de mecânica nas áreas de estática ou dinâmica, e o desenvolvimento deste trabalho surge como uma possibilidade de inserção da empresa em um mercado ainda pouco explorado.
	
	\section{Objetivos}
	
	\subsection{Objetivo geral}
	O objetivo do presente trabalho é modelar numericamente a vibração provocada por rolamentos defeituosos.
	Como resultado, espera-se uma consistência qualitativa do sinal modelado com a literatura empregada.
	A análise quantitativa dos resultados obtidos demanda a construção de um sistema de teste experimental, que foge do escopo deste trabalho.
	
	\subsection{Objetivos específicos} \label{sec:objetivos:especificos}
	O objetivo apresentado é distribuído entre os seguintes objetivos específicos:
	\begin{enumerate}
		\item Determinação das propriedades de cada componente do rolamento a ser modelado;
		\item Cálculo da distribuição de carga entre os elementos rolantes;
		\item Determinação do comportamento da força de excitação provocada pelo defeito no componente;
		\item Modelagem dinâmica dos componentes, obtendo valores para deslocamento, velocidade e aceleração ao longo do tempo;
		\item Obtenção do espectro de frequências do sinal modelado.
	\end{enumerate}

	\chapter{Referencial teórico}
	Neste capítulo são apresentados os conceitos inerentes à modelagem e análise do protótipo desenvolvido.
	
	\section{Análise harmônica}
	Qualquer movimento que se repita em intervalos de tempo iguais é denominado movimento periódico \cite{rao:2008}.
	Dentre eles, entende-se por movimento harmônico simples aquele que é definido por uma função do tipo \cite{timoshenko:1974}
	\begin{equation}
		g(t) = a\cos\omega t + b\sin\omega t
	\end{equation}
	onde \gls{s:funcao-tempo} indica a posição ou deslocamento, \gls{s:freq-rad} é a frequência da oscilação, \gls{s:tempo} denota o tempo e $ a $ e $ b $ são duas constantes quaisquer.
	
	Embora este seja o tipo de movimento mais simples de tratar, muitos dos sistemas vibratórios não exibem oscilação harmônica \cite{rao:2008}.
	No entanto, uma função periódica pode ser representada em termos da soma de suas componentes harmônicas através da série de Fourier \cite{clark:1972}.
	
	\subsection{Série de Fourier} \label{sec:fourier:serie}
	Seja \gls{s:funcao-tempo} uma função periódica com período \gls{s:periodo}, a série de Fourier é definida como \cite{spiegel:1977}
	\begin{equation}\label{eqn:fourier:serie}
		g(t) = \frac{a_0}{2} + \sum_{n=1}^{\infty}\left(a_n\cos n\omega t + b_n\sin n\omega t \right)
	\end{equation}
	sendo a frequência \gls{s:freq-rad} relacionada ao período \gls{s:periodo} por \cite{dimarogonas:1995}
	\begin{equation}\label{eqn:fourier:periodo}
		\omega = \frac{2\pi}{\tau}
	\end{equation}
	e os coeficientes \glspl{s:coef-fourier} representam as harmônicas \cite{dimarogonas:1995}, que valem \cite{spiegel:1977}
	\begin{align}\label{eqn:fourier:coefs}
		a_0 &= \frac{2}{\tau}\int_{0}^{\tau}g(t)\ dt \notag\\
		a_n &= \frac{2}{\tau}\int_{0}^{\tau}g(t)\cos n\omega t\ dt \\
		b_n &= \frac{2}{\tau}\int_{0}^{\tau}g(t)\sin n\omega t\ dt \notag
	\end{align}
	
	Embora seja possível a discussão sobre a convergência da série Fourier na Equação \ref{eqn:fourier:serie} para uma função \gls{s:funcao-tempo}, isso não é tratado aqui, visto que as condições de convergência desta série são satisfeitas nos problemas de ciência e engenharia, em geral \cite{spiegel:1977}.
	
	Utilizando as identidades de Euler
	\begin{equation}\label{eqn:fourier:ident-euler}
		e^{i\omega} = cos\,\omega + i\sin\omega\, , \quad e^{-i\omega} = cos\,\omega - i\sin\omega
	\end{equation}
	a série de Fourier da Equação \ref{eqn:fourier:serie} também pode ser representada na sua forma complexa como \cite{spiegel:1977}
	\begin{equation}\label{eqn:fourier:serie-compl}
		g(t) = \sum_{n=-\infty}^{\infty}\alpha_n\,e^{in\omega t}
	\end{equation}
	onde a harmônica \gls{s:coef-fourier-compl} é calculada por \cite{dimarogonas:1995}
	\begin{equation}\label{eqn:fourier:coefs-compl}
		\alpha_n = \frac{1}{2}(a_n - ib_n) = \frac{1}{\tau}\int_{0}^{\tau}g(t)\,e^{-in\omega t}\ dt
	\end{equation}
	
	\subsection{Representação no domínio da frequência} \label{sec:espectros-frequencia}
	A série de Fourier permite a representação de uma função periódica no domínio da frequência \cite{rao:2008}.
	A Figura \ref{fig:funcao-periodica} mostra, por exemplo, o gráfico de $ g(t) = \sin2\pi t + 0,\!7\cos4\pi t + 2 $ no domínio do tempo.
	Pode-se aplicar a Equação \ref{eqn:fourier:coefs-compl} um número determinado de vezes para representar a função no domínio da frequência com um espectro de dois lados \cite{dimarogonas:1995}.
	Neste gráfico, mostrado na Figura \ref{fig:espectro-2sided}, o eixo horizontal representa os múltiplos inteiros $ n $ da frequência \gls{s:freq-rad} utilizada na Equação \ref{eqn:fourier:coefs-compl}, e o eixo vertical é o próprio valor da harmônica \gls{s:coef-fourier-compl} calculada.
	\begin{figure}[b]
		\incluirimagem{FourierFuncao.png}{Função periódica no domínio do tempo}{o autor (\thedate)}
		\label{fig:funcao-periodica}
	\end{figure}
	\begin{figure}[t]
		\incluirimagem{FourierEspectro2sided.png}{Espectro de dois lados de uma função periódica}{o autor (\thedate)}
		\label{fig:espectro-2sided}
	\end{figure}
	
	De maneira semelhante, o espectro de frequência da Figura \ref{fig:espectro-2sided} pode ser representado por um espectro de frequências de um lado como mostra a Figura \ref{fig:espectro-1sided}, onde a harmônica \gls{s:coef-fourier-compl} de ordem $ n $ é somada com a harmônica de ordem $ -n $ \cite{randall:1987}.
	É importante observar que a componente constante, isto é, $ n=0 $ permanece com o mesmo valor do espectro de dois lados.
	\begin{figure}[t]
		\incluirimagem{FourierEspectro1sided.png}{Espectro de um lado de uma função periódica}{o autor (\thedate)}
		\label{fig:espectro-1sided}
	\end{figure}
	
	\subsection{Integral de Fourier}
	Para o caso de uma função não-periódica, toma-se a definição formal de que $ \tau\to\infty $, fazendo com que a série de Fourier torne-se uma integral de Fourier \cite{spiegel:1977}.
	Seja \gls{s:funcao-tempo} uma função seccionalmente contínua em qualquer intervalo finito e absolutamente integrável em $ (-\infty ,\,\infty) $, a integral de Fourier é definida como \cite{spiegel:1977}
	\begin{equation}\label{eqn:fourier:integral}
		g(t) = \int_{0}^{\infty}\left\lbrace a_\omega\cos\omega t + b_\omega\:\sin\omega t\right\rbrace\, d\omega
	\end{equation}
	tal que os coeficientes \glspl{s:coef-fourier-int} são determinados por
	\begin{align}
		a_\omega = \frac{1}{\pi}\int_{-\infty}^{\infty}g(t)\cos\omega t\ dt \notag\\
		b_\omega = \frac{1}{\pi}\int_{-\infty}^{\infty}g(t)\sin\omega t\ dt
	\end{align}
	
	\subsubsection{Transformada de Fourier} \label{sec:fourier:transform}
	Aplicando na Equação \ref{eqn:fourier:integral} a identidade de Euler da Equação \ref{eqn:fourier:ident-euler} de maneira semelhante ao item \ref{sec:fourier:serie}, obtém-se a transformada de Fourier \cite{savi:2017} de \gls{s:funcao-tempo}, denotada por \gls{s:funcao-freq}
	\begin{equation} \label{eqn:fourier:transform}
		G(\omega) = \mathcal{F}\{g(t)\} = \int_{-\infty}^{\infty}g(t)\ e^{-i\omega t}\:dt
	\end{equation}
	que também é denominada como transformada contínua de Fourier \cite{dimarogonas:1995}.
	A transformada inversa de Fourier, por sua vez, é dada por \cite{spiegel:1977}
	\begin{equation} \label{eqn:fourier:transform-inv}
		g(t) = \mathcal{F}^{-1}\{G(\omega)\} = \frac{1}{2\pi}\int_{-\infty}^{\infty}G(\omega)\ e^{i\omega t}\:d\omega
	\end{equation}
	
	\subsection{Métodos numéricos}
	Na transformada de Fourier apresentada no item \ref{sec:fourier:transform}, a definição matemática considera o fenômeno de oscilação acontecendo em um intervalo de tempo contínuo.
	É muito comum nos instrumentos de medição modernos, no entanto, que os dados sejam descritos em um intervalo de tempo discreto.
	Isto é, a função -- de vibração, por exemplo -- é representada por uma série de tempo, uma sequência de valores em pontos discretos equidistantes, sendo chamada também por função amostrada no tempo \cite{dimarogonas:1995}.
	Uma vez que os sinais obtidos nos sistemas atuais são representados discretamente, os métodos de análise também devem ser adaptados a esse modo de processamento.
	
	\subsubsection{Funções amostradas no tempo}
	O par de transformadas integrais das Equações \ref{eqn:fourier:transform} e \ref{eqn:fourier:transform-inv} vai do intervalo $ -\infty $ até $ \infty $, o que torna impossível a sua manipulação numérica.
	Um intervalo $ (-\tau /2,\:\tau /2) $ é empregado, assumindo que a função \gls{s:funcao-tempo} se repita fora desse intervalo em ambas as direções, e permitindo o uso da frequência fundamental \gls{s:freq:hz} que corresponde a \cite{dimarogonas:1995}
	\begin{equation} \label{eqn:fourier:freq-fundam}
		\mathit{f} = \frac{1}{\tau}
	\end{equation}
	
	Considerando que a amostragem da função \gls{s:funcao-tempo} seja feita em intervalos $ \Delta $\gls{s:tempo}, a frequência de amostragem \gls{s:freq-amostra} é definida como \cite{dimarogonas:1995}
	\begin{equation} \label{eqn:fourier:freq-amostra}
		\mathit{f}_s = \frac{1}{\Delta t}
	\end{equation}
	
	Considerando ainda que alguma perturbação aconteça a cada intervalo \[\Delta t = \frac{\tau}{N} \] onde \gls{s:numero-amostras} corresponde ao número de amostras coletadas para o período \gls{s:periodo}, é de se esperar que exista uma frequência inerente no espectro \gls{s:freq-amostra}, o que introduz no espectro de frequências uma periodicidade \gls{s:freq-amostra} \cite{dimarogonas:1995}.
	
	\subsubsection{Transformada Discreta de Fourier}
	Considerando que a função \gls{s:funcao-tempo} seja discretizada e truncada tanto no domínio como na frequência de modo que
	\begin{align*}
		t_n = n\Delta t\: ,&\quad n = 0, 1, \dots , N\\
		\mathit{f} = k\Delta\mathit{f} = \frac{k}{N\Delta t}\: ,&\quad k = 0, 1, \dots , N
	\end{align*}
	a transformada de Fourier (direta e inversa) torna-se \cite{randall:1987}
	\begin{align}
		G_k &= \frac{1}{N}\sum_{n=0}^{N-1}g_n e^{-i 2\pi k n/N} \label{eqn:fourier:dft}\\
		g_n &= \sum_{k=0}^{N-1}G_k e^{i 2\pi k n/N} \label{eqn:fourier:dft-inv}
	\end{align}
	que é chamada de \gls{sig:DFT} e, por substituir as integrais contínuas e infinitas das Equações \ref{eqn:fourier:transform} e \ref{eqn:fourier:transform-inv} por somas finitas, é muito mais adaptada à computação digital \cite{randall:1987}.
	Adicionalmente, a Equação \ref{eqn:fourier:dft} pode ser representada como \cite{dimarogonas:1995}
	\begin{equation} \label{eqn:fourier:dft-matricial}
		\vec{G} = \frac{1}{N}[\mathbf{A}]\vec{g}
	\end{equation}
	onde $ \vec{G} $ é um vetor contendo as \gls{s:numero-amostras} componentes complexas de frequência $ G_k $ , $ \vec{g} $ é um vetor contendo as \gls{s:numero-amostras} amostras coletadas no tempo, e \gls{s:matriz-unitarios-dft} é uma matriz quadrada $ \gls{s:numero-amostras}\! \times\!	\gls{s:numero-amostras} $ contendo os vetores unitários \gls{s:vetor-matriz-unit-dft} que dependem unicamente do número de amostras e são calculados por
	\begin{equation}
		\textbf{a}_{kn} = e^{-i2\pi kn/N}
	\end{equation}
	
	\subsubsection{Transformada Rápida de Fourier}
	Com larga aplicação nos instrumentos de análise atuais, a \gls{sig:FFT} é um algoritmo para obtenção da \gls{sig:DFT} que reduz consideravelmente o número de operações em relação ao método tradicional \cite{randall:1987}.
	Ao passo de que a resolução da Equação \ref{eqn:fourier:dft-inv} -- que obtém as componentes de frequência da \gls{sig:DFT} -- precisa de um total de $ \gls{s:numero-amostras}^2 $ computações aritméticas, o método da \gls{sig:FFT} fornece o resultado em menos de $ 2\gls{s:numero-amostras} \log_2\! \gls{s:numero-amostras} $ operações \cite{cooley:1965}.
	Para um caso em que \gls{s:numero-amostras} representa 1024 amostras, essa redução é da ordem de 100 vezes \cite{randall:1987}.	
	
	O algoritmo envolve o reordenamento e fatoração da matriz complexa \gls{s:matriz-unitarios-dft} em matrizes que produzem rotações progressivamente menores nos vetores unitários \cite{randall:1987}, e oferece grandes vantagens quando aplicado amostras cuja quantidade é uma potência inteira de dois, devido à natureza binária das operações computacionais \cite{cooley:1965}.
	Embora a abordagem completa do procedimento fuja do escopo deste texto, cabe ressaltar que o resultado obtido e as limitações envolvidas são os mesmos relativos ao cálculo da \gls{sig:DFT} \cite{randall:1987}.
	
	\section{Vibração de sistemas discretos}
	Denominam-se sistemas discretos aqueles que são descritos por um número finito de graus de liberdade.
	Embora grande parte das estruturas e máquinas possuam elementos elásticos e tenham, portanto, infinitos graus de liberdade, é comum discretizá-los através da divisão dos corpos rígidos em um número conhecido e distribuído de massas pontuais, para simplificação do problema \cite{rao:2008}.
	
	
	\subsection{Equação de movimento}
	O movimento linear de um sistema massa-mola-amortecedor qualquer é expresso por \cite{rao:2008}
	\begin{equation} \label{eqn:discr:movimento}
		[m]\ddot{\vec{x}} + [c]\dot{\vec{x}} + [k]\vec{x} = \vec{F}
	\end{equation}
	onde \gls{s:matriz-massa}, \gls{s:matriz-amort} e \gls{s:matriz-rigidez} representam as matrizes de massa, amortecimento e rigidez, respectivamente, e são dadas por
	\begin{gather} 
		[m] = 
		\begin{bmatrix} \label{eqn:matr:massa}
			m_{11} & m_{12} & m_{13} & \dots & m_{1n}\\
			m_{21} & m_{22} & m_{23} & \dots & m_{2n}\\
			\vdots\\
			m_{n1} & m_{n2} & m_{n3} & \dots & m_{nn}\\
		\end{bmatrix}\\
		[c] = 
		\begin{bmatrix} \label{eqn:matr:amort}
			c_{11} & c_{12} & c_{13} & \dots & c_{1n}\\
			c_{21} & c_{22} & c_{23} & \dots & c_{2n}\\
			\vdots\\
			c_{n1} & c_{n2} & c_{n3} & \dots & c_{nn}\\
		\end{bmatrix}\\
		[k] =
		\begin{bmatrix} \label{eqn:matr:rigid}
			k_{11} & k_{12} & k_{13} & \dots & k_{1n}\\
			k_{21} & k_{22} & k_{23} & \dots & k_{2n}\\
			\vdots\\
			k_{n1} & k_{n2} & k_{n3} & \dots & k_{nn}\\
		\end{bmatrix}
	\end{gather}
	onde \gls{s:coef-massa}, \gls{s:coef-amortecimento} e \gls{s:coef-rigidez} são os coeficientes de massa, amortecimento e rigidez da matriz correspondente.	Além destes, os termos \gls{s:vetor-desloc}, \gls{s:vetor-vel}, \gls{s:vetor-acel} e \gls{s:vetor-forcas} são os vetores de deslocamento, velocidade, aceleração e força, definidos por \cite{rao:2008}
	\begin{align} \label{eqn:vetores}
		\vec{x} =
		\begin{Bmatrix}
			x_1(t)\\ x_2(t)\\ \vdots\\ x_n(t)\\
		\end{Bmatrix},\qquad &
		\dot{\vec{x}} = 
		\begin{Bmatrix}
			\dot{x}_1(t)\\ \dot{x}_2(t)\\ \vdots\\ \dot{x}_n(t)\\
		\end{Bmatrix}, \notag\\
		\ddot{\vec{x}} = 
		\begin{Bmatrix}
			\ddot{x}_1(t)\\ \ddot{x}_2(t)\\ \vdots \\ \ddot{x}_n(t)\\  
		\end{Bmatrix},\qquad &
		\vec{F} = 
		\begin{Bmatrix}
			F_1(t)\\ F_2(t)\\ \vdots\\ F_n(t)\\
		\end{Bmatrix}
	\end{align}
	
	\subsection{Coordenadas e forças generalizadas} \label{sec:coord-generalizadas}
	O conjunto de coordenadas independentes necessárias para descrever o movimento de um sistema é denominado coordenadas generalizadas.
	Elas são representadas como $q_1$, $q_2$, \dots\ , $q_n$ e podem ser compostas por medidas lineares, ângulos ou qualquer arranjo de números que defina a configuração do sistema de maneira exclusiva em qualquer instante \cite{rao:2008}.
	
	Com a ação de forças externas, toma-se a mudança das coordenadas generalizadas \gls{s:coord-general} de um incremento $\delta q_i $, $ i = 1, 2, \dots, n $, onde $ n $ denota o número de graus de liberdade do sistema.
	Seja \gls{s:trabalho-coord-general} o trabalho realizado na variação da coordenada generalizada, a força generalizada correspondente \gls{s:forca-general} é definida como \cite{rao:2008}
	\begin{equation}
		Q_i = \frac{U_i}{\delta q_i},\qquad i = 1, 2, \dots, n
	\end{equation}
	
	\subsection{Equação de Lagrange}
	A resolução de um problema de vibração pela abordagem de Lagrange consiste em uma análise das energias potencial -- isto é, de deformação -- e cinética do sistema.
	O uso de grandezas escalares trazido por esta metodologia é conveniente em problemas complexos nos quais a utilização das dimensões vetoriais torna a resolução difícil \cite{savi:2017}.
	
	Considerando como \gls{s:desloc-elemento} o deslocamento de uma massa \gls{s:massa-elemento} e seja \gls{s:forca-elemento} a força nela aplicada, a energia de deformação \gls{s:energ-pot-elem} da i-ésima mola ou elemento elástico de um sistema com $ n $ graus de liberdade é dada por \cite{rao:2008}
	\begin{equation}
		\gls{s:energ-pot-elem} = \frac{1}{2}F_i x_i
	\end{equation}
	
	A energia potencial total \gls{s:energ-pot-total} é, então
	\begin{equation} \label{eqn:discr:energia:deformacao}
		\gls{s:energ-pot-total} = \sum_{i-1}^{n}\gls{s:energ-pot-elem} = \frac{1}{2}\sum_{i-1}^{n}F_i x_i
	\end{equation}
	
	Fazendo a substituição
	\begin{equation}
		F_i = \sum_{j=1}^{n}k_{ij} x_j
	\end{equation}
	onde \gls{s:coef-rigidez} representa o coeficiente de rigidez da i-ésima linha e j-ésima coluna na matriz de rigidez da Equação \ref{eqn:matr:rigid}, a Equação \ref{eqn:discr:energia:deformacao} fornece
	\begin{equation}
		\gls{s:energ-pot-total} = \frac{1}{2}\sum_{i=1}^{n}\left(\sum_{j=1}^{n}k_{ij} x_j\right)x_i = \frac{1}{2}\sum_{i=1}^{n}\sum_{j=1}^{n}k_{ij} x_i x_j
	\end{equation}
	que pode ser expressa na forma matricial como
	\begin{equation}
		\gls{s:energ-pot-total} = \frac{1}{2}\vec{x}^T [k] \vec{x}
	\end{equation}
	de modo que \gls{s:matriz-rigidez} e \gls{s:vetor-desloc} denotam respectivamente a matriz de rigidez da Equação \ref{eqn:matr:rigid} e o vetor de deslocamentos da Equação \ref{eqn:vetores}.
	
	Para a mesma massa \gls{s:massa-elemento}, a energia cinética associada é denotada por \gls{s:energ-cin-elem} e definida como \cite{rao:2008}
	\begin{equation}
		\gls{s:energ-cin-elem} = \frac{1}{2}m_i\dot{x}_i^2
	\end{equation}
	e, de maneira análoga à Equação \ref{eqn:discr:energia:deformacao}, a energia cinética total do sistema, \gls{s:energ-cin-total}, é obtida realizando o somatório para todos os $ n $ elementos, resultando em \cite{savi:2017}
	\begin{equation}\label{eqn:discr:energia:cinetica}
		\gls{s:energ-cin-total} = \frac{1}{2}\dot{\vec{x}^T}[m_d]\dot{\vec{x}}
	\end{equation}
	onde \gls{s:vetor-vel} corresponde ao vetor de velocidades da Equação \ref{eqn:vetores} e \gls{s:matriz-massa-diag} é uma matriz de massa diagonal
	\begin{equation}
		[m_d] =
		\begin{bmatrix}
			m_1 & & & 0\\
			& m_2\\
			& & \ddots\\
			0 & & & m_n\\
		\end{bmatrix}
	\end{equation}
	
	Se, na Equação \ref{eqn:discr:energia:cinetica} forem utilizadas as coordenadas generalizadas \gls{s:coord-general} descritas no item \ref{sec:coord-generalizadas}, arranjadas em um vetor de velocidades generalizadas \gls{s:vetor-veloc-general} dado por
	\begin{equation}
		\dot{\vec{q}} = 
		\begin{Bmatrix}
			\dot{q_1}\\ \dot{q_2}\\ \vdots\\ \dot{q_n}\\
		\end{Bmatrix}
	\end{equation}
	então essa mesma expressão pode ser enunciada como \cite{rao:2008}
	\begin{equation}
		\gls{s:energ-cin-total} = \frac{1}{2}\dot{\vec{q}^T}[m]\dot{\vec{q}}
	\end{equation}
	sendo agora \gls{s:matriz-massa} a matriz de massa da Equação \ref{eqn:matr:massa}.
	
	Uma vez determinadas as expressões em termos das energias potencial e cinética, é possível definir as equações de Lagrange como \cite{savi:2017}
	\begin{equation} \label{eqn:lagrange}
		\frac{d}{dt}\left(\frac{\partial L}{\partial \dot{q}_i}\right) - \frac{\partial L}{\partial q_i} = Q_i^{(n)}, \qquad i = 1, 2, \dots, n
	\end{equation}
	onde \gls{s:lagrangiano} é o lagrangiano
	\begin{equation}
		\gls{s:lagrangiano} = \gls{s:energ-cin-total}-\gls{s:energ-pot-total}
	\end{equation}
	e \gls{s:forca-naoconserv} é a força generalizada não-conservativa correspondente à coordenada generalizada \gls{s:coord-general}.
	Estas forças podem ser provenientes do amortecimento ou de quaisquer influências externas que não são deriváveis de uma função potencial \cite{rao:2008}.
	Se o sistema for conservativo, portanto, $ Q_j^{(n)} = 0 $.
	
	\section{Vibração de sistemas contínuos}
	Em contraste aos sistemas discretos, os sistemas contínuos são aqueles em que não se consegue identificar massas, molas e amortecedores pontuais na vibração de um corpo \cite{rao:2008}.
	Para estes casos, é possível determinar o comportamento vibratório de maneira analítica e sem discretização.
	Através dessa abordagem podem ser modeladas barras, eixos, cabos, vigas e outros elementos \cite{timoshenko:1974}.
	
	Ao tratar um corpo como um elemento elástico contínuo, ele é considerado como sendo composto de um número infinito de partículas infinitesimais, e é tratado como um sistema com infinitos gruas de liberdade.
	Dessa maneira, estruturas geometricamente complexas tornam-se demasiado difíceis -- ou mesmo impossíveis -- de serem modeladas por este método, restando para estes casos métodos que envolvam a discretização em um número finito de graus de liberdade \cite{timoshenko:1974}.
	
	\subsection{Equação de movimento} \label{desenv:cont:eqn-geral}
	A equação geral de movimento de um sistema contínuo pode ser deduzida a partir da vibração vertical de uma corda sob tensão.
	Esteja uma corda com uma massa por unidade de comprimento equivalente a \gls{s:massa-espec}, sujeita a uma força transversal \gls{s:forca} que é função do tempo \gls{s:tempo} e da coordenada $x$, como mostra o lado esquerdo da Figura \ref{fig:vibracao-corda}.
	O lado direito da figura mostra em detalhe o intervalo infinitesimal $dx$ entre os pontos $x_1$ e $x_2$.
	Nele também são esquematizados o deslocamento \gls{s:desloc-elastico} na direção $z$, a força de tração \gls{s:tensao-corda}, o ângulo de inclinação da corda em relação ao eixo $x$ denotado por $ \theta $ e, por fim, a força \gls{s:forca}$(x,t)$ atuante sobre o segmento infinitesimal de comprimento $ds$ da corda.
	\begin{figure}[t] 
	\incluirimagem{VibracaoCorda.png}{Vibração de uma corda tensionada}{adaptado de \citeauthoronline{rao:2008} (\citeyear{rao:2008})} \label{fig:vibracao-corda}
	\end{figure}

	Neste modelo, desconsidera-se a ação da atmosfera e da gravidade sobre a corda, além de supor-se que não existem perdas de energia na sua vibração.
	Também define-se a corda vibrante como tendo uma seção transversal pequena em comparação com seu comprimento, de maneira que as tensões ao longo da área da seção variem pouco e se tornem puramente axiais, desprezando assim tensões de flexão e cisalhamento \cite{clark:1972}.
	
	Com estas considerações, o balanço de forças na direção $z$ para o elemento da Figura \ref{fig:vibracao-corda} é o somatório da força transversal \gls{s:forca} e da componente vertical da força de tração \gls{s:tensao-corda} \cite{rao:2008}.
	A força líquida corresponde à força de inércia atuante sobre o elemento, portanto
	\begin{equation} \label{eqn:cont:balanco}
		(P + dP)\;\sin(\theta + d\theta) + F\:dx - P\;\sin\:\theta = \rho\:dx\:\frac{\partial^2 w}{\partial t^2}
	\end{equation}
	
	Para um comprimento elementar $dx$ a força de tração $dP$ pode ser escrita como \[dP = \frac{\partial P}{\partial x}\:dx\] e, considerando que a corda esteja firmemente esticada, estima-se que o ângulo $ \theta $ seja pequeno, permitindo fazer as simplificações \[\sin\:\theta \simeq tg\:\theta = \frac{\partial w}{\partial x} \] e \[\sin(\theta + d\theta) \simeq tg(\theta + d\theta) = \frac{\partial w}{\partial x} + \frac{\partial^2 w}{\partial x^2}\:dx \] que reduzem a Equação \ref{eqn:cont:balanco} para \[\frac{\partial}{\partial x}\,\left[P\:\frac{\partial w(x,t)}{\partial x}\right] + F(x,t) = \rho(x)\:\frac{\partial^2 w(x,t)}{\partial t^2} \]
	
	Considerando que a massa linear específica \gls{s:massa-espec} é uniforme, que a tensão \gls{s:tensao-corda} é constante e que a corda esteja em vibração livre -- isto é, \gls{s:forca}$(x,t) = 0$ -- chega-se a \cite{rao:2008}
	\begin{align}
		\mathit{c}^2\:\frac{\partial^2 w}{\partial x^2}=\frac{\partial^2 w}{\partial t^2} \label{eqn:cont:onda}\\
		\mathit{c} = \left(\frac{P}{\rho}\right)^{1/2}
	\end{align}
	
	A Equação \ref{eqn:cont:onda} também é conhecida como equação de onda. 
	
	\section{Dinâmica de rotores}
	Nos sistemas analisados nas seções anteriores, considerava-se o deslocamento em relação a uma posição de equilíbrio fixa, de modo que o movimento era analisado apenas em uma direção e a existência de vibração simultânea ao longo de outra coordenada não afetaria a análise, e as respectivas equações de movimento seriam desacopladas entre si.
	Um caso frequente de vibração que ocorre sobre um movimento contínuo -- portanto, com equações acopladas -- é visto na rotação de eixos \cite{dimarogonas:1995}.

	\subsection{Rodopio de eixos}
	Esteja um disco desbalanceado -- a rigor, com seu centro de massa localizando em um ponto diferente do centro de rotação -- montado sobre um eixo flexível.
	O desbalanceamento do disco durante a rotação provocará uma força oscilatória que tende a curvar o eixo, como mostra a Figura \ref{fig:whirl-eixo}.
	O efeito de rodopio -- também chamado de \foreignlanguage{english}{whirling}, em inglês -- é definido como a rotação do plano formado pelas linhas centrais dos mancais e o eixo curvado \cite{rao:2008}.
	\begin{figure}[b]
		\incluirimagem[0.15]{WhirlEixo.png}{Eixo em rodopio}{\citeauthoronline{dimarogonas:1995} (\citeyear{dimarogonas:1995})}
		\label{fig:whirl-eixo}
	\end{figure}

	\subsubsection{Equações de movimento}
	Para a modelagem do sistema, considera-se um eixo cuja posição de equilíbrio quando perfeitamente balanceado é denotada por $\!\mathit{O}$, como mostra a Figura \ref{fig:rotor-excentrico}.
	Enquanto girando, devido à excitação permanente provocada pelo desbalanceamento do rotor, o eixo está sujeito a uma deflexão radial $\!\mathit{A}$ em relação à posição de equilíbrio.
	O centro de massa do rotor, na posição $\!\mathit{G}$, está a uma distância \gls{s:excentricidade} do centro geométrico $\!\mathit{C}$.
	A frequência angular do giro do rotor é representada por \gls{s:freq-rad}.
	O ponto $\!\mathit{O}$ é assumido como sendo a origem de um sistema de coordenadas inercial, no qual os vetores unitários nas direções x e y são representados por \gls{s:vetor-unitario-x} e \gls{s:vetor-unitario-y}, respectivamente.
	\begin{figure}[h]
		\incluirimagem{DiagramaRodopioEixo.png}{Diagrama de rotor excêntrico}{adaptado de \citeauthoronline{rao:2008} (\citeyear{rao:2008})}
		\label{fig:rotor-excentrico}
	\end{figure}
	
	As equações de movimento para o rotor podem ser descritas, portanto, pela soma de forças \cite{rao:2008}
	\begin{equation} \label{eqn:rotores:forcas}
		\vec{F}_i = \vec{F}_e + \vec{F}_{di} + \vec{F}_{de}
	\end{equation}
	que representam, na ordem, as forças de inércia, elástica, de amortecimento interno, e de amortecimento externo.
	Inicialmente, a força de inércia é calculada por \cite{rao:2008}
	\begin{equation} \label{eqn:rotores:forca-inercia}
		\vec{F}_i = m\ddot{\vec{R}}
	\end{equation}
	onde o vetor $ \vec{R} $ indica a posição do centro de massa $ \mathit{G} $ em relação à origem $ \mathit{O} $, sendo então definido como \cite{rao:2008}
	\begin{equation} \label{eqn:rotores:vetor-centro-massa}
		\vec{R} = (x + \mathit{e}\cos\omega t)\gls{s:vetor-unitario-x} + (y + \mathit{e}\sin\omega t)\gls{s:vetor-unitario-y}
	\end{equation}
	de maneira que $ x $ e $ y $ são as coordenadas do centro geométrico $ \mathit{C} $.
	Calculando a segunda derivada da Equação \ref{eqn:rotores:vetor-centro-massa} e substituindo na Equação \ref{eqn:rotores:forca-inercia} fornece
	\begin{equation} \label{eqn:rotres:forca-inercia}
		\vec{F}_i = m\left[ (\ddot{x} - \mathit{e}\omega^2\,\cos\omega t)\gls{s:vetor-unitario-x} +
		(\ddot{y} - \mathit{e}\omega^2\,\sin\omega t)\gls{s:vetor-unitario-y} \right]
	\end{equation}
	
	A força elástica relativa à rigidez do eixo \gls{s:rigidez} resulta em \cite{rao:2008}
	\begin{equation}\label{eqn:rotores:forca-elastica}
		\vec{F}_e = -k(x\gls{s:vetor-unitario-x} + y\gls{s:vetor-unitario-y})
	\end{equation}
	As forças de amortecimento interno e externo são definidas, respectivamente, por \cite{rao:2008}
	\begin{gather}
		\vec{F}_{di} = -c_i\left[(\dot{x} + \omega y)\gls{s:vetor-unitario-x} +
		(\dot{y} + \omega x)\gls{s:vetor-unitario-y} \right] \label{eqn:rotores:forca-amort-interno} \\
		\vec{F}_{de} = -c(\dot{x}\gls{s:vetor-unitario-x} + \dot{y}\gls{s:vetor-unitario-y}) \label{eqn:rotores:forca-amort-externo}
	\end{gather}
	sendo que também são chamadas amortecimento histerético e viscoso.
	Para o amortecimento viscoso, os efeitos estão geralmente relacionados à supressão e eliminação de vibrações \cite{dimarogonas:1995} e sua origem está, como a outra terminologia expressa, em mecanismos externos que atenuam a vibração.
	
	O amortecimento histerético, por outro lado, é relacionado às características do material.
	A imposição de uma deformação cíclica em materiais sólidos -- especialmente metais -- provoca uma diferença de fase entre força e deformação.
	Traçar estes valores durante um ciclo em um diagrama cartesiano mostrará um gráfico como o da Figura \ref{fig:histerese}, que é chamado de ciclo de histerese.
	Visto que a área do gráfico é um produto de força e deslocamento, tem-se que ela corresponde à energia dissipada pelo amortecimento interno do material \cite{dimarogonas:1995}.
	
	\begin{figure}[b]
		\incluirimagem{Histerese.png}{Ciclo de histerese para força-deformação}{adaptado de \citeauthoronline{dimarogonas:1995} (\citeyear{dimarogonas:1995})}
		\label{fig:histerese}
	\end{figure}

	A energia dissipada por conta do amortecimento histerético é praticamente independente da frequência, e é proporcional ao quadrado da amplitude de vibração $ X $ e à rigidez \gls{s:rigidez} \cite{dimarogonas:1995}.
	Representando essa energia como $ D_h $, obtém-se
	\begin{equation}
		D_h = \pi X^2k\gamma
	\end{equation}
	onde a constante $ \gamma $ é uma propriedade do material.
	Através disso, pode-se definir o coeficiente de amortecimento interno $ c_i $ da Equação \ref{eqn:rotores:forca-amort-interno} como \cite{dimarogonas:1995}
	\begin{equation}
		c_i = \frac{\gamma}{\gls{s:freq-rad} - \gls{s:freq-natural}}\,\gls{s:rigidez}
	\end{equation}
	onde \gls{s:freq-rad} é a velocidade angular do eixo e \gls{s:freq-natural} é a frequência natural.
	Cabe observar que, entre as quatro equações de força apresentadas, aquela referente ao amortecimento histerético é a responsável pelo acoplamento entre as direções $ x $ e $ y $, visto que é a única que contém medidas de ambas as coordenadas multiplicando cada um dos vetores unitários \gls{s:vetor-unitario-x} e \gls{s:vetor-unitario-y}.
	
	Finalmente, substituindo as Equações \ref{eqn:rotores:forca-inercia} a \ref{eqn:rotores:forca-amort-externo} na Equação \ref{eqn:rotores:forcas} obtêm-se as equações de movimento \cite{rao:2008}
	\begin{align}
		m\ddot{x} + (c_i + c)\dot{x} + kx - c_i\omega y &= m\omega^2\mathit{e}\cos\omega t \\
		m\ddot{y} + (c_i + c)\dot{y} + ky - c_i\omega x &= m\omega^2\mathit{e}\sin\omega t
	\end{align}
	ou, definindo para o deslocamento \gls{s:desloc-elastico} um número complexo de modo que \[ w = x + iy \] chega-se à equação combinada
	\begin{equation}\label{eqn:rotores:resposta-rodopio}
		m\ddot{w} + (c_i + c)\dot{w} + kw - i\omega c_i w = m\omega^2 \mathit{e}\ e^{i\omega t}
	\end{equation}
	
	\subsubsection{Velocidades críticas}
	Um eixo em rotação atinge uma velocidade crítica quando esta é igual a alguma das frequências naturais do componente.
	A frequência natural não amortecida do conjunto do rotor pode ser obtida \cite{rao:2008} resolvendo a Equação \ref{eqn:rotores:resposta-rodopio} com $ c = c_i = 0 $.
	Isso fornece o valor da frequência natural \gls{s:freq-natural} do sistema
	\begin{equation} \label{eqn:freq-natural}
		\omega_n = \left( \frac{k}{m} \right)^{1/2}
	\end{equation}
	
	Uma vez definida a velocidade crítica, denomina-se \cite{dimarogonas:1995} operação subcrítica quando a velocidade de rotação \gls{s:freq-rad} é menor que a frequência natural, ou seja, $ \omega < \omega_n $, e operação supercrítica quando $ \omega > \omega_n $.
	
	Considerando que um rotor de massa \gls{s:massa} tenha uma pequena massa desbalanceada $ m_e $ a uma distância $ u $ do centro, a órbita do rodopio pode ser definida como um círculo de raio $ r $, definido como \cite{dimarogonas:1995}
	\begin{equation} \label{eqn:rotores:orbita-rodopio}
		r = \frac{m_e u}{m}\ \frac{(\omega/\omega_n)^2}{1 - (\omega/\omega_n)^2}
	\end{equation}
	
	Cabe observar que a razão $ u(m_e/m) $ é equivalente à excentricidade \gls{s:excentricidade} do centro de massa exibida na Figura \ref{fig:rotor-excentrico} e nas equações seguintes, de modo que
	\begin{equation}
		\mathit{e} = \frac{m_e u}{m}
	\end{equation}
	
	Pode-se perceber na Equação \ref{eqn:rotores:orbita-rodopio} que, quando a velocidade de rotação \gls{s:freq-rad} se iguala à frequência natural \gls{s:freq-natural} o raio $ r $ tende ao infinito.
	Isso comprova a definição de uma velocidade crítica como equivalente à frequência natural.
	Embora isso sugira que a rotação na velocidade crítica promova amplitudes de vibração muito altas, isso não é necessariamente verdade \cite{dimarogonas:1995}.
	Considerando a solução na direção $ x $ para o rodopio do eixo como \cite{dimarogonas:1995}
	\begin{equation}
		x(t) = \frac{m\mathit{e}\omega^2}{k-m\omega^2}\,(cos\,\omega t - cos\,\omega_n t)
	\end{equation}
	e rearranjando a Equação \ref{eqn:freq-natural} para definir \gls{s:rigidez} como \[ k = m\omega_n^2 \] pode-se perceber que, quando $ \omega = \omega_n $ a resposta de vibração é uma indefinição da forma $ 0/0 $.
	Aplicando a regra de L'Hôpital em relação a \gls{s:freq-rad} fornece, para $ \omega\to\omega_n $
	\begin{equation}
		x(t) = \frac{\mathit{e}\omega}{2}\ t\sin\omega t
	\end{equation}
	
	Com base nesse resultado, verifica-se que a amplitude da vibração em ressonância cresce, de fato, de maneira ilimitada.
	No entanto, esse aumento de amplitude não é instantâneo \cite{dimarogonas:1995}.
	A Figura \ref{fig:aceleracao-vel-critica} ilustra o deslocamento máximo de vibração $ x_{max} $ para diferentes valores de aceleração angular $ \alpha = \dot{\omega} $.
	O gráfico exibido confirma a constatação em \cite{rao:2008}, de que uma passagem rápida do eixo pela velocidade crítica limitará a amplitude de vibração.
	\begin{figure}
		\incluirimagem{AceleracaoVelCritica.png}{Passagem de um eixo pela velocidade crítica}{adaptado de \citeauthoronline{dimarogonas:1995} (\citeyear{dimarogonas:1995})}
		\label{fig:aceleracao-vel-critica}
	\end{figure}

	\section{Vibração em rolamentos}
	Ao analisar a vibração em rolamentos, é de especial interesse identificar o sinal emitido quando apresentam falhas, para permitir diagnósticos no equipamento com antecedência.
	Os defeitos em rolamentos de esferas ou de rolos podem ser classificados como distribuídos ou localizados.
	Entre os defeitos distribuídos, pode-se citar a rugosidade superficial e ondulação nas pistas do rolamento, elementos rolantes fora das dimensões apropriadas, entre outras imperfeições inerentes da fabricação do componente.
	Já os defeitos localizados geralmente são resultados do processo de fadiga, e se apresentam como trincas, lascados ou pequenos buracos nas pistas do rolamento \cite{tandon:1997,sassi:2007}.
	Tendo como interesse no presente trabalho a investigação de falhas relativas ao uso e desgaste do equipamento, apenas os defeitos localizados serão considerados.
	
	\subsection{Métodos de análise}
	Diferentes métodos de análise para a vibração provocada por defeitos localizados em rolamentos foram desenvolvidos desde a metade do século passado.
	\citeauthor{mcfadden:1984} propuseram um modelo bastante simples para modelar um defeito pontual em uma pista interna de um rolamento, que considera a carga radial aplicada e a condução do sinal até o acelerômetro que mediria a vibração, através do mancal.
	Abordagens mais complexas foram propostas nos anos seguintes, como uma análise proposta por \citeauthor{tandon:1997} que modela os anéis do rolamento como sistemas contínuos, além de incluir cargas radiais e axiais no desenvolvimento.
	
	\citeauthor{sassi:2007} apresentaram um modelo bastante elaborado, que também trata os anéis do rolamento como corpos flexíveis e define a região de análise como um sistema de três graus de liberdade.
	A abordagem considera também as cargas nas direções radial e axial, além de incluir os efeitos de lubrificação entre as pistas e esferas e ruído aleatório no sinal, causado pelo escorregamento dos elementos rolantes.	\citeauthor{patil:2010} modelaram o rolamento como um sistema de dois graus de liberdade que, em contraste aos outros autores, considera os anéis como componentes rígidos e analisa as esferas como deformáveis.
	
	Nas seções seguintes, uma síntese dos métodos utilizados nas diferentes abordagens é apresentada, considerando tanto suas similaridades como as diferenças entre cada metodologia de modelagem.
	
	\subsection{Frequências características}
	Para a modelagem e análise da vibração de um rolamento com defeito pontual, um conjunto de frequências características é definido.
	Considerando, de maneira mais genérica, que ambos os anéis estejam girando e sejam $ \gls{s:freq-rad}_i $ e $ \gls{s:freq-rad}_e $ as velocidades angulares dos anéis interno e externo do rolamento respectivamente, definem-se as frequências \cite{sassi:2007}
	\begin{align}
		\gls{s:freq-rol-gaiola} &= \frac{1}{2}\cdot\left[ \omega_i\left(1-\frac{\gls{s:rol-diam-esferas}\,\cos\alpha}{\gls{s:rol-diam-espec}} \right) + \omega_o\left(1+\frac{\gls{s:rol-diam-esferas}\cos\alpha}{\gls{s:rol-diam-espec}} \right) \right] \label{eqn:rol:ftf}\\
		\gls{s:freq-rol-def-out} &= \frac{N_b}{2}\cdot(\omega_i-\omega_o)\cdot\left(1-\frac{\gls{s:rol-diam-esferas}\cos\alpha}{\gls{s:rol-diam-espec}} \right) \label{eqn:rol:bpfo}\\
		\gls{s:freq-rol-def-in} &= \frac{N_b}{2}\cdot(\omega_i-\omega_o)\cdot\left(1+\frac{\gls{s:rol-diam-esferas}\cos\alpha}{\gls{s:rol-diam-espec}} \right) \label{eqn:rol:bpfi}\\
		\gls{s:freq-rol-esfera} &= \frac{\gls{s:rol-diam-espec}}{2\,\gls{s:rol-diam-esferas}}\cdot(\omega_i-\omega_o)\cdot\left(1-\frac{\gls{s:rol-diam-esferas}^2\cos^2\alpha}{\gls{s:rol-diam-espec}^2} \right) \label{eqn:rol:bsf}
	\end{align}
	que correspondem, na ordem de apresentação, à frequência fundamental da gaiola, frequência de passagem de esferas por um defeito na pista externa, frequência de passagem de esferas por um defeito na pista interna e, por fim, a frequência de rotação das esferas.
	Ainda nas Equações \ref{eqn:rol:ftf} a \ref{eqn:rol:bsf}, \gls{s:rol-num-esferas} representa o número de esferas ou rolos do rolamento, \gls{s:rol-angulo-contato} é o ângulo de contato das esferas, \gls{s:rol-diam-esferas} indica o diâmetro das esferas, e \gls{s:rol-diam-espec} é o diâmetro específico (também denotado em inglês como \foreignlanguage{english}{pitch diameter}), calculado como o diâmetro médio entre o interno e externo.
	
	\subsection{Força de impacto}
	Nos modelos numéricos investigados, um fator importante na determinação da reposta de vibração é a força de impacto, que pode ser calculada de maneiras diferentes.
	Em geral, essa força é influenciada pelo carregamento estático e pela amplificação durante a passagem dos elementos rolantes pelo defeito, que pode ser tratada como um impulso \cite{mcfadden:1984,sassi:2007,cong:2013} ou como um pulso de duração finita \cite{tandon:1997}.
	
	\subsubsection{Carregamento externo}
	Esteja o rolamento esteja sujeito a uma força $ \vec{\gls{s:forca}} $ definida como \[ \vec{F}=F_a\,\gls{s:vetor-unitario-x} + F_r\,\gls{s:vetor-unitario-y} \] onde $ F_a $ e $ F_r $ são os módulos das componentes axial e radial da força, respectivamente.
	Isso provoca um deslocamento relativo $ \vec{\mathit{d}} $ entre o anel interno e externo que pode ser expresso como \[ \vec{\mathit{d}} = \mathit{d}_a\,\gls{s:vetor-unitario-x}+\mathit{d}_r\,\gls{s:vetor-unitario-y} \] sendo que $ \mathit{d}_a $ e $ \mathit{d}_r $ correspondem, analogamente, às partes axial e radial do deslocamento \cite{sassi:2007}.
	
	A condição de equilíbrio estático para essa força pode ser expressa como \[ \vec{F} + \sum_{i=1}^{\gls{s:rol-num-esferas}}\vec{\gls{s:carregamento}}_i=0 \] sendo $ \vec{\gls{s:carregamento}}_i $ a força do carregamento em cada elemento rolante, como mostra a Figura \ref{fig:carregamento-rolamento}.
	Para uma esfera ou rolo a um ângulo \gls{s:rol-angulo-carregamento} máximo $ \gls{s:carregamento}_{max} $, o valor do carregamento $ \gls{s:carregamento}_i $ neste elemento é dado por \cite{mcfadden:1984,sassi:2007,tandon:1997,cong:2013}
	\begin{equation} \label{eqn:rol:carregamento}
		\gls{s:carregamento}_i =
		\begin{cases}
			\gls{s:carregamento}_{max}\,[1-(1/2\gls{s:rol-fator-dist-carga})(1-\cos\gls{s:rol-angulo-carregamento})]^n, & -\Psi_m\leq\gls{s:rol-angulo-carregamento}\leq\Psi_m \\
			0, & \mbox{outras posições}
		\end{cases}
	\end{equation}
	onde o intervalo $ [-\gls{s:rol-angulo-carregamento},\gls{s:rol-angulo-carregamento}] $ é a extensão da zona de carga -- representada em verde na Figura \ref{fig:carregamento-rolamento} -- e o expoente $ n $ vale $ 3/2 $ para rolamentos de esferas e $ 10/9 $ para rolamentos de rolos \cite{tandon:1997}.
	O termo \gls{s:rol-fator-dist-carga} é o fator de distribuição de carga, que pode ser obtido por \cite{sassi:2007}
	\begin{equation}
		\gls{s:rol-fator-dist-carga} = \frac{1}{2}\cdot\left[1+\frac{\mathit{d}_a}{\mathit{d}_r} \tan\gls{s:rol-angulo-contato} \right]
	\end{equation}
	
	\begin{figure}[t]
		\incluirimagem{CarregamentoRolamento.png}{Distribuição do carregamento externo em um rolamento de esferas}{adaptado de \citeauthoronline{sassi:2007} (\citeyear{sassi:2007})}
		\label{fig:carregamento-rolamento}
	\end{figure}

	\subsubsection{Componente dinâmica}
	As referências \cite{mcfadden:1984,tandon:1997} definem a amplitude da força de impacto como sendo dependente somente do carregamento estático \gls{s:carregamento}.
	No modelo proposto na referência \cite{sassi:2007}, por outro lado, a força total de impacto é a soma de \gls{s:carregamento} com uma componente dinâmica $ \gls{s:forca}_D $, descrita por
	\begin{equation}
		\gls{s:forca}_D = \gls{s:carregamento}\cdot\gls{s:rol-coef-impacto}\cdot \left(\frac{\gls{s:rol-comp-defeito}}{\gls{s:rol-diam-esferas}}\right)^2
	\end{equation}
	onde \gls{s:rol-comp-defeito} é o comprimento do defeito e \gls{s:rol-coef-impacto} é um coeficiente de impacto que depende tanto do material como da geometria do rolamento.
	
	Incluindo a componente dinâmica, a força total de impacto $ \gls{s:forca}_T $ resulta em
	\begin{equation} \label{eqn:rol:forca-impacto-tot}
		F_T = \gls{s:carregamento}\left[1 + \gls{s:rol-coef-impacto}\cdot \left(\frac{\gls{s:rol-comp-defeito}}{\gls{s:rol-diam-esferas}}\right)^2 \right]
	\end{equation}
	de modo que o carregamento \gls{s:carregamento} é definido na Equação \ref{eqn:rol:carregamento}.
	
	\subsubsection{Impactos modulados pela força}
	A incidência de um elemento rolante sobre o defeito em uma das frequências calculadas nas Equações \ref{eqn:rol:ftf} a \ref{eqn:rol:bsf} resultará em um impacto, cuja amplitude será dependente da força total de impacto naquele ângulo de rotação \cite{mcfadden:1984}.
	Esta aplicação de força em um período curto de tempo é tratada em alguns dos modelos como um impulso unitário \cite{mcfadden:1984,sassi:2007,cong:2013}, que é também chamado de função delta de Dirac e obedece à seguinte definição \cite{boyce:2017}
	\begin{gather}
		\gls{s:func-impulso}(t - t_0) = 0,\qquad t\neq0\\
		\int_{-\infty}^{\infty}\gls{s:func-impulso}(t-t_0)\,dt = 1
	\end{gather}
	o que resulta em uma função que possui valor não-nulo em um intervalo infinitamente curto, mas preserva o tamanho total de uma unidade.
	
	Considerando o giro do eixo com o rolamento em regime permanente, a função $ p(t) $ que define a influência dos impactos pode ser modelada com uma série infinita de impulsos de igual amplitude $ p_0 $, separados por um período $ \gls{s:periodo}_d $, que é o recíproco da frequência de incidência de um elemento rolante sobre o defeito.
	Um exemplo da função $ p(t) $ está representado na Figura \ref{fig:funcao-impulso} e ela pode ser definida como \cite{mcfadden:1984}
	\begin{equation} \label{eqn:rol:impulsos}
		p(t) = p_0\sum_{k=-\infty}^{\infty}\gls{s:func-impulso}(t-k\,\gls{s:periodo}_d)
	\end{equation}
	\begin{figure}[t]
		\incluirimagem{FuncaoImpulso.png}{Função com impulsos periódicos}{adaptado de \citeauthoronline{cong:2013} (\citeyear{cong:2013})}
		\label{fig:funcao-impulso}
	\end{figure}

	A influência do comportamento em impulsos, modulado pela força de impacto total, é obtida multiplicando as Equações \ref{eqn:rol:forca-impacto-tot} e \ref{eqn:rol:impulsos}.
	
	\subsection{Modelagem do rolamento}
	Entre diversas possibilidades para modelagem do conjunto, os modelos numéricos avaliados buscam uma alternativa de simplificação da situação real, visto que a determinação exata do comportamento dinâmico é complicada e, por vezes, impossível \cite{sassi:2007}.
	Abaixo são apresentados os elementos tratados para a análise do sistema.
	
	\subsubsection{Modelo dos anéis}
	Quando uma oscilação no anel do rolamento é excitada devido ao impacto de um elemento rolante com um defeito localizado, este anel -- que é modelado como um sistema contínuo (flexível) -- deve entrar em uma vibração de flexão, gerada em diversos modos \cite{sassi:2007}.
	A forma que o anel assume no $ n $-ésimo modo é definida pela função \cite{tandon:1997}
	\begin{equation}
		X_n(\phi) = \sin n\phi + \cos n\phi
	\end{equation}
	
	A frequência natural do $n$-ésimo modo de vibração também pode ser determinada, através da fórmula \cite{tandon:1997,sassi:2007,cong:2013}
	\begin{equation} \label{eqn:rol:freq-natural-modo}
		\gls{s:freq-natural} = \frac{n(n^2-1)}{\sqrt{1+n^2}}\sqrt{\frac{\gls{s:modulo-elast} \gls{s:segundo-momento-area}}{\gls{s:massa-espec} \gls{s:raio-eixo-neutro}^4}}
	\end{equation}
	de modo que \gls{s:modulo-elast} é o módulo de elasticidade do material, \gls{s:segundo-momento-area} é o segundo momento de área de seção transversal, e \gls{s:raio-eixo-neutro} é o raio do eixo neutro do anel.
	
	Por questão de simplificação, apenas o primeiro modo de flexão será considerado.
	Isso corresponde ao modo $ n=2 $, uma vez que os modos $ n=0 $ e $ n=1 $ são modos rígidos.
	Neste modo, o anel do rolamento toma a forma de uma elipse \cite{sassi:2007}.
	Conhecendo-se a massa dos anéis e determinando os valores das frequências naturais do modo $ n=2 $ conforme a Equação \ref{eqn:rol:freq-natural-modo}, a Equação \ref{eqn:freq-natural} pode ser rearranjada para determinar as rigidezes $ \gls{s:rigidez}_I $ e $ \gls{s:rigidez}_O $ dos anéis interno e externo, respectivamente \cite{sassi:2007}:
	\begin{align}
		\gls{s:rigidez}_I &= m_I(\gls{s:freq-natural}_{,I})^2\\
		\gls{s:rigidez}_O &= m_O(\gls{s:freq-natural}_{,O})^2
	\end{align}
	
	\subsubsection{Modelo dos elementos rolantes}
	Nas referências \cite{mcfadden:1984,tandon:1997,sassi:2007,cong:2013} as esferas são tratadas como corpos rígidos.
	Isso justifica-se pelo fato de serem os elementos de maior rigidez no sistema \cite{sassi:2007}.
	Uma abordagem alternativa permite, no entanto, considerar a deformação das esferas através da teoria de deformação por contato de Hertz, que relaciona uma força radial $ \gls{s:forca}_r $ com a deformação \gls{s:rol-deform-esfera} como \cite{patil:2010}
	\begin{equation} \label{eqn:rol:deform-esfera}
		\gls{s:forca}_r = \gls{s:rol-fator-deform}\gls{s:rol-deform-esfera}^n
	\end{equation}
	onde o expoente $ n $ tem os mesmos valores definidos para a Equação \ref{eqn:rol:carregamento} e \gls{s:rol-fator-deform} é o fator carga-deformação, definido como \cite{patil:2010}
	\begin{equation} \label{eqn:rol:fator-deform}
		\gls{s:rol-fator-deform} = \left[ \frac{1}{(1/K_I)^{1/n}+(1/K_O)^{1/n}} \right]^n
	\end{equation}
	
	Os parâmetros $ K_I $ e $ K_O $ na Equação \ref{eqn:rol:fator-deform} são, por sua vez, as rigidezes de contato entre a pista e a esfera ou rolo, e podem ser calculados por \cite{patil:2010}
	\begin{equation}
		K_p = 2,\!15\cdot 10^5 \left(\sum\varrho\right)^{-1/2}(\zeta^*)^{-3/2}
	\end{equation}
	onde $ \sum\varrho $ e $ \zeta^* $ são dependentes da geometria do rolamento.
	
	\subsection{Transmissão da oscilação}
	Além do comportamento pulsante que molda a resposta de vibração de um rolamento a partir da força de impacto, outros dois fatores ainda colaboram para modificar o sinal que é efetivamente lido durante a medição da oscilação no componente.
	A condução da vibração através do mancal, dependendo da posição de instalação do transdutor, e o decaimento do impulso lido afetam a característica do sinal recebido pelos instrumentos \cite{mcfadden:1984}.
	
	\subsubsection{Posição relativa do medidor de vibração}
	Quando um impulso é provocado no rolamento, frequências de ressonância serão excitadas nele e na máquina em que ele está instalado, permitindo a medição do sinal em um acelerômetro fixado próximo ao componente.
	Considerando a detecção do transdutor em uma direção, espera-se que a amplitude máxima da vibração medida ocorrerá quando a oscilação ocorrer em paralelamente ao eixo de medição do acelerômetro.
	Da mesma forma, estima-se que a resposta medida pelo dispositivo será próxima de zero quando o impulso ocorrer em uma direção perpendicular ao eixo o acelerômetro \cite{mcfadden:1984}.
	
	Dessa maneira, pode-se determinar a amplitude de vibração \gls{s:rol-amplitude-sensor} medida pelo transdutor em função da amplitude máxima $ \gls{s:rol-amplitude-sensor}_{max} $ e do ângulo $ \theta $ entre esse valor limite e o eixo de medição do acelerômetro, sendo definida por \cite{sassi:2007}
	\begin{equation}
		\gls{s:rol-amplitude-sensor}(\theta) = \gls{s:raio} - \left( \frac{\sin^2\theta}{[\gls{s:raio}-\gls{s:rol-amplitude-sensor}_{max}]^2} + \frac{\cos^2\theta}{[\gls{s:raio}+\gls{s:rol-amplitude-sensor}_{max}]^2} \right)^{-1/2}
	\end{equation}
	
	Nessa equação, \gls{s:raio} denota o raio do anel não-deformado.
	Uma simplificação pode ser feita no caso do anel avaliado ser fixo, o que torna o ângulo $ \theta $ constante.
	
	\subsubsection{Decaimento do sinal de impulso}
	Ao gerar os impulsos, a resposta de vibração do rolamento excitará frequências de ressonância no sistema \cite{cong:2013}.
	A ressonância provocada por um impulso apresenta-se como uma onda cossenoidal, cuja amplitude sofre um decaimento exponencial como mostra a Figura \ref{fig:decaimento-impulso}, que é expresso por \cite{mcfadden:1984,cong:2013}
	\begin{equation}
		\gls{s:func-decaimento}(t) =
		\begin{cases}
			\cos(\gls{s:freq-natural}\gls{s:tempo})\,e^{-B\gls{s:tempo}}, & \gls{s:tempo}>0\\
			0, & \gls{s:tempo}\leq0
		\end{cases}
	\end{equation}
	onde $ B $ é um parâmetro que define a taxa de decaimento da amplitude, e a frequência natural \gls{s:freq-natural} corresponde à frequência natural do modo obtida na Equação \ref{eqn:rol:freq-natural-modo}.
	
	\begin{figure}[b]
		\incluirimagem{DecaimentoImpulso.png}{Decaimento exponencial de um sinal de impulso}{adaptado de \citeauthoronline{cong:2013} (\citeyear{cong:2013})}
		\label{fig:decaimento-impulso}
	\end{figure}

	Denotando como $ g(t) $ a reposta de vibração do sistema com o produto das funções definidas até o momento, isto é:
	\begin{enumerate}
		\item a força total de impacto, $ \gls{s:forca}_T $;
		\item a função de impulsos provocada pela passagem nos defeitos, \gls{s:func-impulso}; e
		\item a amplitude máxima de acordo com a posição do transdutor, \gls{s:rol-amplitude-sensor};
	\end{enumerate}
	pode-se obter o sinal efetivo recebido pelo transdutor ao realizar a convolução entre este produto e a função \gls{s:func-decaimento} \cite{mcfadden:1984,cong:2013}
	\begin{equation}
		\gls{s:funcao-tempo} = [\gls{s:forca}_T(t)\cdot \gls{s:func-impulso}(t)\cdot \gls{s:rol-amplitude-sensor}(t)]\ast\gls{s:func-decaimento}(t)
	\end{equation}
	
	\chapter{Proposta}
	O capítulo a seguir apresenta a metodologia adotada para atingir os objetivos propostos, além da disposição das tarefas ao longo do desenvolvimento do trabalho.
	
	\section{Modelagem numérica}
	Conforme especificado no item \ref{sec:objetivos:especificos}, o trabalho a ser desenvolvido consiste no desenvolvimento de modelos numéricos para duas situações de vibração em máquinas rotativas:
	\begin{enumerate}
		\item resposta de vibração causada pelo desbalanceamento de um rotor montado sobre o eixo;
		\item resposta de vibração provocada pelo defeito controlado no rolamento de um mancal.
	\end{enumerate}

	Ambos os modelos numéricos serão desenvolvidos no MATLAB\textsuperscript{\textregistered}.
	O nível de detalhamento dos modelos iniciará com sistemas de dois graus de liberdade, e poderá ser aprofundado de acordo com os resultados obtidos experimentalmente.
	O sucesso dos modelos será identificado como uma consistência qualitativa entre o resultado da simulação e os valores observados em bancada, não estabelecendo uma dependência da exatidão quantitativa destes dados.
	
	\section{Verificação experimental}
	Para validar os resultados obtidos em simulação, será utilizado o protótipo da bancada didática mencionada no item \ref{sec:justificativa}.
	A Figura \ref{fig:componentes} mostra o protótipo desmontado, que é composto pelos seguintes componentes:
	\begin{enumerate}
		\item uma mesa de $ 1000\ mm $ de largura e $ 720\ mm $ de comprimento (item A), montada a partir de perfis de alumínio de $ 45 \times 90\ mm $, apoiada sobre coxins de elastômero, isolando-a da estrutura rígida em que a bancada é montada;
		\item motor de indução trifásico de dois polos (item E), controlado por um inversor de frequência, para variação da sua velocidade de rotação; 
		\item base para o motor (item E) com parafusos para ajuste fino do alinhamento com o eixo;
		\item relógio comparador Starrett com resolução de $ 0,\!01\ mm $ (item D), para verificação do alinhamento entre o eixo do motor e o eixo movido;
		\item eixo de $ 20\ mm $ de diâmetro e 400 mm de comprimento (item G), usinado em aço ABNT 1045, onde pode ser montado um disco (item F) com furos radiais nos quais podem ser fixados parafusos M5 para deslocar o centro de massa e provocar desbalanceamento;
		\item dois mancais  (item B) em aço ABNT 1045 nos quais são inseridos rolamentos rígidos de esferas tipo 6004 2RSH, do fabricante SKF.
		O rolamento tem encaixe deslizante e pode ser removido do mancal entre os testes.
		O mancal também tem dois furos -- um na direção vertical e outro na direção horizontal -- para fixar acelerômetros;
		\item um rolamento extra, da mesma marca e modelo do daquele utilizado nos mancais (item B), porém com um defeito controlado na pista externa;
		\item par de polias trapezoidais (item H) de perfil padrão A1 com diâmetros de $ 90\ mm $ e $ 60\ mm $, além de correia trapezoidal lisa e esticador ajustável;
		\item um mancal de rolamento duplo (item C), com eixo passante de $ 20\ mm $ de diâmetro e comprimento de $ 200\ mm $, utilizado apenas para apoiar a polia trapezoidal;
		\item conjunto de módulo de aquisição de dados e acelerômetros para coleta de dados de vibração durante os experimentos.
	\end{enumerate}

	\begin{figure}[h]
		\incluirimagem[0.7]{ComponentesBancada.png}{Componentes principais da bancada}{o autor (\thedate)}
		\label{fig:componentes}
	\end{figure}

	De modo a possibilitar a verificação dos experimentos, os modelos numéricos a serem desenvolvidos utilizarão como dados de entrada as medidas e massas dos próprios componentes da bancada.

	\section{Alterações no protótipo}
	Eventuais alterações construtivas no protótipo podem ser consideradas, como a fabricação de componentes extras ou o retrabalho de usinagem em peças já existentes.
	Tais alterações, se necessárias, serão motivadas pelos próprios resultados do modelo numérico.
	No caso destes indicarem valores de vibração inadequados para a instrumentação a ser utilizada, modificações nas medidas dos componentes do sistema serão realizadas.

	\section{Sequência de trabalho}
	A Figura \ref{fig:fluxograma} ilustra o fluxo de trabalho a ser desenvolvido.
	O passo inicial consiste em obter as informações necessárias para construir no MATLAB\textsuperscript{\textregistered} os modelos propostos.
	Visto que serão desenvolvidos dois experimentos, uma sequência de procedimentos será executada duas vezes.
	Ela consiste em desenvolver a simulação e verificar se os resultados trarão valores mensuráveis durante a aquisição de dados na bancada real.
	Um resultado negativo nessa verificação pode indicar, por exemplo, um superdimensionamento do sistema e demandará um retrabalho nos componentes do protótipo ou até mesmo a fabricação de peças novas com diferentes dimensões.
	Caso o modelo numérico retorne valores aceitáveis para uma medição posterior com os instrumentos, os resultados são coletados.
	
	Quando finalizadas as simulações numéricas dos experimentos, o desenvolvimento segue com a montagem da bancada e a criação de uma aplicação no LabVIEW\textsuperscript{\textregistered} para fazer a aquisição de dados durante os testes do sistema real.
	Os experimentos serão conduzidos, e os dados coletados serão tratados e filtrados.
	Por fim, os resultados dos modelos numéricos serão comparados com os dados experimentais para validação dos modelos.

	\begin{figure}[h]
		\incluirimagem[0.4]{Fluxograma.png}{Fluxograma de desenvolvimento do trabalho}{o autor (\thedate)}
		\label{fig:fluxograma}
	\end{figure}
	\clearpage
	
	\section{Planejamento}
	Considera-se como início do desenvolvimento das tarefas descritas no fluxograma da Figura \ref{fig:fluxograma} a semana 28 de 2018, iniciada em 8 de julho.
	A partir desta data, as etapas a serem conduzidas são dispostas no cronograma do Quadro \ref{quad:cronograma}.
	Apesar de se tratar de um trabalho individual, considera-se um paralelismo na execução das tarefas de retrabalho em componentes.
	Isso foi planejado porque, além deste ser um passo que não necessariamente será executado, a execução das alterações nos componentes ou a fabricação de peças novas será realizada por uma empresa terceirizada, havendo apenas a necessidade de enviar os projetos atualizados e aguardar o prazo do serviço.
	
	\begin{quadro}[h]
		\incluirimagem[1]{Cronograma.png}{Cronograma de tarefas}{o autor (\thedate)}
		\label{quad:cronograma}
	\end{quadro}
	
	\postextual
	
	\bibliography{Bibliografia}
		
\end{document}