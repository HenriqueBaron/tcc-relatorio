% Capa e folha de rosto
\imprimircapa
\imprimirfolhaderosto
\clearpage

% Folha de aprovação
%\imprimirfolhadeaprovacao[Auttom Automação e Robótica Ltda.]{10/07/2018}{Me. Joel Vicente Ciapparini}{Me. Alexandre Viecceli}{Gilvan Antonio Menegotto}

% Dedicatória
%\begin{dedicatoria}
%	\vspace*{\fill}
%	
%	\hspace{.45\textwidth}
%	\begin{minipage}[b]{.5\textwidth}
%		\SingleSpacing
%		Dedico este trabalho a\dots
%	\end{minipage}
%\end{dedicatoria}

% Resumo em português
\begin{resumo}
	\SingleSpacing
	Atualmente, o uso da análise de vibrações para diagnóstico de falhas em equipamentos vem ganhando espaço, por conta do aumento da oferta de instrumentos capazes de coletar e analisar tais informações, e também pelo potencial desta metodologia de monitoramento em identificar defeitos nos primeiros estágios de desenvolvimento.
	Em vista disso, a formação de profissionais capacitados para analisar e interpretar estes dados é de suma importância para a indústria.
	Com este foco, o presente trabalho tem por objetivo validar o protótipo de uma bancada didática para análise de vibrações em máquinas, realizando a modelagem numérica, a montagem do conjunto, e a verificação experimental dos resultados da simulação.
	Os modelos e tetes serão focados em dois experimentos com aplicabilidade prática: a resposta de vibração pelo desbalanceamento de rotores, e por danos em rolamentos de esferas, com o intuito de identificar características de cada defeito que possam ser verificadas ao utilizar a bancada.
	\vspace{\onelineskip}
	
	\noindent
	\textbf{Palavras-chave}: Vibração. Modelagem. Rolamentos. Desbalanceamento.
\end{resumo}

% Resumo em inglês
%\begin{resumo}[\normalsize\bfseries ABSTRACT]
%	\SingleSpacing
%	\begin{otherlanguage}{english}
%		Abstract, in English.
%		\vspace{\onelineskip}
%		
%		\noindent
%		\textbf{Keywords}: Key.
	Words.
%	\end{otherlanguage}
%\end{resumo}

% Listas de figuras, quadros, tabelas e siglas
\pdfbookmark[0]{\listfigurename}{lof}
\listoffigures*
\cleardoublepage

\pdfbookmark[0]{\listquadroname}{loq}
\listofquadros*
\cleardoublepage

%\pdfbookmark[0]{\listtablename}{lot}
%\listoftables*
%\cleardoublepage

\pdfbookmark[0]{\listadesiglasname}{las}
\listofsiglas*
\cleardoublepage

% Lista de símbolos
\pdfbookmark[0]{\listadesimbolosname}{lsb}
\listofsimbolos*
\cleardoublepage

% Sumário
\pdfbookmark[0]{\contentsname}{toc}
\tableofcontents*
\cleardoublepage
