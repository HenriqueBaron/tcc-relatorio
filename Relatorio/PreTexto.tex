% Capa e folha de rosto
\imprimircapa
\imprimirfolhaderosto
\clearpage

% Folha de aprovação
%\imprimirfolhadeaprovacao[Auttom Automação e Robótica Ltda.]{10/07/2018}{Me. Joel Vicente Ciapparini}{Me. Alexandre Viecceli}{Gilvan Antonio Menegotto}

% Dedicatória
%\begin{dedicatoria}
%	\vspace*{\fill}
%	
%	\hspace{.45\textwidth}
%	\begin{minipage}[b]{.5\textwidth}
%		\SingleSpacing
%		Dedico este trabalho a\dots
%	\end{minipage}
%\end{dedicatoria}

% Resumo em português
\begin{resumo}
	\SingleSpacing
	A análise de vibrações é atualmente um dos métodos mais utilizados para manutenção preditiva e monitoramento de condição de máquinas na indústria. 
	Para uma compreensão aprofundada da origem dos fenômenos de vibrações nestes equipamentos, a modelagem numérica é uma abordagem usual, pois permite a identificação de comportamentos característicos para cada caso de falha, eliminando interferências oriundas dos instrumentos.
	Com este foco, o presente trabalho tem por objetivo estabelecer modelos matemáticos para a vibração em duas situações práticas da indústria: um sistema rotativo com um rotor desbalanceado, e um mancal de rolamento danificado.
	O resultado obtido com a modelagem servirá como recurso para o desenvolvimento de uma bancada didática para análise de vibrações em máquinas rotativas.
	\vspace{\onelineskip}
	
	\noindent
	\textbf{Palavras-chave}: Vibração. Modelagem. Rolamentos. Desbalanceamento.
\end{resumo}

% Resumo em inglês
%\begin{resumo}[\normalsize\bfseries ABSTRACT]
%	\SingleSpacing
%	\begin{otherlanguage}{english}
%		Abstract, in English.
%		\vspace{\onelineskip}
%		
%		\noindent
%		\textbf{Keywords}: Key. Words.
%	\end{otherlanguage}
%\end{resumo}

% Listas de figuras, quadros, tabelas e siglas
\listoffigures*
\cleardoublepage

\listofquadros*
\cleardoublepage

%\listoftables*
%\cleardoublepage

\listofsiglas*
\cleardoublepage

% Lista de símbolos
\listofsimbolos*
\cleardoublepage

% Sumário
\tableofcontents*
\cleardoublepage
