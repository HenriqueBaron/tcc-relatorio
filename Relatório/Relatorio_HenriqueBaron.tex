\documentclass[12pt,openright,twoside,a4paper,
	chapter=TITLE,section=TITLE,
	english,brazil]{abntex2}
% Referências a bibliografia no padrão ABNT. Opções: citações numeradas, com sobrescrito e substituindo listas com mais de dois autores por "et al." e escrevendo o "et al." em itálico.
\usepackage[num, overcite, abnt-etal-list=2, abnt-etal-text=it]{abntex2cite}
\usepackage[brazil]{babel}
\usepackage[utf8]{inputenc}
\usepackage[T1]{fontenc}
\usepackage{graphicx}
\usepackage{float} % Utilizado para dispor figuras com a posição "h" (Here)
\usepackage{pdfpages} % Permite a inclusão de páginas em PDF dentro do documento. Será útil para incluir uma cópia digitalizada da folha de aprovação, após as assinaturas.
\usepackage{lmodern} % Altera a fonte do documento para Latin Modern, uma evolução à fonte Computer Modern, padrão do LaTeX.
\usepackage{textcomp}
\usepackage{abntUCS}
\citebrackets[] % Configura para que as citações sejam feitas entre colchetes.
\includeonly{PreTexto}

\titulo{Desenvolvimento de uma Bancada Didática para Análise de Vibrações em Máquinas}
\autor{Henrique Baron}
\data{2018}
\instituicao{Universidade de Caxias do Sul}
\local{Caxias do Sul}
\preambulo{Trabalho apresentado como requisito parcial para obtenção do grau no curso de Bacharelado em Engenharia Mecânica da Universidade de Caxias do Sul.}
\orientador{Me. Paulo Roberto Linzmaier}

% Configuração do pacote hyperref, que é chamado pela classe abntex2
\makeatletter
\hypersetup{
	pdftitle={\@title},
	pdfauthor={\@author},
	pdfsubject={\imprimirpreambulo},
	pdfkeywords={Vibrações. Análise. Modelagem. Ensino.},
	pdfcreator={LaTeX with abnTeX2},
	colorlinks=true,
	linkcolor=black,
	citecolor=black,
}
\makeatother

\begin{document}
	% Capa e folha de rosto
\imprimircapa
\imprimirfolhaderosto
\clearpage

% Folha de aprovação
%\imprimirfolhadeaprovacaoUCS[Auttom Automação e Robótica Ltda.]{10/07/2018}{Me. Joel Vicente Ciapparini}{Me. Alexandre Viecceli}{Gilvan Antonio Menegotto}

% Dedicatória
%\begin{dedicatoria}
%	\vspace*{\fill}
%	
%	\hspace{.45\textwidth}
%	\begin{minipage}[b]{.5\textwidth}
%		\SingleSpacing
%		Dedico este trabalho a\dots
%	\end{minipage}
%\end{dedicatoria}

% Resumo em português
%\begin{resumo}
%	\SingleSpacing
%	Resumo do trabalho.
%	\vspace{\onelineskip}
%	
%	\noindent
%	\textbf{Palavras-chave}: Palavras. Chave
%\end{resumo}

% Resumo em inglês
%\begin{resumo}[\normalsize\bfseries ABSTRACT]
%	\SingleSpacing
%	\begin{otherlanguage}{english}
%		Abstract, in English.
%		\vspace{\onelineskip}
%		
%		\noindent
%		\textbf{Keywords}: Key. Words.
%	\end{otherlanguage}
%\end{resumo}

% Listas de figuras, quadros, tabelas e siglas
%\pdfbookmark[0]{\listfigurename}{lof}
%\listoffigures*
%\cleardoublepage
%
%\pdfbookmark[0]{\listtablename}{lot}
%\listoftables*
%\cleardoublepage

\begin{siglas}
	\item[VBM] \textit{Vibration-Based Monitoring}
\end{siglas}

% Sumário
\pdfbookmark[0]{\contentsname}{toc}
\tableofcontents*
\cleardoublepage

	\textual % Comando que define que os elementos textuais começaram, para incluir a numeração de páginas.
	
	\chapter{Introdução}
	O capítulo a seguir situa o leitor em relação aos fatores técnicos e econômicos que motivaram a elaboração deste trabalho. Também são apresentados o ambiente onde foi desenvolvido, seguido pelos objetivos a serem atingidos com a sua realização.
	
	\section{Justificativa}
	Em todo o tipo de sistema produtivo em que se faz o uso de ferramentas ou máquinas, a manutenção é um tema que recebe grande atenção, dada a necessidade de manter o funcionamento de equipamentos com o menor tempo possível de parada, buscando-se encontrar um balanço entre o custo de repetidas paradas para inspeção e o de uma eventual falha ocasionada por falta de acompanhamento. O montante gasto em manutenção pelas empresas no Brasil no ano de 2013 foi equivalente a 4,69\% do PIB nacional, correspondendo a um valor de R\$ 206,5 bilhões\cite{bib:seleme}.
	
	Nos últimos anos, o aumento da oferta de produtos na área de instrumentação e coleta de dados possibilitou à indústria o crescimento do uso de técnicas de manutenção preditiva para análise de condições dos equipamentos. Entre as técnicas existentes, o monitoramento baseado em vibrações confere a vantagem de indicar mudanças nas condições de uma máquina em um estágio inicial de uma falha ou desgaste\cite{bib:al-najjar}, permitindo otimizar a frequência e assertividade das intervenções de manutenção, provocando redução de custos.
	
	Como exemplo, dados do Laboratório Nacional de Energia Renovável dos Estados Unidos (\textit{National Renewable Energy Laboratory} ou NREL) apontam que 76\% dos casos de falha em caixas de engrenagens de aerogeradores são causados por rolamentos, sendo os defeitos nas próprias engrenagens a segunda maior causa em número de ocorrências (17,1\%). Ao mesmo tempo, apenas 10\% dos rolamentos atingem a sua vida útil esperada devido à diversidade de condições de desgaste prematuro às quais estes componentes podem ser submetidos\cite{bib:peeters}. Isso demonstra que, neste caso, um plano de manutenção baseado no monitoramento de condições do sistema é a melhor maneira de minimizar o tempo de parada por falha. Levando em conta ainda o componente em questão, a análise de vibrações se demonstra como o melhor método de investigação de falhas prematuras.
	
	Dada a relevância do tema, é necessária a formação de profissionais competentes no campo da análise de vibrações em máquinas, em ambos os níveis técnico e acadêmico. E, para tanto, surge a demanda por um equipamento didático que aproxime o aluno da técnica aplicada e estabeleça uma conexão com a fundamentação teórica tratada em sala de aula. Ao mesmo tempo, tal produto deve não somente expor uma situação prática, mas também comprovar -- dentro de uma dada tolerância -- que os valores calculados no ambiente teórico se reproduzem no mundo real.
	
	Considerando a escassez da oferta de um equipamento para estudo de vibrações no mercado brasileiro, enxergou-se a possibilidade de desenvolvimento de uma bancada didática que trouxesse os conceitos de análise de vibrações em máquinas para escolas técnicas e universidades. No entanto, para o desenvolvimento de um produto para o mercado de ensino, é vital que exista uma fidelidade entre o equipamento construído e a teoria que ele demonstra. Portanto, a fabricação de um protótipo e a validação deste com um modelo teórico fundamentado são etapas importantes na concepção do produto.
	
	\section{Ambiente de Desenvolvimento}	
	O trabalho foi conduzido no departamento de Pesquisa e Desenvolvimento da unidade de educação da empresa Auttom Automação e Robótica Ltda., situada em Caxias do Sul. A empresa possui duas divisões: uma concentrada em projetos de automação sob demanda; e a outra, onde o trabalho foi realizado, desenvolve bancadas didáticas e sistemas de ensino para escolas técnicas e universidades.
	
	Atualmente, a unidade de educação da empresa conta com uma linha ampla de produtos nas áreas de automação industrial, energias renováveis, eletricidade e refrigeração. Ainda não há uma oferta considerável de produtos para estudo de mecânica -- isto é, desde componentes mecânicos, materiais, ou mesmo a análise de sistemas mecânicos, como é o caso da análise de vibrações -- e o desenvolvimento deste trabalho surge como uma possibilidade de inclusão da empresa em um mercado ainda pouco explorado.
	
	\section{Objetivos}
	Em linhas gerais, o objetivo do presente trabalho é validar o protótipo de uma bancada didática para análise de vibrações em máquinas rotativas. Para tanto, pretende-se desenvolver um modelo numérico no MATLAB\textsuperscript\textregistered\ que estabeleça como deve ser o comportamento do sistema quando em funcionamento. Posteriormente, objetiva-se construir o protótipo e comparar o seu comportamento com o resultado do modelo numérico construído. Isso deverá ser obtido através da instalação de instrumentos no protótipo e criação de uma aplicação no LabVIEW\textsuperscript\textregistered\ para fazer a coleta e análise dos dados.
	
	Finalmente, o intuito é comparar os resultados modelados e medidos para alguns dos experimentos que se pretende executar no produto final. Tais experimentos consistem em avaliar a resposta de vibração do sistema em relação à alteração em algumas variáveis. São elas:
	\begin{itemize}
		\setlength{\itemsep}{0pt plus 2pt minus 1pt}
		\item Desalinhamento entre o motor e o eixo acoplado;
		\item Desbalanceamento de um componente montado ao eixo;
		\item Desgaste no rolamento do mancal.
	\end{itemize}

	Embora o conceito do produto inclua outros experimentos além dos citados, estes vão além do escopo deste trabalho. Com os resultados obtidos nas três situações mencionadas acima, busca-se formar uma base de experiência para o desenvolvimento posterior do protótipo.
	
	\postextual
	
	\bibliography{Bibliografia}
		
\end{document}