\documentclass[12pt,openright,twoside,a4paper,
	chapter=TITLE,section=TITLE,
	english,brazil]{abntex2}
\usepackage[brazil]{babel}
\usepackage[utf8]{inputenc}
\usepackage[T1]{fontenc}
\usepackage{graphicx}
\usepackage{float} % Utilizado para dispor figuras com a posição "h" (Here)
\usepackage{pdfpages} % Permite a inclusão de páginas em PDF dentro do documento. Será útil para incluir uma cópia digitalizada da folha de aprovação, após as assinaturas.
\usepackage{lmodern} % Altera a fonte do documento para Latin Modern, uma evolução à fonte Computer Modern, padrão do LaTeX.
\usepackage{textcomp}
\usepackage{abntUCS}
\includeonly{PreTexto}

\titulo{Desenvolvimento de uma Bancada Didática para Análise de Vibrações em Máquinas}
\autor{Henrique Baron}
\data{2018}
\instituicao{Universidade de Caxias do Sul}
\local{Caxias do Sul}
\preambulo{Trabalho desenvolvido para obtenção do grau no curso de Bacharelado em Engenharia Mecânica.}
\orientador{Me. Paulo Roberto Linzmaier}

% Configuração do pacote hyperref, que é chamado pela classe abntex2
\makeatletter
\hypersetup{
	pdftitle={\@title},
	pdfauthor={\@author},
	pdfsubject={\imprimirpreambulo},
	pdfkeywords={PALAVRAS}{CHAVE},
	pdfcreator={LaTeX with abnTeX2},
	colorlinks=true,
	linkcolor=black,
	citecolor=black,
}
\makeatother

\begin{document}
	% Capa e folha de rosto
\imprimircapa
\imprimirfolhaderosto
\clearpage

% Folha de aprovação
%\imprimirfolhadeaprovacaoUCS[Auttom Automação e Robótica Ltda.]{10/07/2018}{Me. Joel Vicente Ciapparini}{Me. Alexandre Viecceli}{Gilvan Antonio Menegotto}

% Dedicatória
%\begin{dedicatoria}
%	\vspace*{\fill}
%	
%	\hspace{.45\textwidth}
%	\begin{minipage}[b]{.5\textwidth}
%		\SingleSpacing
%		Dedico este trabalho a\dots
%	\end{minipage}
%\end{dedicatoria}

% Resumo em português
%\begin{resumo}
%	\SingleSpacing
%	Resumo do trabalho.
%	\vspace{\onelineskip}
%	
%	\noindent
%	\textbf{Palavras-chave}: Palavras. Chave
%\end{resumo}

% Resumo em inglês
%\begin{resumo}[\normalsize\bfseries ABSTRACT]
%	\SingleSpacing
%	\begin{otherlanguage}{english}
%		Abstract, in English.
%		\vspace{\onelineskip}
%		
%		\noindent
%		\textbf{Keywords}: Key. Words.
%	\end{otherlanguage}
%\end{resumo}

% Listas de figuras, quadros, tabelas e siglas
%\pdfbookmark[0]{\listfigurename}{lof}
%\listoffigures*
%\cleardoublepage
%
%\pdfbookmark[0]{\listtablename}{lot}
%\listoftables*
%\cleardoublepage

\begin{siglas}
	\item[VBM] \textit{Vibration-Based Monitoring}
\end{siglas}

% Sumário
\pdfbookmark[0]{\contentsname}{toc}
\tableofcontents*
\cleardoublepage

	\textual % Comando que define que os elementos textuais começaram, para incluir a numeração de páginas.
	

	\chapter{Introdução}
	O capítulo a seguir situa o leitor em relação aos fatores técnicos e econômicos que motivaram a elaboração deste trabalho. Também são apresentados o ambiente onde foi realizado, seguido por seus objetivos. A disposição do conteúdo deste documento também é incluída nesta parte.
	
	\section{Justificativa}
	
	Nos últimos anos, o aumento da oferta de produtos na área de instrumentação e coleta de dados possibilitou à indústria o crescimento do uso de técnicas de manutenção preditiva para análise de condições dos equipamentos. Entre as técnicas existentes, o monitoramento baseado em vibrações confere a vantagem de indicar mudanças nas condições de uma máquina em um estágio inicial de uma falha ou desgaste\cite{bib:al-najjar}.
	
	Dada a importância do tema no meio industrial, é necessária a formação de profissionais competentes no assunto, em ambos os níveis técnico e acadêmico. E, para tanto, surge a demanda por um equipamento didático que aproxime o aluno da técnica aplicada e estabeleça uma conexão com a fundamentação teórica tratada em sala de aula. Ao mesmo tempo, tal produto deve não somente expor uma situação prática, mas também comprovar -- dentro de uma dada tolerância -- que os valores calculados no ambiente teórico se reproduzem no mundo real.
	
	Considerando a escassez da oferta de um equipamento para estudo de vibrações no mercado brasileiro, enxergou-se a possibilidade de desenvolvimento de uma bancada didática que trouxesse os conceitos de análise de vibrações em máquinas para escolas técnicas e universidades. No entanto, para o desenvolvimento de um produto para o mercado de ensino, é vital que exista uma fidelidade entre o equipamento construído e a teoria que ele demonstra. Portanto, a fabricação de um protótipo e a validação deste com um modelo teórico fundamentado são etapas importantes na concepção do produto.
	
	\section{Ambiente de Desenvolvimento}
	
	O trabalho foi conduzido no departamento de Pesquisa e Desenvolvimento da unidade de educação da empresa Auttom Automação e Robótica Ltda., situada em Caxias do Sul. A empresa possui duas divisões: uma concentrada em projetos de automação sob demanda; e a outra, onde o trabalho foi realizado, desenvolve bancadas didáticas e sistemas de ensino para escolas técnicas e universidades.
	
	Atualmente, a unidade de educação da empresa conta com uma linha ampla de produtos nas áreas de automação industrial, energias renováveis, eletricidade e refrigeração. Ainda não há uma oferta considerável de produtos para estudo em mecânica -- isto é, desde componentes mecânicos, materiais, ou mesmo a análise de sistemas mecânicos, como é o caso da análise de vibrações -- e o desenvolvimento deste trabalho surge como uma possibilidade de inclusão da empresa em um mercado ainda pouco atendido.
	
	\section{Objetivos}
	
	Em linhas gerais, o objetivo do presente trabalho é validar o protótipo de uma bancada didática para análise de vibrações em máquinas rotativas. Portanto, a primeira etapa consiste no desenvolvimento de um modelo numérico no MATLAB\textsuperscript\textregistered\ que estabeleça como deve ser o comportamento do sistema quando em funcionamento. Uma vez aprovado o modelo, o protótipo deve ser construído e testado. Para isso, o conjunto mecânico será instrumentado e uma aplicação será construída no LabVIEW\textsuperscript\textregistered\ para fazer a coleta e análise dos dados. Por fim, objetiva-se comparar os valores retornados pelo modelo matemático com os dados de medição do sistema real para validação do protótipo.
	
	As três fases que compõem os objetivos do trabalho -- modelo numérico, coleta e análise de dados, e comparação de resultados -- pretendem validar especificamente alguns dos experimentos que poderão ser executados no produto final. Estes experimentos avaliam a resposta de vibração do sistema em relação à mudança em alguma variável. São elas:
	\begin{itemize}
		\setlength{\itemsep}{2pt plus 2pt minus 1pt}
		\item Desalinhamento entre o motor e o eixo acoplado;
		\item Desbalanceamento de um componente montado ao eixo;
		\item Dano no rolamento do mancal.
	\end{itemize}

	Embora o conceito final do produto tenha uma gama maior de experimentos em vista, a experimentação destes foge ao escopo deste trabalho.
	
	\postextual
	
	\bibliography{Bibliografia}
		
\end{document}