\documentclass[12pt,openright,twoside,a4paper,
	chapter=TITLE,section=TITLE,
	english,brazil]{abntex2}
\usepackage[brazil]{babel}
\usepackage[utf8]{inputenc}
\usepackage[T1]{fontenc}
\usepackage{graphicx}
\usepackage{float} % Utilizado para dispor figuras com a posição "h" (Here)
\usepackage[alf]{abntex2cite} % Referências a bibliografia no padrão ABNT, como (<autor>, <ano>)
\usepackage{pdfpages} % Permite a inclusão de páginas em PDF dentro do documento. Será útil para incluir uma cópia digitalizada da folha de aprovação, após as assinaturas.
\usepackage{lmodern} % Altera a fonte do documento para Latin Modern, uma evolução à fonte Computer Modern, padrão do LaTeX.
\usepackage{abntUCS}
\includeonly{PreTexto}

\titulo{Desenvolvimento de uma Bancada Didática para Análise de Vibrações em Máquinas}
\autor{Henrique Baron}
\data{2018}
\instituicao{Universidade de Caxias do Sul}
\local{Caxias do Sul}
\preambulo{Trabalho desenvolvido para obtenção do grau no curso de Bacharelado em Engenharia Mecânica.}
\orientador{Me. Paulo Roberto Linzmaier}

% Configuração do pacote hyperref, que é chamado pela classe abntex2
\makeatletter
\hypersetup{
	pdftitle={\@title},
	pdfauthor={\@author},
	pdfsubject={\imprimirpreambulo},
	pdfkeywords={PALAVRAS}{CHAVE},
	pdfcreator={LaTeX with abnTeX2},
	colorlinks=true,
	linkcolor=black,
	citecolor=black,
}
\makeatother

\begin{document}
	% Capa e folha de rosto
\imprimircapa
\imprimirfolhaderosto
\clearpage

% Folha de aprovação
%\imprimirfolhadeaprovacaoUCS[Auttom Automação e Robótica Ltda.]{10/07/2018}{Me. Joel Vicente Ciapparini}{Me. Alexandre Viecceli}{Gilvan Antonio Menegotto}

% Dedicatória
%\begin{dedicatoria}
%	\vspace*{\fill}
%	
%	\hspace{.45\textwidth}
%	\begin{minipage}[b]{.5\textwidth}
%		\SingleSpacing
%		Dedico este trabalho a\dots
%	\end{minipage}
%\end{dedicatoria}

% Resumo em português
%\begin{resumo}
%	\SingleSpacing
%	Resumo do trabalho.
%	\vspace{\onelineskip}
%	
%	\noindent
%	\textbf{Palavras-chave}: Palavras. Chave
%\end{resumo}

% Resumo em inglês
%\begin{resumo}[\normalsize\bfseries ABSTRACT]
%	\SingleSpacing
%	\begin{otherlanguage}{english}
%		Abstract, in English.
%		\vspace{\onelineskip}
%		
%		\noindent
%		\textbf{Keywords}: Key. Words.
%	\end{otherlanguage}
%\end{resumo}

% Listas de figuras, quadros, tabelas e siglas
%\pdfbookmark[0]{\listfigurename}{lof}
%\listoffigures*
%\cleardoublepage
%
%\pdfbookmark[0]{\listtablename}{lot}
%\listoftables*
%\cleardoublepage

\begin{siglas}
	\item[VBM] \textit{Vibration-Based Monitoring}
\end{siglas}

% Sumário
\pdfbookmark[0]{\contentsname}{toc}
\tableofcontents*
\cleardoublepage

	\textual % Comando que define que os elementos textuais começaram, para incluir a numeração de páginas.
	

	\chapter{Introdução}
	O capítulo a seguir situa o leitor em relação aos fatores técnicos e econômicos que motivaram a elaboração deste trabalho. Também são apresentados o ambiente onde foi realizado, seguido por seus objetivos. A disposição do conteúdo deste documento também é incluída nesta parte.
	
	\section{Justificativa}
	Nos últimos anos, o aumento da oferta de produtos na área de instrumentação e coleta de dados possibilitou à indústria o crescimento do uso de técnicas de manutenção preditiva para análise de condições de equipamentos. 
	
	
	\postextual
	
	\bibliography{Bibliografia}
		
\end{document}