%!TeX spellcheck = en_US
The application of rolling element bearings ranges from simple mechanical systems to high complexity and precision machinery.
Since the vibration analysis is the main way to early identify failure in these units, it is of great importance to determine the vibrational pattern of this component in a defect situation.
With that in focus, this work is concentrated in the development of a non-linear, three degrees-of-freedom numeric model for the vibration response of a radial ball bearing with a point defect on the outer ring.
For this, the concepts of the \emph{hertzian} contact deformation theory are employed, in order to determine the force on a rolling element when it strikes the defect.
This force value is used to model three pulse profiles for the load caused by the impact between the ball and the defect.
At the same time, the influence of the lubricant fluid film on the contact between rolling elements and rings is evaluated under the fundamentals of the \emph{elastohydrodynamic} lubrication theory.
The developed model is investigated on its stability and convergence, observing the computational cost of the different pulse shapes for the impact load.
The simulated vibration response is compared to simulation data and experimental tests from previous works in the same application field, analyzing it in both time and frequency domain, in order to conclude on the qualitative consistency of the model.
\vspace{\onelineskip}

\noindent
\textbf{Keywords}: Vibration. Model. Defect. Bearing.