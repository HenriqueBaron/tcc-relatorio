% Capa e folha de rosto
\imprimircapa
\imprimirfolhaderosto
\clearpage

% Folha de aprovação
\imprimirfolhadeaprovacao{29/11/2018}

% Agradecimentos
\begin{agradecimentos}
	Gostaria de agradecer primeiramente à minha família pelo enorme suporte e incentivo ao desenvolvimento deste trabalho, em especial nos momentos de maior dificuldade, no qual fizeram tudo o que estava ao seu alcance para me ajudar a superar os problemas que enfrentava.
	A eles e à minha namorada, Fernanda, agradeço a compreensão e paciência durante este período extenuante de estudo e entrega.
	
	Às professoras Dra. Kátia Cavalca e Dra. Isolda Gianni de Lima, agradeço pela prontidão e disposição em me ajudar nos momentos em que eu não pude encontrar uma resposta para minhas dúvidas.
	
	Agradeço também à empresa Auttom Automação e Robótica, por propiciar a oportunidade de desenvolvimento pessoal com o tema deste trabalho, e pelo suporte que sempre deu à condução dos meus estudos.
\end{agradecimentos}

% Resumo em português
\begin{resumo}
	\SingleSpacing
	O emprego de mancais de rolamento se aplica desde sistemas mecânicos simples até o maquinário de alta complexidade e precisão.
	Sendo a análise de vibrações o meio utilizado para identificação precoce de falhas nestas unidades, é de grande importância determinar o comportamento deste componente em situação de defeito.
	Com este foco, este trabalho concentra-se no desenvolvimento de um modelo numérico não-linear de três graus de liberdade para a resposta de vibração de um rolamento radial de esferas com um defeito pontual no anel externo.
	Para isso, são empregados os conceitos da teoria \emph{hertziana} de deformação por contato, de modo a determinar a força sobre um elemento rolante quando este incide sobre o defeito.
	O valor de força é utilizado para modelar três perfis de pulso diferentes para a carga de impacto entre a esfera do rolamento e o defeito.
	Ao mesmo tempo, a influência do filme de fluido lubrificante no contato entre esferas e anéis é analisado sob os fundamentos da teoria de lubrificação \emph{elastohidrodinâmica}.
	O modelo desenvolvido é investigado quanto à sua estabilidade e convergência, observando o custo computacional das diferentes formas de pulso para a carga de impacto.
	A resposta de vibração simulada é comparada a dados de simulações e verificações experimentais de trabalhos anteriores no mesmo campo de aplicação, analisando-a tanto no domínio do tempo como da frequência, para concluir quanto à consistência qualitativa do modelo.
	\vspace{\onelineskip}
	
	\noindent
	\textbf{Palavras-chave}: Vibração. Modelagem. Defeito. Rolamento.
\end{resumo}

% Resumo em inglês
\begin{resumo}[\normalsize\bfseries ABSTRACT]
	\SingleSpacing
	\begin{otherlanguage}{english}
		%!TeX spellcheck = en_US
The application of rolling element bearings ranges from simple mechanical systems to high complexity and precision machinery.
Since the vibration analysis is the main way to early identify failure in these units, it is of great importance to determine the vibrational pattern of this component in a defect situation.
With that in focus, this work is concentrated in the development of a non-linear, three degrees-of-freedom numeric model for the vibration response of a radial ball bearing with a point defect on the outer ring.
For this, the concepts of the \emph{hertzian} contact deformation theory are employed, in order to determine the force on a rolling element when it strikes the defect.
This force value is used to model three pulse profiles for the load caused by the impact between the ball and the defect.
At the same time, the influence of the lubricant fluid film on the contact between rolling elements and rings is evaluated under the fundamentals of the \emph{elastohydrodynamic} lubrication theory.
The developed model is investigated on its stability and convergence, observing the computational cost of the different pulse shapes for the impact load.
The simulated vibration response is compared to simulation data and experimental tests from previous works in the same application field, analyzing it in both time and frequency domain, in order to conclude on the qualitative consistency of the model.
\vspace{\onelineskip}

\noindent
\textbf{Keywords}: Vibration. Model. Defect. Bearing.
	\end{otherlanguage}
\end{resumo}
% Listas de figuras, quadros, tabelas e siglas
\listoffigures*
\cleardoublepage

%\listofquadros*
%\cleardoublepage

\listoftables*
\cleardoublepage

\listofsiglas*
\cleardoublepage

\listofsimbolos*
\cleardoublepage

% Sumário
\tableofcontents*
\cleardoublepage
