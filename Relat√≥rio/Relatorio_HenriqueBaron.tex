\documentclass[12pt,openright,oneside,a4paper,
	chapter=TITLE,section=TITLE,
	english,brazil]{abntex2}
% Referências a bibliografia no padrão ABNT. Opções: citações numeradas, com sobrescrito e substituindo listas com mais de dois autores por "et al." e escrevendo o "et al." em itálico.
\usepackage[num,abnt-etal-list=2,abnt-etal-text=it,abnt-full-initials=no]{abntex2cite}
\usepackage{ifluatex} % Verificação do compilador utilizado
\ifluatex
	\usepackage{fontspec}
\else
	\usepackage[utf8]{inputenc}
	\usepackage[T1]{fontenc}
\fi

\usepackage[brazil]{babel}

\makeatletter

\makeatother

\usepackage{graphicx}
\graphicspath{{./Imagens/}} % Definição do caminho para as imagens
\usepackage{float} % Utilizado para dispor figuras com a posição "h" (Here)
\usepackage{pdfpages} % Permite a inclusão de páginas em PDF dentro do documento. Será útil para incluir uma cópia digitalizada da folha de aprovação, após as assinaturas.
\usepackage{lmodern} % Altera a fonte do documento para Latin Modern, uma evolução à fonte Computer Modern, padrão do LaTeX.
\usepackage{amsmath} % Digitação de alguns símbolos matemáticos.
\usepackage{abntUCS}
\usepackage{glossaries}
\citebrackets[] % Configura para que as citações sejam feitas entre colchetes.
\includeonly{PreTexto} % Inclusão do arquivo de elementos pré-textuais sem deixar páginas em branco

\makeglossaries % Habilita a criação dos arquivos adicionais necessários à lista de símbolos.
% Para o uso de siglas em outros idiomas, as keys traducao e desctraduzida foram implementadas.
\glsaddkey{traducao}
	{\glsentrylong{\glslabel}}
	{\glsentrytraducao}
	{\Glsentrytraducao}
	{\glstraducao}
	{\Glstraducao}
	{\GLStraducao}
	
\glsaddkey{desctraduzida}
	{\glsentrytraducao{\glslabel} (\glsentrylong{\glslabel}, \glsentryshort{\glslabel})}
	{\glsentrydesctraduzida}
	{\Glsentrydesctraduzida}
	{\glsdesctraduzida}
	{\Glsdesctraduzida}
	{\GLSdesctraduzida}
%-------------------------------------------------------------------------------------------------

\newacronym[traducao={Laboratório Nacional de Energia Renovável}]{sig:nrel}{NREL}{\textit{\foreignlanguage{english}{National Renewable Energy Laboratory}}}
\newacronym[traducao={Transformada Discreta de Fourier}]{sig:dft}{DFT}{\textit{\foreignlanguage{english}{Discrete Fourier Transform}}}
\newcommand{\unidademassa}{[kg]}
\newcommand{\unidadeamortecimento}{[N\cdot s/m]}
\newcommand{\unidaderigidez}{[N/m]}
\newcommand{\unidadeforca}{[N]}
\newcommand{\unidadeposicao}{[m]}
\newcommand{\unidadevelocidade}{[m/s]}
\newcommand{\unidadeaceleracao}{[m/s\textsuperscript{2}]}
\newcommand{\unidadetempo}{[s]}
\newcommand{\unidadeenergia}{[J]}
\newcommand{\unidadeangulo}{[rad]}
\newcommand{\unidadevelocidadeangular}{[rad/s]}
\newcommand{\unidadetorque}{[N\cdot m]}
\newcommand{\unidadearea}{[m\textsuperscript{2}]}
\newcommand{\semunidade}{[-]}

% Medidas elementares e propriedades
\newglossaryentry{smb:graus-liberdade}{name={$ n $},description={número de graus de liberdade do sistema \semunidade}}
\newglossaryentry{smb:tempo}{name={$ t $},description={tempo \unidadetempo}}
\newglossaryentry{smb:periodo}{name={$ \tau $},description={período \unidadetempo}}
\newglossaryentry{smb:area}{name={$ A $},description={área \unidadearea}}
\newglossaryentry{smb:modulo-elast}{name={$ E $},description={módulo de elasticidade [Pa] }}
\newglossaryentry{smb:segundo-momento-area}{name={$ I $},description={segundo momento de área [m\textsuperscript{4}]}}

% Massa
\newglossaryentry{smb:massa-espec}{name={$ \rho $},description={massa específica por unidade de comprimento $ [kg/m] $}}
\newglossaryentry{smb:massa-elemento}{name={$ m_i $},description={massa de um elemento do sistema \unidademassa}}

% Posição, deslocamento, velocidade
\newglossaryentry{smb:posicao}{name={$ x $},description={posição ou deslocamento \unidadeposicao}}
\newglossaryentry{smb:coord-general}{name={$ q_i $},description={posição em coordenadas generalizadas \semunidade}}
\newglossaryentry{smb:desloc-elemento}{name={$ x_i $},description={deslocamento de um elemento do sistema \unidadeposicao}}
\newglossaryentry{smb:desloc-vertical}{name={$ w $},description={deslocamento vertical do corpo elástico \unidadeposicao}}
\newglossaryentry{smb:constante-onda}{name={$ \textit{c} $},description={velocidade de propagação da onda \unidadevelocidade}}

% Frequência
\newglossaryentry{smb:freq}{name={$ \omega $},description={frequêcia \unidadevelocidadeangular}}
\newglossaryentry{smb:freq-natural}{name={$ \omega_n $},description={frequência natural \unidadevelocidadeangular}}
\newglossaryentry{smb:freq-natural-modo}{name={$ \omega_i $},description={frequência natural do i-ésimo modo normal \unidadevelocidadeangular}}

% Vetores
\newglossaryentry{smb:vetor-desloc}{name={$ \vec{x} $},description={vetor de deslocamentos \unidadeposicao}}
\newglossaryentry{smb:vetor-vel}{name={$ \dot{\vec{x}} $},description={vetor de velocidades \unidadevelocidade}}
\newglossaryentry{smb:vetor-acel}{name={$ \ddot{\vec{x}} $},description={vetor de acelerações \unidadeaceleracao}}
\newglossaryentry{smb:vetor-forcas}{name={$ \vec{F} $},description={vetor de forças externas \unidadeforca}}
\newglossaryentry{smb:vetor-veloc-general}{name={$ \dot{\vec{q}} $},description={vetor de velocidades generalizadas \unidadevelocidade}}

% Matrizes e seus componentes
\newglossaryentry{smb:matriz-massa}{name={$ [m] $},description={matriz de massa \unidademassa}}
\newglossaryentry{smb:matriz-massa-diag}{name={$ [m_d] $},description={matriz diagonal de massa \unidademassa}}
\newglossaryentry{smb:matriz-amort}{name={$ [c] $},description={matriz de amortecimento \unidadeamortecimento}}
\newglossaryentry{smb:matriz-rigidez}{name={$ [k] $},description={matriz de rigidez \unidaderigidez}}
\newglossaryentry{smb:coef-massa}{name={$ m_{ij} $},description={coeficiente de massa \unidademassa}}
\newglossaryentry{smb:coef-amortecimento}{name={$ c_{ij} $},description={coeficiente de amortecimento \unidadeamortecimento}}
\newglossaryentry{smb:coef-rigidez}{name={$ k_{ij} $},description={coeficiente de rigidez \unidaderigidez}}

% Força e momento
\newglossaryentry{smb:forca}{name={$ F $},description={força \unidadeforca}}
\newglossaryentry{smb:forca-elemento}{name={$ F_i $},description={força aplicada sobre um elemento do sistema \unidadeforca}}
\newglossaryentry{smb:forca-general}{name={$ Q $},description={força generalizada \unidadeforca}}
\newglossaryentry{smb:forca-naoconserv}{name={$ Q_i^{(n)} $},description={força generalizada não-conservativa do i-ésimo grau de liberdade \unidadeforca}}
\newglossaryentry{smb:forca-cisalhamento}{name={$ C $},description={força de cisalhamento \unidadeforca}}
\newglossaryentry{smb:momento-fletor}{name={$ M $},description={momento fletor \unidadetorque}}
\newglossaryentry{smb:tensao-corda}{name={$ P $},description={força tensora \unidadeforca}}
\newglossaryentry{smb:angulo-tensao}{name={$ \theta $},description={ângulo da corda \unidadeangulo}}

% Energia e trabalho
\newglossaryentry{smb:energ-pot-total}{name={$ V $},description={energia potencial total \unidadeenergia}}
\newglossaryentry{smb:energ-cin-total}{name={$ K $},description={energia cinética total \unidadeenergia}}
\newglossaryentry{smb:energ-pot-elem}{name={$ V_i $},description={energia de deformação de um elemento elástico \unidadeenergia}}
\newglossaryentry{smb:energ-cin-elem}{name={$ K_i $},description={energia cinética do elemento \unidadeenergia}}
\newglossaryentry{smb:lagrangiano}{name={$ L $},description={lagrangiano \unidadeenergia}}
\newglossaryentry{smb:trabalho-coord-general}{name={$ U $},description={trabalho \unidadeenergia}}

% Auxiliares em cálculos
\newglossaryentry{smb:sol-parcial-pos}{name={$ W $},description={solução parcial, dependente da posição x [m\textsuperscript{1/2}]}}
\newglossaryentry{smb:sol-parcial-tempo}{name={$ T $},description={solução parcial, dependente do tempo [m\textsuperscript{1/2}]}}


% Elementos agrupados: criação de uma nova chave para o glossaries que possibilitará exibir na lista de símbolos as variáveis separadas por vírgula e no texto, separadas com a palavra "e" no último termo.
\glsaddkey{expansao}
{\glsentrytext}
{\glsentryexpansao}
{\Glsentryexpansao}
{\glsexpansao}
{\Glsexpansao}
{\GLSexpansao}

\newglossaryentry{smb:const-dif-tempo}{name={$ X, Y $},description={constantes da solução particular da equação diferencial para T \semunidade},expansao={$ X $ e $ Y $}}

\newglossaryentry{smb:const-dif-desloc}{name={$ C_1, C_2, C_3, C_4 $},description={constantes da solução particular da equação diferencial para W \semunidade},expansao={$ C_1, C_2, C_3 $ e $ C_4 $}}

\newglossaryentry{smb:coef-fourier}{name={$ a_0, a_n, b_n $},description={coeficientes de Fourier \semunidade},expansao={$ a_0, a_n $ e $ b_n $}} % Arquivo que contém os símbolos.

\titulo{Desenvolvimento de uma Bancada Didática para Análise de Vibrações em Máquinas}
\autor{Henrique Baron}
\data{2018}
\instituicao{Universidade de Caxias do Sul}
\local{Caxias do Sul}
\preambulo{Trabalho de conclusão de curso apresentado à Universidade de Caxias do Sul como requisito parcial à obtenção do grau de Engenheiro Mecânico. Área de concentração: Projetos de Máquinas: Estática e Dinâmica Aplicada.}
\orientador{prof. Me. Paulo Roberto Linzmaier}

% Configuração do pacote hyperref, que é chamado pela classe abntex2
\makeatletter
\hypersetup{
	unicode=true,
	pdftitle={\@title},
	pdfauthor={\@author},
	pdfsubject={\imprimirpreambulo},
	pdfkeywords={Vibrações. Análise. Modelagem. Ensino.},
	pdfcreator={LaTeX with abnTeX2},
	colorlinks=true,
	linkcolor=black,
	citecolor=black,
}
\makeatother

\begin{document}
	% Capa e folha de rosto
\imprimircapa
\imprimirfolhaderosto
\clearpage

% Folha de aprovação
\imprimirfolhadeaprovacao{29/11/2018}

% Agradecimentos
\begin{agradecimentos}
%	Gostaria de agradecer primeiramente à minha família pelo enorme suporte e incentivo ao desenvolvimento deste trabalho, em especial nos momentos de maior dificuldade, no qual fizeram tudo o que estava ao seu alcance para me ajudar a superar os problemas que enfrentava.
%	A eles e à minha namorada, Fernanda, agradeço a compreensão e paciência durante este período extenuante de estudo e entrega.
%	
%	Às professoras Dra. Kátia Cavalca e Dra. Isolda Gianni de Lima, agradeço pela prontidão e disposição em me ajudar nos momentos em que eu não pude encontrar uma resposta para minhas dúvidas.
%	
%	Agradeço também à empresa Auttom Automação e Robótica, por propiciar a oportunidade de desenvolvimento pessoal com o tema deste trabalho, e pelo suporte que sempre deu à condução dos meus estudos.
\end{agradecimentos}

% Resumo em português
\begin{resumo}
	\SingleSpacing
	O emprego de mancais de rolamento se aplica desde sistemas mecânicos simples até o maquinário de alta complexidade e precisão.
	Sendo a análise de vibrações o meio utilizado para identificação precoce de falhas nestas unidades, é de grande importância determinar o comportamento deste componente em situação de defeito.
	Com este foco, este trabalho concentra-se no desenvolvimento de um modelo numérico não-linear de três graus de liberdade para a resposta de vibração de um rolamento com um defeito pontual no anel externo.
	Para isso, são empregados os conceitos da teoria \emph{hertziana} de deformação por contato, de modo a determinar a força sobre um elemento rolante quando este incide sobre o defeito.
	O valor de força é utilizado para modelar três perfis de pulso diferentes para a carga de impacto entre a esfera do rolamento e o defeito.
	Ao mesmo tempo, a influência do filme de fluido lubrificante no contato entre esferas e anéis é analisado sob os fundamentos da teoria de lubrificação \emph{elastohidrodinâmica}.
	O modelo desenvolvido é investigado quanto à sua estabilidade e convergência, observando o custo computacional das diferentes formas de pulso para a carga de impacto.
	A resposta de vibração simulada é comparada a dados de simulações e verificações experimentais de trabalhos anteriores no mesmo campo de aplicação, analisando-a no domínio do tempo e no domínio da frequência, para concluir quanto à consistência qualitativa do modelo.
	\vspace{\onelineskip}
	
	\noindent
	\textbf{Palavras-chave}: Vibração. Modelagem. Defeito. Rolamento.
\end{resumo}


% Resumo em inglês
\begin{resumo}[\normalsize\bfseries ABSTRACT]
	\SingleSpacing
	\begin{otherlanguage}{english}
		The application of rolling element bearings ranges from simple mechanical systems to high complexity and precision machinery.
		Since the vibration analysis is the main way to early identify failure in these units, it is of great importance to determine the vibrational pattern of this component in a defect situation.
		An experimental approach for the analysis would offer a strict condition for the problem investigation.
		For this reason, the construction of a numeric model is considered to be a more extensive way to understand the phenomenon, while still eliminating interferences caused by the test system.
		With that in focus, the present work has as objective to model the vibration of damaged rolling element bearings and identify the main characteristics of this behavior.
		\vspace{\onelineskip}
		
		\noindent
		\textbf{Keywords}: Vibration. Model. Defect. Bearing.
	\end{otherlanguage}
\end{resumo}
% Listas de figuras, quadros, tabelas e siglas
\listoffigures*
\cleardoublepage

%\listofquadros*
%\cleardoublepage

\listoftables*
\cleardoublepage

\listofsiglas*
\cleardoublepage

\listofsimbolos*
\cleardoublepage

% Sumário
\tableofcontents*
\cleardoublepage

	\textual % Comando que define que os elementos textuais começaram, para incluir a numeração de páginas.
	
	\chapter{Introdução}
	Em todo sistema produtivo em que se faz o uso de ferramentas ou máquinas, a manutenção é um tema que recebe grande atenção, dada a necessidade de manter o funcionamento de equipamentos com o menor tempo possível de parada. Aliado a isso, busca-se continuamente encontrar um balanço entre o custo de repetidas paradas para inspeção e o de uma eventual falha ocasionada por falta de acompanhamento. O montante gasto em manutenção pelas empresas no Brasil no ano de 2013 foi equivalente a 4,69\% do PIB nacional, correspondente a um valor de R\$ 206,5 bilhões \cite{seleme:2015}.
	
	Dentre as metodologias de manutenção existentes, há uma divisão em três grupos principais: corretiva, preditiva e preventiva. Este último, que se configura pelo planejamento de uma rotina de manutenção e inspeções baseadas em intervalos regulares de tempo, ainda é a mais comum na indústria atualmente. Porém, a realização de manutenção baseada no tempo vem mostrando uma baixa confiabilidade nos últimos anos, conforme Hashemian e Bean. Em um teste realizado pelo SKF Group no qual 30 rolamentos de esferas idênticos foram testados até a falha, o tempo de vida das unidades variou de 15 horas até mais de 300 horas \cite{hashemian:2011}.  
	
	Da mesma forma, dados do \glsdesctraduzida{sig:NREL} dos Estados Unidos apontam que 76\% dos casos de falha em caixas de engrenagens de aerogeradores são causados por rolamentos, sendo os defeitos nas próprias engrenagens a segunda maior causa em número de ocorrências (17,1\%). Ao mesmo tempo, apenas 10\% dos rolamentos atingem a sua vida útil esperada por conta da diversidade de condições de desgaste prematuro às quais estes componentes podem ser submetidos \cite{peeters:2017}. Sendo assim, mesmo com uma expectativa de vida estimada para partes de uma máquina, é impossível determinar de maneira exata a sua durabilidade. Isso corrobora, portanto, que a aplicação de uma metodologia de manutenção com inspeções periódicas não elimina o risco de paradas provocadas por falhas inesperadas.
	
	Atualmente, o aumento da oferta de produtos na área de instrumentação e coleta de dados possibilita à indústria o crescimento do uso de técnicas de manutenção preditiva para análise de condições dos equipamentos. A grande vantagem da manutenção preditiva é que, diferentemente da preventiva, o gatilho para a realização de uma operação de manutenção não é um intervalo de tempo, mas sim a mudança em algum sinal emitido pela máquina, que pode ser o indício de um defeito. Entre as técnicas existentes, a medição da vibração em pontos específicos de um sistema traz a vantagem de fornecer sinais de mudança na condição do equipamento em um estágio inicial do possível problema \cite{al-najjar:2003}.
	
	O campo de aplicação mais desenvolvido para a detecção de falhas através de vibrações é o monitoramento de máquinas rotativas. Sua metodologia de análise é baseada no reconhecimento de padrões de frequência e amplitude dos sinais medidos durante o funcionamento normal do equipamento \cite{carden:2004}. Componentes ligados ao eixo de um motor, por exemplo, emitirão vibrações com uma frequência igual ou múltipla da frequência de rotação do eixo. Uma vez entendido um padrão de comportamento, mudanças em propriedades físicas do sistema -- como rigidez, amortecimento ou distribuição de massa -- causarão alterações detectáveis nos dados coletados \cite{qiao:2011}, permitindo identificar a origem da variação no sinal analisado. Isso faz da análise de vibração um dos principais artifícios da manutenção preditiva neste tipo de equipamento.
	
	\section{Justificativa}
	Dada a relevância do conhecimento em vibrações nas aplicações de engenharia, é necessária a formação de profissionais competentes no campo da análise de vibrações em máquinas, em ambos os níveis técnico e acadêmico. E, para tanto, surge a demanda por um equipamento didático que aproxime o aluno da técnica aplicada e estabeleça uma conexão com a fundamentação teórica tratada em sala de aula. Ao mesmo tempo, tal produto deve não somente expor uma situação prática, mas também comprovar -- dentro de uma dada tolerância -- que os valores calculados no ambiente teórico se reproduzem no mundo real.
	
	Considerando a escassez da oferta de um equipamento para estudo de vibrações no mercado brasileiro, enxergou-se a possibilidade de desenvolvimento de uma bancada didática que trouxesse os conceitos de análise de vibrações em máquinas para escolas técnicas e universidades. No entanto, para o desenvolvimento de um produto para o mercado de ensino, é vital que exista uma fidelidade entre o equipamento construído e a teoria que ele demonstra. Portanto, a fabricação de um protótipo e a validação deste com um modelo teórico fundamentado são etapas importantes na concepção do produto.
	
	\section{Ambiente de desenvolvimento}	
	O trabalho foi conduzido no departamento de Pesquisa e Desenvolvimento da unidade de educação da empresa Auttom Automação e Robótica Ltda., situada em Caxias do Sul. A empresa possui duas divisões: uma concentrada em projetos de automação sob demanda; e a outra, onde o trabalho foi realizado, desenvolve bancadas didáticas e sistemas de ensino para escolas técnicas e universidades.
	
	Atualmente, a unidade de educação da empresa conta com uma linha ampla de produtos nas áreas de automação industrial, energias renováveis, eletricidade e refrigeração. Ainda não há uma oferta considerável de produtos para estudo de mecânica -- isto é, desde componentes mecânicos, materiais, ou mesmo a análise de sistemas, como é o caso da análise de vibrações -- e o desenvolvimento deste trabalho surge como uma possibilidade de inserção da empresa em um mercado ainda pouco explorado.
	
	\section{Objetivos}
	
	\subsection{Objetivo geral}
	O objetivo do presente trabalho é validar o protótipo de uma bancada didática para análise de vibrações em máquinas rotativas, através de simulação numérica e testes com protótipo montado.
	
	\subsection{Objetivos específicos}
	Para o cumprimento do objetivo geral, são estabelecidos os seguintes objetivos específicos:
	\setenumerate[0]{label=\alph*)}
	\begin{enumerate}
		\setlength{\itemsep}{0pt plus 2pt minus 1pt}
		\item Desenvolvimento de um modelo numérico da bancada no MATLAB\textsuperscript{\textregistered};
		\item Montagem do protótipo e instalação de um sistema de aquisição de dados;
		\item Desenvolvimento de uma aplicação no LabVIEW\textsuperscript{\textregistered} para coleta e análise dos dados;
		\item Comparação dos resultados obtidos no modelo numérico e no protótipo construído.
	\end{enumerate}

	\chapter{Desenvolvimento}
	
	\section{Vibração de sistemas discretos}
	Denominam-se sistemas discretos aqueles que são descritos por um número finito de graus de liberdade. Embora grande parte das estruturas e máquinas possuam elementos elásticos e tenham, portanto, infinitos graus de liberdade, é comum discretizá-los através da divisão dos corpos rígidos em um número conhecido e distribuído de massas pontuais, para simplificação do problema. 
	
	\subsection{Equação de movimento}
	O movimento linear de um sistema massa-mola-amortecedor qualquer é expresso por \cite{rao:2008}
	\begin{equation} \label{eqn:discr:movimento}
		[m]\ddot{\vec{x}} + [c]\dot{\vec{x}} + [k]\vec{x} = \vec{F}
	\end{equation}
	onde \gls{smb:matriz-massa}, \gls{smb:matriz-amort} e \gls{smb:matriz-rigidez} representam as matrizes de massa, amortecimento e rigidez, respectivamente, e são dadas por
	\begin{gather} 
		[m] = 
		\begin{bmatrix} \label{eqn:matr:massa}
			m_{11} & m_{12} & m_{13} & \dots & m_{1n}\\
			m_{21} & m_{22} & m_{23} & \dots & m_{2n}\\
			\vdots\\
			m_{n1} & m_{n2} & m_{n3} & \dots & m_{nn}\\
		\end{bmatrix}\\
		[c] = 
		\begin{bmatrix} \label{eqn:matr:amort}
			c_{11} & c_{12} & c_{13} & \dots & c_{1n}\\
			c_{21} & c_{22} & c_{23} & \dots & c_{2n}\\
			\vdots\\
			c_{n1} & c_{n2} & c_{n3} & \dots & c_{nn}\\
		\end{bmatrix}\\
		[k] =
		\begin{bmatrix} \label{eqn:matr:rigid}
			k_{11} & k_{12} & k_{13} & \dots & k_{1n}\\
			k_{21} & k_{22} & k_{23} & \dots & k_{2n}\\
			\vdots\\
			k_{n1} & k_{n2} & k_{n3} & \dots & k_{nn}\\
		\end{bmatrix}
	\end{gather}
	onde \gls{smb:coef-massa}, \gls{smb:coef-amortecimento} e \gls{smb:coef-rigidez} são os coeficientes de massa, amortecimento e rigidez da matriz correspondente.	Além destes, os termos \gls{smb:vetor-desloc}, \gls{smb:vetor-vel}, \gls{smb:vetor-acel} e \gls{smb:vetor-forcas} são os vetores de deslocamento, velocidade, aceleração e força, definidos por
	\begin{align} \label{eqn:vetores}
		\vec{x} =
		\begin{Bmatrix}
			x_1(t)\\ x_2(t)\\ \vdots\\ x_n(t)\\
		\end{Bmatrix},\qquad &
		\dot{\vec{x}} = 
		\begin{Bmatrix}
			\dot{x}_1(t)\\ \dot{x}_2(t)\\ \vdots\\ \dot{x}_n(t)\\
		\end{Bmatrix}, \notag\\
		\ddot{\vec{x}} = 
		\begin{Bmatrix}
			\ddot{x}_1(t)\\ \ddot{x}_2(t)\\ \vdots \\ \ddot{x}_n(t)\\  
		\end{Bmatrix},\qquad &
		\vec{F} = 
		\begin{Bmatrix}
			F_1(t)\\ F_2(t)\\ \vdots\\ F_n(t)\\
		\end{Bmatrix}
	\end{align}
	
	\subsection{Coordenadas e forças generalizadas} \label{sec:coord-generalizadas}
	O conjunto de coordenadas independentes necessárias para descrever o movimento de um sistema é denominado coordenadas generalizadas. Elas são representadas como $q_1$, $q_2$, \dots\ , $q_n$ e podem ser compostas por medidas lineares, ângulos ou qualquer arranjo de números que defina a configuração do sistema de maneira exclusiva em qualquer instante \cite{rao:2008}.
	
	Com a ação de forças externas, toma-se a mudança das coordenadas generalizadas \gls{smb:coord-general} de um incremento $\delta q_i $, $ i = 1, 2, \dots, n $, onde \gls{smb:graus-liberdade} denota o número de graus de liberdade do sistema. Seja \gls{smb:trabalho-coord-general} o trabalho realizado na variação da coordenada generalizada, a força generalizada correspondente \gls{smb:forca-general} é definida por \citeauthoronline{rao:2008} (\citeyear{rao:2008}) como
	\begin{equation}
		Q_i = \frac{U_i}{\delta q_i},\qquad i = 1, 2, \dots, n
	\end{equation}
	
	\subsection{Equação de Lagrange}
	A resolução de um problema de vibração pela abordagem de Lagrange consiste em uma análise das energias potencial -- isto é, de deformação -- e cinética do sistema. O uso de grandezas escalares trazido por esta metodologia é conveniente em problemas complexos nos quais a utilização das dimensões vetoriais torna a resolução difícil \cite{savi:2017}.
	
	Considerando como \gls{smb:desloc-elemento} o deslocamento de uma massa \gls{smb:massa-elemento} e seja \gls{smb:forca-elemento} a força nela aplicada, a energia de deformação \gls{smb:energ-pot-elem} da i-ésima mola ou elemento elástico de um sistema com \gls{smb:graus-liberdade} graus de liberdade é dada por \cite{rao:2008}
	\begin{equation}
		V_i = \frac{1}{2}F_i x_i
	\end{equation}
	A energia potencial total \gls{smb:energ-pot-total} é, então
	\begin{equation} \label{eqn:discr:energia:deformacao}
		V = \sum_{i-1}^{n}V_i = \frac{1}{2}\sum_{i-1}^{n}F_i x_i
	\end{equation}
	Fazendo a substituição
	\begin{equation}
		F_i = \sum_{j=1}^{n}k_{ij} x_j
	\end{equation}
	onde \gls{smb:coef-rigidez} representa o coeficiente de rigidez da i-ésima linha e j-ésima coluna na matriz de rigidez da Equação (\ref{eqn:matr:rigid}), a Equação (\ref{eqn:discr:energia:deformacao}) fornece
	\begin{equation}
		V = \frac{1}{2}\sum_{i=1}^{n}\left(\sum_{j=1}^{n}k_{ij} x_j\right)x_i = \frac{1}{2}\sum_{i=1}^{n}\sum_{j=1}^{n}k_{ij} x_i x_j
	\end{equation}
	que pode ser expressa na forma matricial como
	\begin{equation}
		V = \frac{1}{2}\vec{x}^T [k] \vec{x}
	\end{equation}
	de modo que \gls{smb:matriz-rigidez} e \gls{smb:vetor-desloc} denotam respectivamente a matriz de rigidez da Equação (\ref{eqn:matr:rigid}) e o vetor de deslocamentos da Equação (\ref{eqn:vetores}).
	
	Para a mesma massa \gls{smb:massa-elemento}, a energia cinética associada é denotada por \gls{smb:energ-cin-elem} e definida como \cite{rao:2008}
	\begin{equation}
		K_i = \frac{1}{2}m_i\dot{x}_i^2
	\end{equation}
	e, de maneira análoga à Equação (\ref{eqn:discr:energia:deformacao}), a energia cinética total do sistema, \gls{smb:energ-cin-total}, é obtida realizando o somatório para todos os \gls{smb:graus-liberdade} elementos, resultando em \cite{savi:2017}
	\begin{equation}\label{eqn:discr:energia:cinetica}
		K = \frac{1}{2}\dot{\vec{x}^T}[m_d]\dot{\vec{x}}
	\end{equation}
	onde \gls{smb:vetor-vel} corresponde ao vetor de velocidades da Equação (\ref{eqn:vetores}) e \gls{smb:matriz-massa-diag} é uma matriz de massa diagonal
	\begin{equation}
		[m_d] =
		\begin{bmatrix}
			m_1 & & & 0\\
			& m_2\\
			& & \ddots\\
			0 & & & m_n\\
		\end{bmatrix}
	\end{equation}
	
	Se, na Equação (\ref{eqn:discr:energia:cinetica}) forem utilizadas as coordenadas generalizadas \gls{smb:coord-general} descritas na Seção \ref{sec:coord-generalizadas}, arranjadas em um vetor de velocidades generalizadas \gls{smb:vetor-veloc-general} dado por
	\begin{equation}
		\dot{\vec{q}} = 
		\begin{Bmatrix}
			\dot{q_1}\\ \dot{q_2}\\ \vdots\\ \dot{q_n}\\
		\end{Bmatrix}
	\end{equation}
	então essa mesma expressão pode ser enunciada como \cite{rao:2008}
	\begin{equation}
		K = \frac{1}{2}\dot{\vec{q}^T}[m]\dot{\vec{q}}
	\end{equation}
	sendo agora \gls{smb:matriz-massa} a matriz de massa da Equação (\ref{eqn:matr:massa}).
	
	Uma vez determinadas as expressões em termos das energias potencial e cinética, é possível definir as equações de Lagrange como \cite{savi:2017}
	\begin{equation} \label{eqn:lagrange}
		\frac{d}{dt}\left(\frac{\partial L}{\partial \dot{q}_i}\right) - \frac{\partial L}{\partial q_i} = Q_i^{(n)}, \qquad i = 1, 2, \dots, n
	\end{equation}
	onde \gls{smb:lagrangiano} é o lagrangiano
	\begin{equation}
		L = K-V
	\end{equation}
	e \gls{smb:forca-naoconserv} é a força generalizada não-conservativa correspondente à coordenada generalizada \gls{smb:coord-general} Estas forças podem ser provenientes do amortecimento ou de quaisquer influências externas que não são deriváveis de uma função potencial \cite{rao:2008}. Se o sistema for conservativo, portanto, $ Q_j^{(n)} = 0 $.
	
	\section{Vibração de sistemas contínuos}
	Em contraste aos sistemas discretos, os sistemas contínuos são aqueles em que não se consegue identificar massas, molas e amortecedores pontuais na vibração de um corpo \cite{rao:2008}. Para estes casos, é possível determinar o comportamento vibratório de maneira analítica e sem discretização. Através dessa abordagem podem ser modeladas barras, eixos, cabos, vigas e outros elementos \cite{timoshenko:1974}. 
	
	Ao tratar um corpo como um elemento elástico contínuo, ele é considerado como sendo composto de um número infinito de partículas infinitesimais, e é tratado como um sistema com infinitos gruas de liberdade. Dessa maneira, estruturas geometricamente complexas tornam-se demasiado difíceis -- ou mesmo impossíveis -- de serem modeladas por este método, restando para estes casos métodos que envolvam a discretização em um número finito de graus de liberdade \cite{timoshenko:1974}.
	
	\subsection{Equação de movimento} \label{desenv:cont:eqn-geral}
	A equação geral de movimento de um sistema contínuo pode ser deduzida a partir da vibração vertical de uma corda sob tensão. Esteja uma corda com uma massa por unidade de comprimento equivalente a \gls{smb:massa-espec}, sujeita a uma força transversal \gls{smb:forca} que é função do tempo \gls{smb:tempo} e da coordenada $x$, como mostra o lado esquerdo da Figura \ref{fig:vibracao-corda}. O lado direito da figura mostra em detalhe o intervalo infinitesimal $dx$ entre os pontos $x_1$ e $x_2$. Nele também são esquematizados o deslocamento \gls{smb:desloc-vertical} na direção $z$, a força de tração \gls{smb:tensao-corda}, o ângulo de inclinação da corda em relação ao eixo $x$ denotado por \gls{smb:angulo-tensao} e, por fim, a força \gls{smb:forca}$(x,t)$ atuante sobre o segmento infinitesimal de comprimento $ds$ da corda.
	\begin{figure}[t] 
	\incluirimagem{VibracaoCorda.png}{Vibração de uma corda tensionada}{Fonte: adaptado de Rao (2008)} \label{fig:vibracao-corda}
	\end{figure}

	Neste modelo, desconsidera-se a ação da atmosfera e da gravidade sobre a corda, além de supor-se que não existem perdas de energia na sua vibração. Também define-se a corda vibrante como tendo uma seção transversal pequena em comparação com seu comprimento, de maneira que as tensões ao longo da área da seção variem pouco e se tornem puramente axiais, desprezando assim tensões de flexão e cisalhamento \cite{clark:1972}.
	
	Com estas considerações, o balanço de forças na direção $z$ para o elemento da Figura \ref{fig:vibracao-corda} é o somatório da força transversal \gls{smb:forca} e da componente vertical da força de tração \gls{smb:tensao-corda} \cite{rao:2008}. A força líquida corresponde à força de inércia atuante sobre o elemento, portanto
	\begin{equation} \label{eqn:cont:balanco}
		(P + dP)\;sen(\theta + d\theta) + F\:dx - P\;sen\:\theta = \rho\:dx\:\frac{\partial^2 w}{\partial t^2}
	\end{equation}
	
	Para um comprimento elementar $dx$ a força de tração $dP$ pode ser escrita como \[dP = \frac{\partial P}{\partial x}\:dx\] e, considerando que a corda esteja firmemente esticada, estima-se que o ângulo \gls{smb:angulo-tensao} seja pequeno, permitindo fazer as simplificações \[sen\:\theta \simeq tg\:\theta = \frac{\partial w}{\partial x} \] e \[sen(\theta + d\theta) \simeq tg(\theta + d\theta) = \frac{\partial w}{\partial x} + \frac{\partial^2 w}{\partial x^2}\:dx \] que reduzem a Equação (\ref{eqn:cont:balanco}) para \[\frac{\partial}{\partial x}\,\left[P\:\frac{\partial w(x,t)}{\partial x}\right] + F(x,t) = \rho(x)\:\frac{\partial^2 w(x,t)}{\partial t^2} \]
	
	Considerando que a massa linear específica \gls{smb:massa-espec} é uniforme, que a tensão \gls{smb:tensao-corda} é constante e que a corda esteja em vibração livre -- isto é, \gls{smb:forca}$(x,t) = 0$ -- chega-se a \cite{rao:2008}
	\begin{align}
		\textit{c}^2\:\frac{\partial^2 w}{\partial x^2}=\frac{\partial^2 w}{\partial t^2} \label{eqn:cont:onda}\\
		\textit{c} = \left(\frac{P}{\rho}\right)^{1/2}
	\end{align}
	
	A Equação (\ref{eqn:cont:onda}) também é conhecida como equação de onda. 
	
	\subsection{Vibração de vigas} \label{sec:vigas}
	Aqui, o conceito apresentado na Seção \ref{desenv:cont:eqn-geral} será empregado na determinação da vibração de vigas elásticas. Considera-se a viga como um elemento estrutural com um módulo de elasticidade \gls{smb:modulo-elast} e momento de inércia de área \gls{smb:segundo-momento-area}, que vibra com deslocamento \gls{smb:desloc-vertical} na direção vertical $z$. Diferentemente da vibração de uma corda, que não possui rigidez de flexão, para a viga a rigidez $EI$ deve ser considerada \cite{timoshenko:1974}. Seja uma viga sujeita a um momento fletor \gls{smb:momento-fletor} e a uma força de cisalhamento \gls{smb:forca-cisalhamento}, conforme mostra o elemento infinitesimal da Figura \ref{fig:elemento-viga}. É possível mostrar que a vibração livre de uma viga de massa linear específica \gls{smb:massa-espec} e área de seção transversal \gls{smb:area} constantes também obedece à Equação (\ref{eqn:cont:onda}), no entanto, com a constante da onda \gls{smb:constante-onda} valendo \cite{rao:2008}
	\begin{equation}
		\textit{c} = \sqrt{\frac{EI}{\rho A}}
	\end{equation}		
	\begin{figure}[b]
		\incluirimagem{ElementoViga.png}{Elemento de viga em flexão}{Fonte: adaptado de Timoshenko (1974)}
		\label{fig:elemento-viga}
	\end{figure}

	A Equação (\ref{eqn:cont:onda}) pode ser resolvida pelo método da separação de variáveis \cite{clark:1972}, de modo que o deslocamento \gls{smb:desloc-vertical}$(x,t)$ é descrito como
	\begin{equation} \label{eqn:cont:sep-variaveis}
		w(x,t) = W(x)\cdot T(t)
	\end{equation}
	que mostra que \gls{smb:sol-parcial-pos} é uma função dependente somente da posição em $x$ e \gls{smb:sol-parcial-tempo} depende unicamente do tempo \gls{smb:tempo}.
	
	Substituindo a Equação (\ref{eqn:cont:sep-variaveis}) na Equação (\ref{eqn:cont:onda}) e rearranjando os termos, obtém-se 
	\begin{equation} \label{eqn:cont:geral-vigas}
		\frac{1}{T}\,\frac{d^2T}{dt^2} + \frac{\textit{c}^2}{X}\,\frac{d^4W}{dx^4} = 0
	\end{equation}
	o que permite \cite{clark:1972} igualar cada parte da equação a uma constante, de modo que
	\begin{align*}
		\frac{1}{T}\,\frac{d^2T}{dt^2} = -\omega^2 \\
		\frac{\textit{c}^2}{W}\,\frac{d^4W}{dx^2} = \omega^2
	\end{align*}
	onde a constante \gls{smb:freq-natural} acrescentada é a frequência natural. 
	
	Para a equação diferencial de \gls{smb:sol-parcial-tempo}$(t)$, a solução é uma oscilação harmônica simples 
	\begin{equation} \label{eqn:cont:sep-T}
		T(t) = X\,cos\,\omega t + Y\,sen\,\omega t
	\end{equation}
	com as constantes \glsexpansao{smb:const-dif-tempo} para a devida solução particular. Para a função \gls{smb:sol-parcial-pos}$(x)$, pode-se reescrever como \[\frac{d^4W}{dx^4} = \beta^4W \] definindo a constante $\beta$ que vale \[\beta^4 = \frac{\omega^2}{\textit{c}^2} \] e obtendo para esta equação diferencial uma solução da forma
	\begin{equation} \label{eqn:cont:sep-W}
		W(x) = C_1\:sen\,\beta x + C_2\:cos\,\beta x + C_3\:senh\,\beta x + C_4\:cosh\,\beta x
	\end{equation}
	onde as constantes \glsexpansao{smb:const-dif-desloc} definem a respectiva solução particular.
	
	Visto que solução total de cada uma das Equações (\ref{eqn:cont:sep-T}) e (\ref{eqn:cont:sep-W}) é a soma de todas as combinações lineares possíveis das soluções de cada equação \cite{boyce:2017}, segue que a equação que rege o movimento pode ser definida como a série infinita \cite{clark:1972}
	\begin{equation}
		w(x,t) = \sum_{i=1}^{\infty}w_i(x,t) = \sum_{i=1}^{\infty}W_i(x)(X_i\,cos\,\omega_i t + Y_i\,sen\,\omega_i t)
	\end{equation}
	onde o índice $i$ representa o $i$-ésimo modo normal de vibração.
	
	\section{Dinâmica de rotores}
	Nos sistemas analisados nas seções anteriores, considerava-se o deslocamento em relação a uma posição de equilíbrio fixa, de modo que o movimento era analisado apenas em uma direção e a existência de vibração simultânea ao longo de outra coordenada não afetaria a análise, e as respectivas equações de movimento seriam desacopladas entre si. Um caso frequente de vibração que ocorre sobre um movimento contínuo -- portanto, com equações acopladas -- é visto na rotação de eixos \cite{dimarogonas:1995}.

	\postextual
	
	\bibliography{Bibliografia}
		
\end{document}