\documentclass[12pt,openright,oneside,a4paper,
	chapter=TITLE,section=TITLE,
	english,brazil]{abntex2}
% Referências a bibliografia no padrão ABNT. Opções: citações numeradas, com sobrescrito e substituindo listas com mais de dois autores por "et al." e escrevendo o "et al." em itálico.
\usepackage[num, abnt-etal-list=2, abnt-etal-text=it]{abntex2cite}
\usepackage[brazil]{babel}
\usepackage[utf8]{inputenc}
\usepackage[T1]{fontenc}
\usepackage{graphicx}
\usepackage{float} % Utilizado para dispor figuras com a posição "h" (Here)
\usepackage{pdfpages} % Permite a inclusão de páginas em PDF dentro do documento. Será útil para incluir uma cópia digitalizada da folha de aprovação, após as assinaturas.
\usepackage{lmodern} % Altera a fonte do documento para Latin Modern, uma evolução à fonte Computer Modern, padrão do LaTeX.
\usepackage{textcomp}
\usepackage{amsmath} % Digitação de alguns símbolos matemáticos.
\usepackage{abntUCS}
\citebrackets[] % Configura para que as citações sejam feitas entre colchetes.
\includeonly{PreTexto}

\titulo{Desenvolvimento de uma Bancada Didática para Análise de Vibrações em Máquinas}
\autor{Henrique Baron}
\data{2018}
\instituicao{Universidade de Caxias do Sul}
\local{Caxias do Sul}
\preambulo{Trabalho de conclusão de curso apresentado à Universidade de Caxias do Sul como requisito parcial à obtenção do grau de Engenheiro Mecânico. Área de concentração: Projetos de Máquinas: Estática e Dinâmica Aplicada.}
\orientador{prof. Me. Paulo Roberto Linzmaier}

% Configuração do pacote hyperref, que é chamado pela classe abntex2
\makeatletter
\hypersetup{
	pdftitle={\@title},
	pdfauthor={\@author},
	pdfsubject={\imprimirpreambulo},
	pdfkeywords={Vibrações. Análise. Modelagem. Ensino.},
	pdfcreator={LaTeX with abnTeX2},
	colorlinks=true,
	linkcolor=black,
	citecolor=black,
}
\makeatother

\begin{document}
	% Capa e folha de rosto
\imprimircapa
\imprimirfolhaderosto
\clearpage

% Folha de aprovação
\imprimirfolhadeaprovacao{29/11/2018}

% Agradecimentos
\begin{agradecimentos}
%	Gostaria de agradecer primeiramente à minha família pelo enorme suporte e incentivo ao desenvolvimento deste trabalho, em especial nos momentos de maior dificuldade, no qual fizeram tudo o que estava ao seu alcance para me ajudar a superar os problemas que enfrentava.
%	A eles e à minha namorada, Fernanda, agradeço a compreensão e paciência durante este período extenuante de estudo e entrega.
%	
%	Às professoras Dra. Kátia Cavalca e Dra. Isolda Gianni de Lima, agradeço pela prontidão e disposição em me ajudar nos momentos em que eu não pude encontrar uma resposta para minhas dúvidas.
%	
%	Agradeço também à empresa Auttom Automação e Robótica, por propiciar a oportunidade de desenvolvimento pessoal com o tema deste trabalho, e pelo suporte que sempre deu à condução dos meus estudos.
\end{agradecimentos}

% Resumo em português
\begin{resumo}
	\SingleSpacing
	O emprego de mancais de rolamento se aplica desde sistemas mecânicos simples até o maquinário de alta complexidade e precisão.
	Sendo a análise de vibrações o meio utilizado para identificação precoce de falhas nestas unidades, é de grande importância determinar o comportamento deste componente em situação de defeito.
	Com este foco, este trabalho concentra-se no desenvolvimento de um modelo numérico não-linear de três graus de liberdade para a resposta de vibração de um rolamento com um defeito pontual no anel externo.
	Para isso, são empregados os conceitos da teoria \emph{hertziana} de deformação por contato, de modo a determinar a força sobre um elemento rolante quando este incide sobre o defeito.
	O valor de força é utilizado para modelar três perfis de pulso diferentes para a carga de impacto entre a esfera do rolamento e o defeito.
	Ao mesmo tempo, a influência do filme de fluido lubrificante no contato entre esferas e anéis é analisado sob os fundamentos da teoria de lubrificação \emph{elastohidrodinâmica}.
	O modelo desenvolvido é investigado quanto à sua estabilidade e convergência, observando o custo computacional das diferentes formas de pulso para a carga de impacto.
	A resposta de vibração simulada é comparada a dados de simulações e verificações experimentais de trabalhos anteriores no mesmo campo de aplicação, analisando-a no domínio do tempo e no domínio da frequência, para concluir quanto à consistência qualitativa do modelo.
	\vspace{\onelineskip}
	
	\noindent
	\textbf{Palavras-chave}: Vibração. Modelagem. Defeito. Rolamento.
\end{resumo}


% Resumo em inglês
\begin{resumo}[\normalsize\bfseries ABSTRACT]
	\SingleSpacing
	\begin{otherlanguage}{english}
		The application of rolling element bearings ranges from simple mechanical systems to high complexity and precision machinery.
		Since the vibration analysis is the main way to early identify failure in these units, it is of great importance to determine the vibrational pattern of this component in a defect situation.
		An experimental approach for the analysis would offer a strict condition for the problem investigation.
		For this reason, the construction of a numeric model is considered to be a more extensive way to understand the phenomenon, while still eliminating interferences caused by the test system.
		With that in focus, the present work has as objective to model the vibration of damaged rolling element bearings and identify the main characteristics of this behavior.
		\vspace{\onelineskip}
		
		\noindent
		\textbf{Keywords}: Vibration. Model. Defect. Bearing.
	\end{otherlanguage}
\end{resumo}
% Listas de figuras, quadros, tabelas e siglas
\listoffigures*
\cleardoublepage

%\listofquadros*
%\cleardoublepage

\listoftables*
\cleardoublepage

\listofsiglas*
\cleardoublepage

\listofsimbolos*
\cleardoublepage

% Sumário
\tableofcontents*
\cleardoublepage

	\textual % Comando que define que os elementos textuais começaram, para incluir a numeração de páginas.
	
	\chapter{Introdução}
	Em todo o tipo de sistema produtivo em que se faz o uso de ferramentas ou máquinas, a manutenção é um tema que recebe grande atenção, dada a necessidade de manter o funcionamento de equipamentos com o menor tempo possível de parada, buscando-se encontrar um balanço entre o custo de repetidas paradas para inspeção e o de uma eventual falha ocasionada por falta de acompanhamento. O montante gasto em manutenção pelas empresas no Brasil no ano de 2013 foi equivalente a 4,69\% do PIB nacional, correspondente a um valor de R\$ 206,5 bilhões \cite{bib:seleme}.
	
	Dentre as metodologias de manutenção existentes, há uma divisão em três grupos principais: corretiva, preditiva e preventiva. Este último, que se configura pelo planejamento de uma rotina de manutenção e inspeções baseadas em intervalos regulares de tempo, ainda é a mais comum na indústria atualmente. Porém, a realização de manutenção baseada no tempo vem mostrando uma baixa confiabilidade nos últimos anos, conforme Hashemian e Bean. Em um teste realizado pelo SKF Group no qual 30 rolamentos de esferas idênticos foram testados até a falha, o tempo de vida das unidades variou de 15 horas até mais de 300 horas \cite{bib:hashemian}. 
	
	Da mesma forma, dados do Laboratório Nacional de Energia Renovável dos Estados Unidos (\textit{\foreignlanguage{english}{National Renewable Energy Laboratory}} ou NREL) apontam que 76\% dos casos de falha em caixas de engrenagens de aerogeradores são causados por rolamentos, sendo os defeitos nas próprias engrenagens a segunda maior causa em número de ocorrências (17,1\%). Ao mesmo tempo, apenas 10\% dos rolamentos atingem a sua vida útil esperada devido à diversidade de condições de desgaste prematuro às quais estes componentes podem ser submetidos \cite{bib:peeters}. Isso demonstra que, mesmo com uma expectativa de vida estimada para componentes de uma máquina, é impossível determinar de maneira exata qual será a sua durabilidade, corroborando portanto, que a aplicação de uma metodologia de manutenção com inspeções periódicas não elimina o risco de paradas provocadas por falhas inesperadas.
	
	Atualmente, o aumento da oferta de produtos na área de instrumentação e coleta de dados possibilita à indústria o crescimento do uso de técnicas de manutenção preditiva para análise de condições dos equipamentos. A grande vantagem da manutenção preditiva é que, diferentemente da preventiva, o gatilho para a realização de uma operação de manutenção não é um intervalo de tempo, mas sim a mudança em algum sinal emitido pela máquina, que pode ser o indício de um defeito. Entre as técnicas existentes, a medição da vibração em pontos específicos de um sistema traz a vantagem de fornecer sinais de mudança na condição do equipamento em um estágio inicial do possível problema \cite{bib:al-najjar}.
	
	O campo de aplicação mais desenvolvido para a detecção de falhas através de vibrações é o monitoramento de máquinas rotativas. Sua metodologia de análise é baseada no reconhecimento de padrões de frequência e amplitude dos sinais medidos durante o funcionamento normal do equipamento \cite{bib:carden}. Componentes ligados ao eixo de um motor, por exemplo, emitirão vibrações com uma frequência igual ou múltipla da frequência de rotação do eixo. Uma vez entendido um padrão de comportamento, mudanças em propriedades físicas do sistema -- como rigidez, amortecimento ou distribuição de massa -- causarão alterações detectáveis nos dados coletados \cite{bib:qiao}, permitindo identificar a origem da variação no sinal analisado. Isso faz da análise de vibração um dos principais artifícios da manutenção preditiva neste tipo de equipamento.
	
	\section{Justificativa}
	Dada a relevância do conhecimento em vibrações nas aplicações de engenharia, é necessária a formação de profissionais competentes no campo da análise de vibrações em máquinas, em ambos os níveis técnico e acadêmico. E, para tanto, surge a demanda por um equipamento didático que aproxime o aluno da técnica aplicada e estabeleça uma conexão com a fundamentação teórica tratada em sala de aula. Ao mesmo tempo, tal produto deve não somente expor uma situação prática, mas também comprovar -- dentro de uma dada tolerância -- que os valores calculados no ambiente teórico se reproduzem no mundo real.
	
	Considerando a escassez da oferta de um equipamento para estudo de vibrações no mercado brasileiro, enxergou-se a possibilidade de desenvolvimento de uma bancada didática que trouxesse os conceitos de análise de vibrações em máquinas para escolas técnicas e universidades. No entanto, para o desenvolvimento de um produto para o mercado de ensino, é vital que exista uma fidelidade entre o equipamento construído e a teoria que ele demonstra. Portanto, a fabricação de um protótipo e a validação deste com um modelo teórico fundamentado são etapas importantes na concepção do produto.
	
	\section{Ambiente de desenvolvimento}	
	O trabalho foi conduzido no departamento de Pesquisa e Desenvolvimento da unidade de educação da empresa Auttom Automação e Robótica Ltda., situada em Caxias do Sul. A empresa possui duas divisões: uma concentrada em projetos de automação sob demanda; e a outra, onde o trabalho foi realizado, desenvolve bancadas didáticas e sistemas de ensino para escolas técnicas e universidades.
	
	Atualmente, a unidade de educação da empresa conta com uma linha ampla de produtos nas áreas de automação industrial, energias renováveis, eletricidade e refrigeração. Ainda não há uma oferta considerável de produtos para estudo de mecânica -- isto é, desde componentes mecânicos, materiais, ou mesmo a análise de sistemas mecânicos, como é o caso da análise de vibrações -- e o desenvolvimento deste trabalho surge como uma possibilidade de inclusão da empresa em um mercado ainda pouco explorado.
	
	\section{Objetivos}
	
	\subsection{Objetivo geral}
	O objetivo do presente trabalho é validar o protótipo de uma bancada didática para análise de vibrações em máquinas rotativas, através de simulação numérica e testes com protótipo montado.
	
	\subsection{Objetivos específicos}
	Para o cumprimento do objetivo geral, são estabelecidos os seguintes objetivos específicos:
	\setenumerate[0]{label=\alph*)}
	\begin{enumerate}
		\setlength{\itemsep}{0pt plus 2pt minus 1pt}
		\item Desenvolvimento de um modelo numérico da bancada no MATLAB\textsuperscript{\textregistered}; \label{obj:modelo}
		\item Montagem do protótipo e instalação de um sistema de aquisição de dados;
		\item Desenvolvimento de uma aplicação no LabVIEW\textsuperscript{\textregistered} para coleta e análise dos dados; \label{obj:teste}
		\item Comparação dos resultados obtidos no modelo numérico e no protótipo construído.
	\end{enumerate}

	\postextual
	
	\bibliography{Bibliografia}
		
\end{document}