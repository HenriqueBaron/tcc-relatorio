\documentclass[12pt,openright,twoside,a4paper,
	chapter=TITLE,section=TITLE,
	english,brazil]{abntex2}
\usepackage[brazil]{babel}
\usepackage[utf8]{inputenc}
\usepackage[T1]{fontenc}
\usepackage{graphicx}
\usepackage{float} % Utilizado para dispor figuras com a posição "h" (Here)
\usepackage{pdfpages} % Permite a inclusão de páginas em PDF dentro do documento. Será útil para incluir uma cópia digitalizada da folha de aprovação, após as assinaturas.
\usepackage{lmodern} % Altera a fonte do documento para Latin Modern, uma evolução à fonte Computer Modern, padrão do LaTeX.
\usepackage{abntUCS}
\includeonly{PreTexto}

\titulo{Desenvolvimento de uma Bancada Didática para Análise de Vibrações em Máquinas}
\autor{Henrique Baron}
\data{2018}
\instituicao{Universidade de Caxias do Sul}
\local{Caxias do Sul}
\preambulo{Trabalho desenvolvido para obtenção do grau no curso de Bacharelado em Engenharia Mecânica.}
\orientador{Me. Paulo Roberto Linzmaier}

% Configuração do pacote hyperref, que é chamado pela classe abntex2
\makeatletter
\hypersetup{
	pdftitle={\@title},
	pdfauthor={\@author},
	pdfsubject={\imprimirpreambulo},
	pdfkeywords={PALAVRAS}{CHAVE},
	pdfcreator={LaTeX with abnTeX2},
	colorlinks=true,
	linkcolor=black,
	citecolor=black,
}
\makeatother

\begin{document}
	% Capa e folha de rosto
\imprimircapa
\imprimirfolhaderosto
\clearpage

% Folha de aprovação
\imprimirfolhadeaprovacao{29/11/2018}

% Agradecimentos
\begin{agradecimentos}
%	Gostaria de agradecer primeiramente à minha família pelo enorme suporte e incentivo ao desenvolvimento deste trabalho, em especial nos momentos de maior dificuldade, no qual fizeram tudo o que estava ao seu alcance para me ajudar a superar os problemas que enfrentava.
%	A eles e à minha namorada, Fernanda, agradeço a compreensão e paciência durante este período extenuante de estudo e entrega.
%	
%	Às professoras Dra. Kátia Cavalca e Dra. Isolda Gianni de Lima, agradeço pela prontidão e disposição em me ajudar nos momentos em que eu não pude encontrar uma resposta para minhas dúvidas.
%	
%	Agradeço também à empresa Auttom Automação e Robótica, por propiciar a oportunidade de desenvolvimento pessoal com o tema deste trabalho, e pelo suporte que sempre deu à condução dos meus estudos.
\end{agradecimentos}

% Resumo em português
\begin{resumo}
	\SingleSpacing
	O emprego de mancais de rolamento se aplica desde sistemas mecânicos simples até o maquinário de alta complexidade e precisão.
	Sendo a análise de vibrações o meio utilizado para identificação precoce de falhas nestas unidades, é de grande importância determinar o comportamento deste componente em situação de defeito.
	Com este foco, este trabalho concentra-se no desenvolvimento de um modelo numérico não-linear de três graus de liberdade para a resposta de vibração de um rolamento com um defeito pontual no anel externo.
	Para isso, são empregados os conceitos da teoria \emph{hertziana} de deformação por contato, de modo a determinar a força sobre um elemento rolante quando este incide sobre o defeito.
	O valor de força é utilizado para modelar três perfis de pulso diferentes para a carga de impacto entre a esfera do rolamento e o defeito.
	Ao mesmo tempo, a influência do filme de fluido lubrificante no contato entre esferas e anéis é analisado sob os fundamentos da teoria de lubrificação \emph{elastohidrodinâmica}.
	O modelo desenvolvido é investigado quanto à sua estabilidade e convergência, observando o custo computacional das diferentes formas de pulso para a carga de impacto.
	A resposta de vibração simulada é comparada a dados de simulações e verificações experimentais de trabalhos anteriores no mesmo campo de aplicação, analisando-a no domínio do tempo e no domínio da frequência, para concluir quanto à consistência qualitativa do modelo.
	\vspace{\onelineskip}
	
	\noindent
	\textbf{Palavras-chave}: Vibração. Modelagem. Defeito. Rolamento.
\end{resumo}


% Resumo em inglês
\begin{resumo}[\normalsize\bfseries ABSTRACT]
	\SingleSpacing
	\begin{otherlanguage}{english}
		The application of rolling element bearings ranges from simple mechanical systems to high complexity and precision machinery.
		Since the vibration analysis is the main way to early identify failure in these units, it is of great importance to determine the vibrational pattern of this component in a defect situation.
		An experimental approach for the analysis would offer a strict condition for the problem investigation.
		For this reason, the construction of a numeric model is considered to be a more extensive way to understand the phenomenon, while still eliminating interferences caused by the test system.
		With that in focus, the present work has as objective to model the vibration of damaged rolling element bearings and identify the main characteristics of this behavior.
		\vspace{\onelineskip}
		
		\noindent
		\textbf{Keywords}: Vibration. Model. Defect. Bearing.
	\end{otherlanguage}
\end{resumo}
% Listas de figuras, quadros, tabelas e siglas
\listoffigures*
\cleardoublepage

%\listofquadros*
%\cleardoublepage

\listoftables*
\cleardoublepage

\listofsiglas*
\cleardoublepage

\listofsimbolos*
\cleardoublepage

% Sumário
\tableofcontents*
\cleardoublepage

	\textual % Comando que define que os elementos textuais começaram, para incluir a numeração de páginas.
	

	\chapter{Introdução}
	O capítulo a seguir situa o leitor em relação aos fatores técnicos e econômicos que motivaram a elaboração deste trabalho. Também são apresentados o ambiente onde foi realizado, seguido por seus objetivos. A disposição do conteúdo deste documento também é incluída nesta parte.
	
	\section{Justificativa}
	Nos últimos anos, o aumento da oferta de produtos na área de instrumentação e coleta de dados possibilitou à indústria o crescimento do uso de técnicas de manutenção preditiva para análise de condições dos equipamentos. Entre as técnicas de manutenção preditiva, o monitoramento baseado em vibrações (\textit{Vibration-Based Maintenance} ou simplesmente VBM) confere a vantagem de indicar mudanças nas condições de uma máquina em um estágio inicial de uma falha ou desgaste\cite{bib:al-najjar}.
	
	\postextual
	
	\bibliography{Bibliografia}
		
\end{document}