\newcommand{\unidademassa}{[kg]}
\newcommand{\unidadeamortecimento}{[N\cdot s/m]}
\newcommand{\unidaderigidez}{[N/m]}
\newcommand{\unidadeforca}{[N]}
\newcommand{\unidadeposicao}{[m]}
\newcommand{\unidadevelocidade}{[m/s]}
\newcommand{\unidadeaceleracao}{[m/s\textsuperscript{2}]}
\newcommand{\unidadetempo}{[s]}
\newcommand{\unidadeenergia}{[J]}
\newcommand{\unidadeangulo}{[rad]}
\newcommand{\unidadevelocidadeangular}{[rad/s]}
\newcommand{\unidadetorque}{[N\cdot m]}
\newcommand{\unidadearea}{[m\textsuperscript{2}]}
\newcommand{\semunidade}{[-]}

% Medidas elementares e propriedades
\newglossaryentry{s:graus-liberdade}{name={\ensuremath{n}},description={número de graus de liberdade do sistema \semunidade}}
\newglossaryentry{s:tempo}{name={\ensuremath{t}},description={tempo \unidadetempo}}
\newglossaryentry{s:periodo}{name={\ensuremath{\tau}},description={período \unidadetempo}}
\newglossaryentry{s:area}{name={\ensuremath{A}},description={área \unidadearea}}
\newglossaryentry{s:modulo-elast}{name={\ensuremath{E}},description={módulo de elasticidade [Pa] }}
\newglossaryentry{s:segundo-momento-area}{name={\ensuremath{I}},description={segundo momento de área [m\textsuperscript{4}]}}

% Massa
\newglossaryentry{s:massa-espec}{name={\ensuremath{\rho}},description={massa específica por unidade de comprimento $ [kg/m] $}}
\newglossaryentry{s:massa-elemento}{name={\ensuremath{m_i}},description={massa de um elemento do sistema \unidademassa}}

% Posição, deslocamento, velocidade
\newglossaryentry{s:posicao-tempo}{name={\ensuremath{x(t)}},description={posição em função do tempo \unidadeposicao}}
\newglossaryentry{s:posicao-freq}{name={\ensuremath{X(\omega)}},description={posição em função da frequência \unidadevelocidadeangular}}
\newglossaryentry{s:coord-general}{name={\ensuremath{q_i}},description={posição em coordenadas generalizadas \semunidade}}
\newglossaryentry{s:desloc-elemento}{name={\ensuremath{x_i}},description={deslocamento de um elemento do sistema \unidadeposicao}}
\newglossaryentry{s:desloc-vertical}{name={\ensuremath{w}},description={deslocamento vertical do corpo elástico \unidadeposicao}}
\newglossaryentry{s:constante-onda}{name={\ensuremath{\mathit{c}}},description={velocidade de propagação da onda \unidadevelocidade}}

% Frequência
\newglossaryentry{s:freq}{name={\ensuremath{\omega}},description={frequêcia \unidadevelocidadeangular}}
\newglossaryentry{s:freq-natural}{name={\ensuremath{\omega_n}},description={frequência natural \unidadevelocidadeangular}}
\newglossaryentry{s:freq-natural-modo}{name={\ensuremath{\omega_i}},description={frequência natural do i-ésimo modo normal \unidadevelocidadeangular}}

% Vetores
\newglossaryentry{s:vetor-desloc}{name={\ensuremath{\vec{x}}},description={vetor de deslocamentos \unidadeposicao}}
\newglossaryentry{s:vetor-vel}{name={\ensuremath{\dot{\vec{x}}}},description={vetor de velocidades \unidadevelocidade}}
\newglossaryentry{s:vetor-acel}{name={\ensuremath{\ddot{\vec{x}}}},description={vetor de acelerações \unidadeaceleracao}}
\newglossaryentry{s:vetor-forcas}{name={\ensuremath{\vec{F}}},description={vetor de forças externas \unidadeforca}}
\newglossaryentry{s:vetor-veloc-general}{name={\ensuremath{\dot{\vec{q}}}},description={vetor de velocidades generalizadas \unidadevelocidade}}

% Matrizes e seus componentes
\newglossaryentry{s:matriz-massa}{name={\ensuremath{[m]}},description={matriz de massa \unidademassa}}
\newglossaryentry{s:matriz-massa-diag}{name={\ensuremath{[m_d]}},description={matriz diagonal de massa \unidademassa}}
\newglossaryentry{s:matriz-amort}{name={\ensuremath{[c]}},description={matriz de amortecimento \unidadeamortecimento}}
\newglossaryentry{s:matriz-rigidez}{name={\ensuremath{[k]}},description={matriz de rigidez \unidaderigidez}}
\newglossaryentry{s:coef-massa}{name={\ensuremath{m_{ij}}},description={coeficiente de massa \unidademassa}}
\newglossaryentry{s:coef-amortecimento}{name={\ensuremath{c_{ij}}},description={coeficiente de amortecimento \unidadeamortecimento}}
\newglossaryentry{s:coef-rigidez}{name={\ensuremath{k_{ij}}},description={coeficiente de rigidez \unidaderigidez}}

% Força e momento
\newglossaryentry{s:forca}{name={\ensuremath{F}},description={força \unidadeforca}}
\newglossaryentry{s:forca-elemento}{name={\ensuremath{F_i}},description={força aplicada sobre um elemento do sistema \unidadeforca}}
\newglossaryentry{s:forca-general}{name={\ensuremath{Q}},description={força generalizada \unidadeforca}}
\newglossaryentry{s:forca-naoconserv}{name={\ensuremath{Q_i^{(n)}}},description={força generalizada não-conservativa do i-ésimo grau de liberdade \unidadeforca}}
\newglossaryentry{s:forca-cisalhamento}{name={\ensuremath{C}},description={força de cisalhamento \unidadeforca}}
\newglossaryentry{s:momento-fletor}{name={\ensuremath{M}},description={momento fletor \unidadetorque}}
\newglossaryentry{s:tensao-corda}{name={\ensuremath{P}},description={força tensora \unidadeforca}}
\newglossaryentry{s:angulo-tensao}{name={\ensuremath{\theta}},description={ângulo da corda \unidadeangulo}}

% Energia e trabalho
\newglossaryentry{s:energ-pot-total}{name={\ensuremath{V}},description={energia potencial total \unidadeenergia}}
\newglossaryentry{s:energ-cin-total}{name={\ensuremath{K}},description={energia cinética total \unidadeenergia}}
\newglossaryentry{s:energ-pot-elem}{name={\ensuremath{V_i}},description={energia de deformação de um elemento elástico \unidadeenergia}}
\newglossaryentry{s:energ-cin-elem}{name={\ensuremath{K_i}},description={energia cinética do elemento \unidadeenergia}}
\newglossaryentry{s:lagrangiano}{name={\ensuremath{L}},description={lagrangiano \unidadeenergia}}
\newglossaryentry{s:trabalho-coord-general}{name={\ensuremath{U}},description={trabalho \unidadeenergia}}

% Auxiliares em cálculos
\newglossaryentry{s:sol-parcial-pos}{name={\ensuremath{W}},description={solução parcial, dependente da posição x [m\textsuperscript{1/2}]}}
\newglossaryentry{s:sol-parcial-tempo}{name={\ensuremath{T}},description={solução parcial, dependente do tempo [m\textsuperscript{1/2}]}}
\newglossaryentry{s:coef-fourier-compl}{name={\ensuremath{\alpha_n}},description={harmônica da série de Fourier em notação complexa \semunidade}}


% Elementos agrupados: criação de uma nova chave para o glossaries que possibilitará exibir na lista de símbolos as variáveis separadas por vírgula e no texto, separadas com a palavra "e" no último termo.
\glsaddkey{expansao}
{\glsentrytext}
{\glsentryexpansao}
{\Glsentryexpansao}
{\glsexpansao}
{\Glsexpansao}
{\GLSexpansao}

\newglossaryentry{s:const-dif-tempo}{name={\ensuremath{\mathit{A}, \mathit{B}}},description={constantes da solução particular da equação diferencial para T \semunidade},expansao={\ensuremath{\mathit{A}} e \ensuremath{\mathit{B}}}}

\newglossaryentry{s:const-dif-desloc}{name={\ensuremath{C_1, C_2, C_3, C_4}},description={constantes da solução particular da equação diferencial para W \semunidade},expansao={\ensuremath{C_1, C_2, C_3} e \ensuremath{C_4}}}

\newglossaryentry{s:coef-fourier}{name={\ensuremath{a_0, a_n, b_n}},description={harmônicas da série de Fourier \semunidade},expansao={\ensuremath{a_0, a_n} e \ensuremath{b_n}}}

\newglossaryentry{s:coef-fourier-int}{name={\ensuremath{a_\omega, b_\omega}},description={harmônicas da integral de Fourier \semunidade},expansao={\ensuremath{a_\omega} e \ensuremath{b_\omega}}}